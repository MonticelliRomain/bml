\documentclass[a4paper]{article}
\usepackage[T1]{fontenc}
\usepackage[utf8]{inputenc}
\usepackage{newcent}
\usepackage{helvet}
\usepackage{graphicx}
\usepackage[pdftex, pdfborder={0 0 0}]{hyperref}
\frenchspacing

\include{bml}
\title{Système Zweiffel}
\author{Monticelli's/ Dehaye's}
\begin{document}
\maketitle
\tableofcontents

\section{Questions}

\begin{itemize}
\item Ouverture de 2\s\ ?

\end{itemize}

\begin{itemize}
\item 1\c\ 3è ?

\end{itemize}

\begin{itemize}
\item Enchères d'essai naturelles !

\end{itemize}

\begin{itemize}
\item 1\h -2\c -2\h\ Non forcing ?

\end{itemize}

\begin{itemize}
\item 1NT-3x TDC en Texas ?

\end{itemize}

\begin{itemize}
\item Après ouverture de 2\d\ : contre = des \d\ ?

\end{itemize}

\section{Enchères}

\begin{bidtable}
1\c--\+\\
1\d \> Walsh\\
2x \> fort\+\\
3\c \> beaux \c \\
reste \> arrêts\-\\
3x \> faible - QT9xxxx NV\-
\end{bidtable}

\begin{bidtable}
1M--\+\\
1NT\\
2\c \> Relais pour 2\d , mains limites ou limitées à la manche avec 5m\\
2\d \> FM Ensuite c'est plus l'ouvreur qui se décrit\\
2NT \> Relais \c \-
\end{bidtable}

\begin{bidtable}
1m-1M--\+\\
2NT\+\\
3\c \> Relais\\
3\d \> courte - demande de courte pp\\
3M \> NF\-\\
3\c\+\\
3x \> arrêts\\
3M' \> forcing, option pour jouer dans la couleur (?)\-\\
3M' \> 6/5 13-16\-
\end{bidtable}

\begin{bidtable}
1\s-(P)-2NT-(3\c)--\+\\
3x \> court \c \\
Dbl \> 3 cartes à \c \-
\end{bidtable}

\begin{bidtable}
1\s-1NT-2\c--\+\\
2\d \> 5 cartes à \h , 8-10H\\
2\h \> 6 cartes à \h , 6-9H (à confirmer)\-
\end{bidtable}

\begin{bidtable}
1NT\+\\
2\c \> Stayman faible 3 paliers (plutôt 1-5-4-3 ou 1-4-4-4 ?)\+\\
2\d\+\\
2\h\s \> 5-4 invit\-\-\\
2\s \> mineure inconnue\+\\
2NT \> intérêt pour les deux\-\\
2NT \> Naturel\\
3X \> Transfert TDC\+\\
Rect \> Main pas entièrement nulle\-\\
4\c \> BIC min TDC\\
4\d \> BIc maj pour la manche\\
(P)\+\\
2\c\+\\
(D)\+\\
Rdbl \> 4 belles cartes\\
Pass \> attend le X, ensuite, réponse avec M inversées !!\\
X \> naturel avec l'arrêt\-\-\-\-
\end{bidtable}

\begin{bidtable}
(1NT)--\\
2\c \> Landy\+\\
(D)\+\\
Pass \> des \c \\
Rdbl \> Choisis ta M\\
2\d\h\s \> Naturel\-\-
\end{bidtable}

Michaëls non précisés

\begin{bidtable}
1x--\+\\
(1/2y)\+\\
2NT \> 5+/5+ des 2 autres\-\-
\end{bidtable}

\begin{bidtable}
(1\c)--\+\\
Pass\+\\
(1\h) \> Du \s \+\\
Dbl \> contre sur les \s \\
1\s \> irrégulier pas trop fort\\
2\c \> 5/5 plus constructif\-\-\-
\end{bidtable}

\begin{bidtable}
(2\h)--\+\\
Dbl\+\\
(P)\+\\
2NT \> Modérateur\\
\>(P)\+\\
3\c\+\\
3\h \> Demi-arrêt\\
3\s \> invit avec 4\s \\
3NT \> FM avec 4\s \-\-\-\-\-
\end{bidtable}

\begin{bidtable}
(2\d)-- \> Multi\+\\
Dbl\+\\
(2\h)\+\\
Dbl \> 4 belles cartes (à \h\ ?) ou ambition\\
2NT \> option pour jouer en mineure\\
3\h \> cue-bid\-\-\-
\end{bidtable}

\begin{bidtable}
(2\d)-- \> Faible à \h \+\\
Dbl \> d'appel = Contre sur 2\h\ faible\\
2\h \> 5\s /5m\+\\
(2/3X)\+\\
Dbl \> punitif\-\-\-
\end{bidtable}

Good/Bad :

\begin{bidtable}
1\c--\+\\
(P)\+\\
1\s\+\\
2\h\+\\
2NT \> Zone faible (sans arrêt ?) avec des longs \c \\
3\c \> Zone forte (15+) avec des longs \c \-\-\-\-
\end{bidtable}

Pas de (sur)contre de soutien, toujours des mains fortes

\section{Jeu de la carte}

Smith par les grosses.

\subsection{Sans-atout}

\subsubsection{Entame}

Entame quatrième meilleure avec au moins le valet.

Le 8 est toujours négatif.

Le 9 est toujours positif ou 98(/98x ?).

Si couleur du partenaire : parité sauf sur 1\c .

As = ARxx ou ARx => Appel avec les petites.

R = RD10xx ou ARVxx et on débloque sauf si single au mort.

D peut être RD10x.

Si single au mort => appel italien (après l'entame de l'as à SA donc ? Une impaire demande de continuer la couleur entamée)

\subsubsection{En troisième main}

Petit = positif sur l'entame d'un honneur (juste à SA ?)

\subsection{A sans-atout et à la couleur}

Dans 4 cartes :

\begin{itemize}
\item Si le partenaire joue, parité stricte (la troisième !!)

\end{itemize}

\begin{itemize}
\item Si l'adversaire joue, la troisième indique une qualité de couleur, la deuxième pas.

\end{itemize}

\subsection{À la couleur}

\begin{itemize}
\item Entame de l'as : on appelle (par les petites) avec la dame

\end{itemize}

\subsection{Au switch}

\begin{itemize}
\item Valet = RDV ou V10

\end{itemize}

\begin{itemize}
\item 10 = celle au-dessus et une autre ou 109

\end{itemize}

\subsection{Cours du jeu}

\begin{itemize}
\item Si, au cours du jeu, une grosse carte anormale est jouée, un switch est à trouver.

\end{itemize}

\begin{itemize}
\item Défausse italienne

\end{itemize}

\end{document}
