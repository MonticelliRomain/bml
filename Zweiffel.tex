\documentclass[a4paper]{article}
\usepackage[T1]{fontenc}
\usepackage[utf8]{inputenc}
\usepackage{newcent}
\usepackage{helvet}
\usepackage{graphicx}
\usepackage[pdftex, pdfborder={0 0 0}]{hyperref}
\frenchspacing

\include{bml}
\title{Système Zweiffel}
\author{Monticelli's/ Dehaye's}
\begin{document}
\maketitle
\tableofcontents

\section{Questions}

\begin{itemize}
\item Ouverture de 2\s\ ?

\end{itemize}

\begin{itemize}
\item 1\c\ 3è ?

\end{itemize}

\begin{itemize}
\item Enchères d'essai naturelles !

\end{itemize}

\begin{itemize}
\item 1\h -2\c -2\h\ Non forcing ?

\end{itemize}

\begin{itemize}
\item 1NT-3x TDC en Texas ?

\end{itemize}

\begin{itemize}
\item Après ouverture de 2\d\ : contre = des \d\ ?

\end{itemize}

\section{Enchères}

\begin{bidtable}
1\c--\+\\
1\d \> Walsh\\
2x \> fort\+\\
3\c \> beaux \c \\
reste \> arrêts\-\\
3x \> faible - QT9xxxx NV\-
\end{bidtable}

\begin{bidtable}
1M--\+\\
1NT\\
2\c \> Relais pour 2\d , mains limites ou limitées à la manche avec 5m\\
2\d \> FM Ensuite c'est plus l'ouvreur qui se décrit\\
2NT \> Relais \c \-
\end{bidtable}

\begin{bidtable}
1m-1M--\+\\
2NT\+\\
3\c \> Relais\\
3\d \> courte - demande de courte pp\\
3M \> NF\-\\
3\c\+\\
3x \> arrêts\\
3M' \> forcing, option pour jouer dans la couleur (?)\-\\
3M' \> 6/5 13-16\-
\end{bidtable}

\begin{bidtable}
1\s-(P)-2NT-(3\c)--\+\\
3x \> court \c \\
Dbl \> 3 cartes à \c \-
\end{bidtable}

\begin{bidtable}
1\s-1NT-2\c--\+\\
2\d \> 5 cartes à \h , 8-10H\\
2\h \> 6 cartes à \h , 6-9H (à confirmer)\-
\end{bidtable}

\begin{bidtable}
1NT\+\\
2\c \> Stayman faible 3 paliers (plutôt 1-5-4-3 ou 1-4-4-4 ?)\+\\
2\d\+\\
2\h\s \> 5-4 invit\-\-\\
2\s \> mineure inconnue\+\\
2NT \> intérêt pour les deux\-\\
2NT \> Naturel\\
3X \> Transfert TDC\+\\
Rect \> Main pas entièrement nulle\-\\
4\c \> BIC min TDC\\
4\d \> BIc maj pour la manche\\
(P)\+\\
2\c\+\\
(D)\+\\
Rdbl \> 4 belles cartes\\
Pass \> attend le X, ensuite, réponse avec M inversées !!\\
X \> naturel avec l'arrêt\-\-\-\-
\end{bidtable}

\begin{bidtable}
(1NT)--\\
2\c \> Landy\+\\
(D)\+\\
Pass \> des \c \\
Rdbl \> Choisis ta M\\
2\d\h\s \> Naturel\-\-
\end{bidtable}

Michaëls non précisés

\begin{bidtable}
1x--\+\\
(1/2y)\+\\
2NT \> 5+/5+ des 2 autres\-\-
\end{bidtable}

\begin{bidtable}
(1\c)--\+\\
Pass\+\\
(1\h) \> Du \s \+\\
Dbl \> contre sur les \s \\
1\s \> irrégulier pas trop fort\\
2\c \> 5/5 plus constructif\-\-\-
\end{bidtable}

\begin{bidtable}
(2\h)--\+\\
Dbl\+\\
(P)\+\\
2NT \> Modérateur\\
\>(P)\+\\
3\c\+\\
3\h \> Demi-arrêt\\
3\s \> invit avec 4\s \\
3NT \> FM avec 4\s \-\-\-\-\-
\end{bidtable}

\begin{bidtable}
(2\d)-- \> Multi\+\\
Dbl\+\\
(2\h)\+\\
Dbl \> 4 belles cartes (à \h\ ?) ou ambition\\
2NT \> option pour jouer en mineure\\
3\h \> cue-bid\-\-\-
\end{bidtable}

\begin{bidtable}
(2\d)-- \> Faible à \h \+\\
Dbl \> d'appel = Contre sur 2\h\ faible\\
2\h \> 5\s /5m\+\\
(2/3X)\+\\
Dbl \> punitif\-\-\-
\end{bidtable}

Good/Bad :

\begin{bidtable}
1\c--\+\\
(P)\+\\
1\s\+\\
2\h\+\\
2NT \> Zone faible (sans arrêt ?) avec des longs \c \\
3\c \> Zone forte (15+) avec des longs \c \-\-\-\-
\end{bidtable}

Pas de (sur)contre de soutien, toujours des mains fortes

\section{Ouverture de 2\pdfc}

\subsection{Définition}

\begin{itemize}
\item FAIBLE en \d\ (constructif VUL et en 1ère et 2ème NV) et 11-13 en 4ème

\item 22-24 BAL

\item GAMBLING solide en \c\ 

\item FM sauf Bicolores Majeurs

\end{itemize}

\begin{bidtable}
2\c--\+\\
2\d \> relais\+\\
Pass \> FAIBLE \d \\
2\h \> 24+ balancé\\
\>Bicolore avec au moins 4\h \\
\>Unicolore \h\ balancé 22-25\+\\
2\s \> relais\+\\
2NT \> 25H et + (développements comme sur 2N)\\
3\c \> 5+\c , =4\h \\
3\d \> 5+\d , =4\h \\
3\h \> 5+\h , 4+\c \\
3\s \> 5+\h , 4+\d \\
3NT \> Unicolore \h\ Gambling (style régulier avec 6/7\h\ plein et 2 As)\\
4\c\d \> 6\h /5m moins fort que 3\h\ suivi de 4\c\ (5+/5)\\
\>Ex: \s\ 2 \h\ ADV987 \d\ 8 \c\ ARDT2\-\-\\
2\s \> Bicolore avec 4+\s \\
\>Unicolore \s\ balancé 22-25\+\\
2NT \> relais\+\\
3\c \> 5+\c , =4\s \\
3\d \> 5+\d , =4\s \\
3\h \> 5+\s , 4+\c \\
3\s \> 5+\s , 4+\d \\
3NT \> Unicolore \s\ Gambling\\
4\c\d \> 6\s /5m moins fort que 3\s\ suivi de 4\c\ (5+/5)\-\-\\
2NT \> 22+ - 24H\\
3\c\d \> Unicolores ou bicolore mineur (la collante est ambigue)\\
3\h\s \> Unicolores irréguliers\\
3NT \> Gambling \c \\
4\c\d \> Bicolores 6/5m \h\ (6\h -5\c /\d\ + faible que 2\c -2\d -2\h -2\s -4\c /\d )\\
4\h\s \> Bicolores 6/5m \s\ (pareil)\-\\
2\h\s3\c \> F1T (pas forcing en paires !)\\
\>Nouvelle couleur GF\+\\
2NT \> Max misfit\\
3\d \> Min\\
3x/4x \> Fit, Splinters mains faibles\-\\
2NT \> Relais F1 avec espoir de manche en face d’un faible à \d \+\\
3\c \> faible \d\ et une courte\+\\
3\d \> NF courte s’annonce par paliers si max\\
\>(=> Passe si min du faible, annonce la courte par palier si max du faible)\-\\
3\d \> faible \d\ sans courte\\
3\h \> Max pièce \c /\h , pas de courte\+\\
3P \> relais\+\\
3NT \> court \c \\
4\c \> court \h \-\-\\
3\s \> Max force sans courte\\
3NT \> beaux \d\ ARD ou ARVT sans courte\\
4X \> GF\-\\
3\d \> 7-11H (Forcing > 4NT si main forte, mêmes dvpts)\\
3\h\s4\c \> NAT + Fit\-
\end{bidtable}

\subsection{Avec les mains TRICOLORES, on reparle toujours sur 3NT.}

Ex : \s\ - \h\ ADV8 \d\ ARD98 \c\ ARV10
(2\c -2\d -2\h -2\s -3\d -3NT-4\c -...)

\subsection{Après interventions}

\begin{itemize}
\item Contre punitif

\item Toute enchère F1T

\item 2NT R fort

\item Cue demande d'arret

\end{itemize}

\subsection{Après contre}

\begin{itemize}
\item Passe des \c !

\item Surcontre FORT ou UNICOLORE

\item 2\d\ ambigu

\item Nouvelle couleur F1T

\end{itemize}

\section{Ouverture de 2\pdfd}

\subsection{Définition}

\begin{itemize}
\item Unicolore FI

\item Bicolore Majeur FI ou FM

\end{itemize}

\begin{bidtable}
2\d-2\h\\
2\s \> Montre un bicolore majeur \textbf{+ long à \s }\+\\
2NT \> Relais\+\\
3\h \> 5/5 MAJ FI 18-21 PH\\
3\s \> 6\s /4\h\ FI 18-21 PH\\
3NT \> 5422 FM\\
4\c\d \> Tricolore parfait\\
4\h \> 65 FM\\
4\s \> 74 FM\\
3\d \> 5431 FM\\
3\c \> FM ambigu\+\\
3\d \> Relais\+\\
3\h \> 55 FM\\
3\s \> 64 FM\\
3NT \> 5413 FM\-\-\-\\
3\c\d \> Aucun intérêt pour les majeures, naturel (pas belle 6e)\\
3\h\s \> TDC\\
4\h\s \> NF\-
\end{bidtable}

\begin{bidtable}
2\d-2\h\\
2NT \> Montre un bicolore majeur \textbf{+ long à \h }\+\\
3\c \> Relais\+\\
3\d \> Résidu mineur 31 ou 30\+\\
3\h \> Relais\+\\
3\s \> 4513\\
3NT \> 4531\\
4X \> 4603 ou 4630\-\-\\
3\h\\
3\s \> 65 FM\\
3NT \> 5422 FM\\
4\c\d \> Tricolore parfait\\
4\h \> 74 FM\-\-
\end{bidtable}

\begin{bidtable}
2\d-2\h\\
3\c\d\h\s \> ACOL naturel (=FI) Collante ambigue, éventuellement pour tenter de jouer 3NT de la main forte\+\\
3\c\+\\
3\s \> Ambigu\+\\
3NT \> pour jouer\+\\
4\h \> TDC avec cue \s \\
4NT \> Quantitatif\-\-\\
4X \> Cue, TDC fit \h \-\\
3NT \> 9-10 plis en \d\ avec au moins 2 grosses pièces extérieures\-
\end{bidtable}

\begin{bidtable}
2\d\+\\
2\s \> Naturel 5 cartes avec deux gros honneurs ou 5/5 correct\\
3\c\d\h\s \> Transferts pour unicolores Q+ avec 2 gros honneurs\\
\>Le soutien ou la rectification du texas montre un intérêt pour la couleur du répondant\\
2NT \> 5/5 Mineur FM\-
\end{bidtable}

\subsection{Après intervention en 2e}

\subsubsection{Contre}

\begin{itemize}
\item Punitif si intervention au palier de 2

\item Appel si intervention au palier de 3 ou plus

\end{itemize}

\subsubsection{Passe RAS}

Ensuite, contre de l'ouvreur est punitif au niveau de 2 et montre le bicolore majeur ou une demande d'arrêt au niveau de 3.

\subsubsection{3\pdfc\pdfd\pdfh\pdfs\ Transferts (Transfert cue - Tricolore court)}

\subsubsection{2NT Bicolore mineur}

\subsubsection{Après 2\pdfd\ - X}

\begin{itemize}
\item Passe : envie de jouer

\item Tout le reste sauf 2\h\ : Naturel

\item XX : Naturel (4 belles cartes, pas 5 ou 6)

\end{itemize}

\subsection{Après intervention en 4e}

\begin{itemize}
\item PASSE, CUE-BID et ENCHERES A SAUT montrent le BICOLORE MAJEUR JUSQU’À 3\s .

\end{itemize}

\begin{itemize}
\item CONTRE demande l’arrêt

\end{itemize}

\subsection{Corollaires (à vérifier avec Gazilli !)}

\begin{bidtable}
1\s-1NT\\
3\h \> montre un 5=/4= dans la zone 18-21H
\end{bidtable}

\begin{bidtable}
1\s-1NT\\
4\h \> montre un 6/5 de 14 à 17H
\end{bidtable}

\begin{bidtable}
1\h-1NT\\
3\s \> montre un 6/5 de 14 à 17H, c’est NF !\\
* \> Jeu de la carte
\end{bidtable}

Smith par les grosses.

\subsection{Sans-atout}

\subsubsection{Entame}

Entame quatrième meilleure avec au moins le valet.

Le 8 est toujours négatif.

Le 9 est toujours positif ou 98(/98x ?).

Si couleur du partenaire : parité sauf sur 1\c .

As = ARxx ou ARx => Appel avec les petites.

R = RD10xx ou ARVxx et on débloque sauf si single au mort.

D peut être RD10x.

Si single au mort => appel italien (après l'entame de l'as à SA donc ? Une impaire demande de continuer la couleur entamée)

\subsubsection{En troisième main}

Petit = positif sur l'entame d'un honneur (juste à SA ?)

\subsection{A sans-atout et à la couleur}

Dans 4 cartes :

\begin{itemize}
\item Si le partenaire joue, parité stricte (la troisième !!)

\end{itemize}

\begin{itemize}
\item Si l'adversaire joue, la troisième indique une qualité de couleur, la deuxième pas.

\end{itemize}

\subsection{À la couleur}

\begin{itemize}
\item Entame de l'as : on appelle (par les petites) avec la dame

\end{itemize}

\subsection{Au switch}

\begin{itemize}
\item Valet = RDV ou V10

\end{itemize}

\begin{itemize}
\item 10 = celle au-dessus et une autre ou 109

\end{itemize}

\subsection{Cours du jeu}

\begin{itemize}
\item Si, au cours du jeu, une grosse carte anormale est jouée, un switch est à trouver.

\end{itemize}

\begin{itemize}
\item Défausse italienne

\end{itemize}

\end{document}
