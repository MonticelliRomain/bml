\documentclass[a4paper]{article}
\usepackage[T1]{fontenc}
\usepackage[utf8]{inputenc}
\usepackage{newcent}
\usepackage{helvet}
\usepackage{graphicx}
\usepackage[pdftex, pdfborder={0 0 0}]{hyperref}
\frenchspacing

\include{bml}
\title{Système Monique}
\author{Thomas Monticelli}
\begin{document}
\maketitle
\tableofcontents

\section{Introduction}

\section{Ouverture de 1m}

\subsection{Le Walsh}

À partir de 2 cartes. Le Walsh, basé sur le principe ``la majeure d'abord'', consistant à ne présenter les \d\ que dans les mains fortes ou sans majeures, est utilisé.
En cas de fit fort sans majeure, le fit mineur inversé est utilisé.

\begin{bidtable}
1\c---\\
1\d \> 3 possibilités :\\
\>5+\d\ et inv+\\
\>5+\d , 3-\h \s\ et 6-10H\\
\>BAL 5-7H et 3-\h \s \\
1\h \> \h /4+ (possibilité de \d\ plus longs que \h\ si <11h)\\
1\s \> \s /4+ (possibilité de \d\ plus longs que \s\ si <11h)\\
1NT \> 8-10 3-\h \s \\
2\c \> FMI : fit \c , 10+ (voir développements plus loin)\\
2\d \> 6+\d\ (et 2GH), 18H+\\
2\h \> 5+\s\ 4+\h , 5-9H, idem sur 1\d \\
2\s \> 5+\s\ 4+\h , 10-11H, idem sur 1\d \\
2NT	 \> 11-12 BAL (3\c\ NF, reste FM)\\
3\c	 \> 5\c , main irrégulière ~5-9 H\\
3X	 \> Barrage 7 cartes\\
3NT	 \> 12-14 BAL
\end{bidtable}

\subsection{La séquence 1\pdfc\ - 1\pdfd}

\begin{bidtable}
1\c-1\d;\\
1M \> 12-17 H, M/4 Main irrégulière (5\c )\\
1\h\+\\
1\s \> 4eme couleur forcing\\
\>(L’ouvreur réagit comme sur une enchère NAT mais ne peut pas dire 4\s ) \+\\
2\s \> 12-14\\
3\s \> 15+\+\\
3NT \> SO\\
4X \> cue fit \s \-\-\\
2\s \> 5\d 4\s\ FM\-\\
1NT \> 12-14 H, BAL\+\\
2\c \> checkback stayman avec des mains INV+\+\\
2X\+\\
3\c \> 5\d 5\c\ FM (fit différé)\\
3\d \> 6+\d\ FM\-\\
2\d \> Pas de M/4\\
2\h \> 4\h\ (avec ou sans 4\s )\\
2\s \> 4\s \-\\
3\c \> 5\d 4\c\ INV\\
3\d \> 6\d\ INV\-
\end{bidtable}

\subsection{La séquence 1\pdfc\ - 1M}

\begin{bidtable}
1\c-1M;\\
1NT\+\\
2\d \> 5\d 4M faible to play\-\\
2\c\+\\
2\d \> 3eme couleur forcing\-
\end{bidtable}

\subsection{La séquence 1m - 1M}

\begin{bidtable}
1m-1M;\\
2NT \> pas de fit\+\\
3\c \> réponses dans l'ordre, style Baron\+\\
3\d \> 5m\\
3M \> 3cM\\
3M' \> 4cM'\\
3NT \> aucun des cas précédents\-\\
3\d \> fit dans la m (FM ou TDC)\\
3M \> NAT NF\\
3M' \> BIC (55 ou 65) FM\\
3NT \> SO\\
4m/M' \> cue avec atout M\\
4\h \> (sur 1\s ) BIC NF\\
4M \> SO\\
4NT \> quantitatif\-\\
3NT \> fit main régulière\\
4(3)M’ \> splinter\\
4m’ \> splinter\\
4m \> 5+m 4+M\\
4M \> main extrêmement irrégulière avec peu de points d’honneurs
\end{bidtable}

\subsection{La séquence 1m - 2\pdfh}

\begin{bidtable}
1m-2\h \> 5+\s\ 4+\h\ 5-9\+\\
2\s \> Nat NF\\
2NT \> Relay\+\\
3\c \> Mini\+\\
3\d \> Relay\+\\
3\h \> 54\\
3\s \> 55\\
3NT \> 64\-\-\\
3\d \> Max 54\\
3\h \> Max 55\\
3\s \> Max 64\-\\
3\c \> Nat NF\\
3\d \> Nat NF\\
3M \> Invit\\
3NT \> Nat\-
\end{bidtable}

\subsection{La séquence 1m - 2\pdfs}

Cfr 1m-2\h 

\subsection{La séquence 1\pdfd\ - 2\pdfc}

\begin{bidtable}
1\d-2\c;\\
2\d \> irrégulier 11-15, BAL 12-13\\
2M \> inversée NAT\\
2NT \> BAL 14 ou 18-19\+\\
3\c \> relais\+\\
3\d \> 18-19, le reste NAT 14\-\-\\
3\c \> 16+ PJS Fit 4ème ou 18-19 et 3 cartes\\
3\d \> 6+\d , 16+ problème dans une ou deux M\\
3M \> 65 sans inversée\\
3NT \> 6+\d , 16+ avec les arrêts M
\end{bidtable}

\subsection{Le fit mineur inversé}

\begin{bidtable}
1\c-2\c \> --\\
2NT \> 12-14- BAL\+\\
3\c \> SO\\
3\d \> arrêt \d \+\\
3\s \> arrêt \s \+\\
4\c \> SO, pas d'arrêt \h \-\-\\
Autre \> : arrêt\\
Sauts \> : courtes\+\\
4NT \> : to play (des points dans les courtes)\-\-\\
2\d \> 14+, arrêt \d , arrêts \h /\s\ possibles\+\\
2\h \> arrêt \h\ et pas mini\\
2\s \> arrêt \s , pas d'arrêt \h\ et pas mini\\
2NT \> arrêts des autres couleurs mais mini\\
3\c \> pas d'arrêts des autres couleurs et mini\\
3\d \> singleton (ou chicane) \d \-\\
2\h \> 14+, arrêt \h , pas d'arrêt \d , arrêt \s\ possible\\
2\s \> 14+, arrêt \s , pas d'arrêts \h /\d \\
3\c \> 12-14- irrégulier\+\\
Sauts \> : courtes\+\\
4NT \> : to play (des points dans les courtes)\-\-\\
3\d \> 14+, court \d , \c\ mieux que Vxxx\\
3\h \> 14+, court \h , \c\ mieux que Vxxx\\
3\s \> 14+, court \s , \c\ mieux que Vxxx\\
3NT \> 18-19 Bal\\
4\d \> 15-18, chicane \d , beaux \c \\
4\h \> 15-18, chicane \h , beaux \c \\
4\s \> 15-18, chicane \s , beaux \c 
\end{bidtable}

\begin{bidtable}
1\d-2\d \> --\\
2NT \> 12-14- BAL\+\\
3\c \> arrêt \c \+\\
3\s \> arrêt \s \+\\
4\d \> SO, pas d'arrêt \h \-\-\\
3\d \> SO\\
Autre \> : arrêt\\
Sauts \> : courtes\+\\
4NT \> : to play (des points dans les courtes)\-\-\\
2\h \> 14+, arrêt \h , arrêts \s /\c\ possibles\+\\
2\s \> arrêt \s\ et pas mini\\
2NT \> arrêts des autres couleurs mais mini\\
3\c \> arrêt \c , pas d'arrêt \s\ et pas mini\\
3\d \> pas d'arrêts des autres couleurs et mini\\
3\h \> singleton (ou chicane) \h \-\\
2\s \> 14+, arrêt \s , pas d'arrêt \h , arrêt \c\ possible\\
3\c \> 14+, arrêt \c , pas d'arrêts \h /\s \\
3\d \> 12-14- irrégulier\+\\
Sauts \> : courtes\+\\
4NT \> : to play (des points dans les courtes)\-\-\\
3\h \> 14+, court \h , \d\ mieux que Vxxx\\
3\s \> 14+, court \s , \d\ mieux que Vxxx\\
3NT \> 18-19 Bal\\
4\c \> 14+, court \c , \d\ mieux que Vxxx\\
4\h \> 15-18, chicane \h , beaux \d \\
4\s \> 15-18, chicane \s , beaux \d \\
5\c \> 15-18, chicane \c , beaux \d 
\end{bidtable}

\subsubsection{Soutien au niveau de 3}

\begin{bidtable}
1m\+\\
3m \> NF, on peut jouer ce contrat\+\\
4m \> prolongation de barrage\\
3x \> demande d'arrêt dans cette couleur\\
4x \> courte\-\-
\end{bidtable}

\subsubsection{Après interventions}

Le FMI ne s'applique plus (passer par le surcontre)

\begin{bidtable}
1\d-X\+\\
2\d \> 6-9\\
2NT \> 10-11 fitté (Truscott)\\
3\d \> barrage 6+\-
\end{bidtable}

\begin{bidtable}
1\c-1\s\+\\
2\c \> fit 6-10\\
2\s \> 11+, pas 4\h , souvent fitté \c \\
2NT \> naturel 10-12 avec arrêt\\
3\c \> barrage\\
4\c \> barrage\-
\end{bidtable}

\subsection{Enchère du répondant après une inversée}

Pour s'arrêter à une partielle après une inversée de l'ouvreur, le répondant dispose de deux enchères :

\begin{itemize}
\item La répétition de sa couleur au niveau de 2 montrant 6 cartes (parfois 5, en cas de distribution 5521 ou belle couleur).
  La plupart du temps, l'ouvreur passera.

\end{itemize}

Cependant, il dispose de trois enchères non forcing :
1. 2SA avec l'arrêt dans la dernière couleur (et probablement un complément dans la couleur du répondant).
2. Le soutien au palier de 3, invitationnel.
3. La répétition de sa couleur d'ouverture.
Toutes les autres enchères sont FM.

\begin{itemize}
\item L'enchère conventionnelle de 2NT (Lebensohl, transfert pour 3\c ) : main faible ou une main avec 5 cartes en majeure si celle-ci est répétable au niveau de 2.
L'ouvreur rectifiera le transfert sauf s'il est maximum. Il se décrira alors le plus naturellement possible, rendant la séquence FM.
Après la rectification du transfert, le répondant :

\end{itemize}

\begin{enumerate}
\item Passe s'il souhaite jouer 3\c\ 

\item Corrige à 3\d , 3\h\ ou 3\s\ SO (sauf répétition de la couleur de réponse)

\item Répète, le cas échéant, sa couleur au niveau de 3 avec 5 cartes pour laisser l'ouvreur choisir la manche adéquate (voir exemple)

\item Enchérit la manche avec des mains ayant eu de très ténus espoirs de chelem

\end{enumerate}

\dealdiagramenw
{}
{\vhand{AQ642}{T52}{K86}{J5}}
{}
{N}

\begin{bidtable}
1\c-1\s\\
2\h-2NT\\
3\c-3\s
\end{bidtable}

\section{Ouverture de 1M}

Séquence 1S - 2S - 4H : pour jouer
1x - X - 2y (au-dessus) : rencontre
Cue-bid de l'ouverture majeure : FM - Développements à discuter

\subsection{2NT Fitté par 3 cartes 11+}

\begin{bidtable}
1M-2NT\\
3\c \> 12-14 avec une courte\+\\
3\d \> Relais\+\\
3\h \> Court \c \\
3\s \> Court \d \\
3NT \> Court M'\-\\
3M \> Invit\-\\
3\d \> 12-14 Pas de courte\+\\
3M \> Invit\-\\
3\h \> 15-17 toutes mains\+\\
3\s \> Relais\+\\
3NT \> pas de courte\\
4\c \> court \d \\
4\d \> court M'\\
4\h \> court \c \-\-\\
3\s \> 18+ avec une courte\+\\
3NT \> Relais\+\\
4\c \> court \d \\
4\d \> court M'\\
4\h \> court \c \-\-\\
3NT \> 18-19 Balancé\\
4X \> Bicolore 12-15 concentré
\end{bidtable}

\subsubsection{Après intervention}

\begin{bidtable}
1\h\+\\
(P)\+\\
2NT\+\\
3X\+\\
Dbl \> Contrôle X TDC\\
Pass \> Pour jouer 3\h\ en face de 11 pts\\
3\h \> Main forte sans contrôle X\\
4\h \> Main faible sans contrôle X\\
4X \> Splinter dans une main assez faible\\
3NT \> Contrôle X avec une main faible\\
3/4Y \> (Sans saut) 5\h\ 5Y\-\-\-\-
\end{bidtable}

\subsection{3\pdfc\ Fitté par 4+ cartes 11+}

\begin{bidtable}
1M-3\c\\
3\d \> 12-14 toutes mains\+\\
3M \> Invit\\
3M' \> Relais\+\\
3\s \> (sur 3\h ) court \c \\
3NT \> Pas de courte\\
4\c \> Court \d \\
4\d \> Court M'\\
4\h \> (sur 3\s ) court \c \-\-\\
3\h \> 15-17 toutes mains\+\\
3\s \> Relais\+\\
3NT \> pas de courte\\
4\c \> court \d \\
4\d \> court M'\\
4\h \> court \c \-\-\\
3\s \> 18+ avec une courte\+\\
3NT \> Relais\+\\
4\c \> court \d \\
4\d \> court M'\\
4\h \> court \c \-\-\\
3NT \> 18-19 Balancé\\
4X \> Bicolore 12-15 concentré
\end{bidtable}

\subsection{3\pdfd\ Fitté par 4 cartes, 8-10 DH}

\begin{bidtable}
1\h-3\d\\
3\h \> SO\\
3\s \> Demande de courte\+\\
3NT \> Pas de courte\+\\
4\c \> Court \c \\
4\d \> Court \d \\
4\h \> Court \s \-\-\\
3NT \> Force à \s \\
4\c\d \> Forces\\
4\h \> SO
\end{bidtable}

\begin{bidtable}
1\s-3\d\\
3\h \> Force à \h \\
3\s \> SO\\
3NT \> Demande de courte\+\\
4\c \> Court \c \\
4\d \> Court \d \\
4\h \> Court \h \-\\
4\c\d\h \> Forces\\
4\s \> SO
\end{bidtable}

\section{Ouverture de 1NT}

L’ouverture de 1SA se fait avec une main régulière de 15-17H. 
Les mains 5m-4\h 22 et 54m22 de 15-17H sont également recommandées pour cette 
ouverture.
Elle peut comporter une majeure cinquième de 15-16H dans une main régulière. Il est 
conseillé alors que le doubleton comporte un honneur. L’autre majeure troisième est un plus 
également

\begin{bidtable}
1NT\+\\
2\c \> Stayman\\
2\d \> Texas \h \\
2\h \> Texas \s \\
2\s \> ambigu : 8-9 H REG ou Texas \c \\
2NT \> Texas \d \\
3\c \> Puppet Stayman\\
3\d \> 6 belles \d , 6-7 H sans singleton, NF.\\
3\h \> 5/4m court \h , FM\\
3\s \> 5/4m court \s , FM\\
3NT \> Pour jouer\\
4\c \> Bic mineur\+\\
4\d \> fit \d \\
4\h\s \> fit \c \\
4NT \> Coup de frein\-\\
4\d \> Bic M : certitude de manche ou de chelem\\
4\h\s \> Pour jouer\\
4NT \> Quantitatif\\
5\c\d \> Pour jouer\-
\end{bidtable}

\subsection{Stayman}

Pas utilisé avec une main régulière, FM et qui a 1 seule majeure 4e (dans ce cas de figure, passer par 3\c\ Puppet).
Utilisé avec une majeure 5e dans la zone 8-9H.

\begin{bidtable}
1NT-2\c\\
2\d \> Pas de majeure 4e\+\\
2\h \> Misère dorée : 8-9H et 5\h \\
2\s \> Misère dorée : 8-9H et 5\s \\
2NT \> 8-9HL, NF\\
3\c\d \> 5\c /\d , FM, problème pour 3SA ou TDC\\
3\h \> 5\s /4\h , FM\\
3\s \> 5\h /4\s , FM\\
4m \> BIC M TDC, court m\+\\
4M \> coup de frein\\
4NT \> BW \h \-\\
4NT \> Quantitatif\-
\end{bidtable}

\begin{bidtable}
2\h\+\\
2\s \> Misère dorée : 8-9H et 5\s \\
2NT \> 8-9HL, NF\\
3\c\d \> 5\c /\d , FM, problème pour 3SA ou TDC\\
3\h \> fit \h\ NF\\
3\s \> Fit \h\ TDC\\
4\c\d \> Splinter\\
4NT \> Quantitatif\-\\
2\s\+\\
2NT \> 8-9HL, NF. L'ouvreur dit 3\h\ max avec 3\h\ pour retrouver le fit en cas de misère dorée\\
3\c\d \> 5\c /\d , FM, problème pour 3SA ou TDC\\
3\h \> fit \s\ TDC\\
3\s \> Fit \s\ NF\\
4\c\d \> Splinter\\
4NT \> Quantitatif\-\\
2NT\+\\
3\c \> Texas \h \\
3\d \> Texas \s \\
3\h \> TDC\\
3\s \> TDC\\
4\c \> Texas \h \\
4\d \> Texas \s \\
4\h\s \> to play\-\\
3\c \> 5\s\ 4\h 
\end{bidtable}

\subsection{1NT - 2\pdfd\ : Texas \pdfh}

Remarques :

\begin{itemize}
\item Avec 4 cartes à \h\ et minimum, 4333 on rectifie le plus souvent à 3\h\ 

\item avec 4 cartes à \h\ et maximum, on rectifie à 2NT

\end{itemize}

\begin{bidtable}
1NT-2\d\\
2\h\+\\
2\s \> Bicolore 5\h /5x dans la zone 6-7H pour trouver une manche miracle\+\\
2NT \> Relais\+\\
3\c \> \h /\c \\
3\d \> \h /\d \\
3\h \> \h /\s \-\-\\
2NT \> Relais FM. Soit BIC 5\h /4m, soit Unicolore \h\ TDC\+\\
3\c \> 2 cartes à \h \+\\
3\d \> Singleton \s \+\\
3\h\+\\
3\s \> les \c \\
3NT \> les \d\ NF\\
4\c \> les \d\ Forcing\-\-\\
3\h \> Singleton \c \\
3\s \> singleton \d \\
3NT \> 5\h 4m22 honneurs concentrés\\
4\c\d\h \> contrôle, unicolore \h\ TDC\-\\
3\d \> Fit \h , max\+\\
3NT \> enclenche les contrôles\\
3\s4\c\d \> court\-\\
3\h \> Fit \h , min\+\\
3NT \> enclenche les contrôles\\
3\s4\c\d \> court\-\\
3\s \> 5 cartes à \s\ sans 3 cartes à \h \-\\
3m \> BIC 5+\h /5+m, TDC (au moins 11H concentrés).\\
\>Sans ambition de chelem, le 5/5 sera traité comme un 5/4\\
3\s/4\c/4\d \> Splinter\\
4NT \> Quantitatif 16-17 HL - 5332\-
\end{bidtable}

\subsection{1NT - 2\pdfh\ : Texas \pdfs}

Remarques :

\begin{itemize}
\item avec 4 cartes à \s\ et maximum, on rectifie à 2NT

\item Avec 4 cartes à \s\ et maximum 2SA, ensuite 4 \h\ est Texas.

\end{itemize}

\begin{bidtable}
1NT-2\h\\
2\s\+\\
2NT \> Relais FM. Soit BIC 5\s /4m, soit Unicolore \s\ TDC\+\\
3\c \> 2 cartes à \s \+\\
3\d \> Singleton \h \+\\
3\h\+\\
3\s \> les \c \\
3NT \> les \d\ NF\\
4\c \> les \d\ Forcing\-\-\\
3\h \> Singleton \c \\
3\s \> singleton \d \\
3NT \> 5\s 4m22 honneurs concentrés\\
4\c\d\h \> contrôle, unicolore \s\ TDC\-\\
3\d \> Fit \s , max\+\\
3NT \> enclenche les contrôles\\
3\s4\c\d \> court\-\\
3\h \> 5 cartes à \h\ sans 3 cartes à \s \\
3\s \> Fit \s , min\+\\
3NT \> enclenche les contrôles\\
4\c\d\h \> court\-\-\\
3m \> BIC 5+\s /5+m, TDC (au moins 11H concentrés).\\
\>Sans ambition de chelem, le 5/5 sera traité comme un 5/4\\
3\h \> BIC majeur fort\\
4\c\d\h \> Splinter\\
4NT \> Quantitatif 16-17 HL - 5332\-
\end{bidtable}

\subsection{1NT-2\pdfs\ : Texas \pdfc\ ou 8-9H régulier}

\begin{bidtable}
1NT-2\s\\
2NT \> mini\+\\
Pass \> 8-9H réguliers\\
3\c \> SO\\
3\d \> 5/5 mineur FM\\
3\h \> singleton \s \\
3\s \> singleton \h \\
3NT \> singleton \d \\
4\d \> singleton \d\ ambition de chelem\\
4\h \> 6\c /5\h\ NF\\
4\s \> 6\c /5\s\ NF\-\\
3\c \> maxi\+\\
3NT \> 8-9 réguliers\\
3\d \> singleton \d\ ou 5/5 mineur\+\\
3\h \> relais\+\\
3\s \> 5/5 mineur\\
3NT \> singleton \d \\
4\d \> singleton \d\ ambition de chelem\-\-\\
3\h \> singleton \s \\
3\s \> singleton \h \\
4\h \> 6\c /5\h\ NF\\
4\s \> 6\c /5\s\ NF\-
\end{bidtable}

\subsection{1NT-2NT : Texas \pdfd}

Remarque : L'ouvreur peut moduler son soutien, 3\c\ max fitté et 3\d\ RAS.
Il est préférable que l'ouvreur ait une tenue à \c\ lorsqu'il répond 3\c .

\begin{bidtable}
1NT-2NT\\
3\d\+\\
Pass \> 0-6 HL\\
3\h \> singleton \s \\
3\s \> singleton \h \\
3NT \> singleton \c \\
4\c \> TDC \d , singleton \c \\
4\d \> TDC \d\ sans singleton\\
4\h \> 6\d /5\h\ NF\\
4\s \> 6\d /5\s\ NF\-\\
3\c \> maxi\+\\
3\d \> 0-6HL\\
3\h \> singleton \s \\
3\s \> singleton \h \\
3NT \> singleton \c \\
4\c \> TDC \d , singleton \c \\
4\d \> TDC \d\ sans singleton\\
4\h \> 6\d /5\h\ NF\\
4\s \> 6\d /5\s\ NF\-
\end{bidtable}

\subsection{1NT-3\pdfc\ : Puppet Stayman}

Remarque importante

\begin{itemize}
\item Forcing manche

\item Ne s'emploie pas avec des mains comportant un singleton 

\item Ne s'emploie pas avec les 2 majeures 4e

\end{itemize}

\begin{bidtable}
1NT-3\c\\
3\d \> 0, 1 ou 2 majeures 4e\+\\
3\h \> 4\s . L'ouvreur avec 4\s\ fit en disant 3\s\ (3SA = camouflage)\\
3\s \> 4\h \\
3NT \> pas de majeures 4e, pour jouer\\
4\c\d \> 5 cartes 15+, 5332\-\\
3\h \> 5 cartes à \h \+\\
3\s \> ambition de chelem à \h \\
3NT \> pour jouer\\
4\c\d \> 15+, 5332\-\\
3\s \> 5 cartes à \s \+\\
4\h \> ambition de chelem à \s \\
3NT \> pour jouer\\
4\c\d \> 15+, 5332\-\\
3NT \> 5\h /4\s 
\end{bidtable}

\subsection{1NT-3\pdfh/\pdfs\ : 5/4 mineurs court \pdfh/\pdfs}

Bicolore mineur 5/4 avec une courte dans la couleur annoncée, forcing manche avec au 
moins 9H. L’ouvreur donne 3SA avec au moins 1 arrêt et demi dans la couleur de la courte.
Sinon, il se décrit de manière naturelle. Il peut même fitter le résidu s’il possède cette majeure 5ème.

\section{Ouverture de 2\pdfc}

\subsection{Définition}

\begin{itemize}
\item FAIBLE en \d\ (constructif VUL et en 1ère et 2ème NV) et 11-13 en 4ème

\item 22-24 BAL

\item GAMBLING solide en \c\ 

\item FM sauf Bicolores Majeurs

\end{itemize}

\begin{bidtable}
2\c--\+\\
2\d \> relais\+\\
Pass \> FAIBLE \d \\
2\h \> 24+ balancé\\
\>Bicolore avec au moins 4\h \\
\>Unicolore \h\ balancé 22-25\+\\
2\s \> relais\+\\
2NT \> 25H et + (développements comme sur 2N)\\
3\c \> 5+\c , =4\h \\
3\d \> 5+\d , =4\h \\
3\h \> 5+\h , 4+\c \\
3\s \> 5+\h , 4+\d \\
3NT \> Unicolore \h\ Gambling (style régulier avec 6/7\h\ plein et 2 As)\\
4\c\d \> 6\h /5m moins fort que 3\h\ suivi de 4\c\ (5+/5)\\
\>Ex: \s\ 2 \h\ ADV987 \d\ 8 \c\ ARDT2\-\-\\
2\s \> Bicolore avec 4+\s \\
\>Unicolore \s\ balancé 22-25\+\\
2NT \> relais\+\\
3\c \> 5+\c , =4\s \\
3\d \> 5+\d , =4\s \\
3\h \> 5+\s , 4+\c \\
3\s \> 5+\s , 4+\d \\
3NT \> Unicolore \s\ Gambling\\
4\c\d \> 6\s /5m moins fort que 3\s\ suivi de 4\c\ (5+/5)\-\-\\
2NT \> 22+ - 24H\\
3\c\d \> Unicolores ou bicolore mineur (la collante est ambigue)\\
3\h\s \> Unicolores irréguliers\\
3NT \> Gambling \c \\
4\c\d \> Bicolores 6/5m \h\ (6\h -5\c /\d\ + faible que 2\c -2\d -2\h -2\s -4\c /\d )\\
4\h\s \> Bicolores 6/5m \s\ (pareil)\-\\
2\h\s3\c \> F1T (pas forcing en paires !)\\
\>Nouvelle couleur GF\+\\
2NT \> Max misfit\\
3\d \> Min\\
3x/4x \> Fit, Splinters mains faibles\-\\
2NT \> Relais F1 avec espoir de manche en face d’un faible à \d \+\\
3\c \> faible \d\ et une courte\+\\
3\d \> NF courte s’annonce par paliers si max\\
\>(=> Passe si min du faible, annonce la courte par palier si max du faible)\-\\
3\d \> faible \d\ sans courte\\
3\h \> Max pièce \c /\h , pas de courte\+\\
3P \> relais\+\\
3NT \> court \c \\
4\c \> court \h \-\-\\
3\s \> Max force sans courte\\
3NT \> beaux \d\ ARD ou ARVT sans courte\\
4X \> GF\-\\
3\d \> 7-11H (Forcing > 4NT si main forte, mêmes dvpts)\\
3\h\s4\c \> NAT + Fit\-
\end{bidtable}

Avec les mains TRICOLORES, on reparle toujours sur 3NT.

Ex : \s\ - \h\ ADV8 \d\ ARD98 \c\ ARV10
(2\c -2\d -2\h -2\s -3\d -3NT-4\c -...)

\subsection{Après intervention}

\begin{itemize}
\item Contre punitif

\item Toute enchère F1T

\item 2NT R fort

\item Cue demande d'arret

\end{itemize}

\subsection{Après contre}

\begin{itemize}
\item Passe des \c !

\item Surcontre FORT ou UNICOLORE

\item 2\d\ ambigu

\item Nouvelle couleur F1T

\end{itemize}

\section{Ouverture de 2\pdfd}

\subsection{Définition}

\begin{itemize}
\item Unicolore FI

\item Bicolore Majeur FI ou FM

\end{itemize}

\begin{bidtable}
2\d-2\h\\
2\s \> Montre un bicolore majeur \textbf{+ long à \s }\+\\
2NT \> Relais\+\\
3\h \> 5/5 MAJ FI 18-21 PH\\
3\s \> 6\s /4\h\ FI 18-21 PH\\
3NT \> 5422 FM\\
4\c\d \> Tricolore parfait\\
4\h \> 65 FM\\
4\s \> 74 FM\\
3\d \> 5431 FM\\
3\c \> FM ambigu\+\\
3\d \> Relais\+\\
3\h \> 55 FM\\
3\s \> 64 FM\\
3NT \> 5413 FM\-\-\-\\
3\c\d \> Aucun intérêt pour les majeures, naturel (pas belle 6e)\\
3\h\s \> TDC\\
4\h\s \> NF\-
\end{bidtable}

\begin{bidtable}
2\d-2\h\\
2NT \> Montre un bicolore majeur \textbf{+ long à \h }\+\\
3\c \> Relais\+\\
3\d \> Résidu mineur 31 ou 30\+\\
3\h \> Relais\+\\
3\s \> 4513\\
3NT \> 4531\\
4X \> 4603 ou 4630\-\-\\
3\h\\
3\s \> 65 FM\\
3NT \> 5422 FM\\
4\c\d \> Tricolore parfait\\
4\h \> 74 FM\-\-
\end{bidtable}

\begin{bidtable}
2\d-2\h\\
3\c\d\h\s \> ACOL naturel (=FI) Collante ambigue, éventuellement pour tenter de jouer 3NT de la main forte\+\\
3\c\+\\
3\s \> Ambigu\+\\
3NT \> pour jouer\+\\
4\h \> TDC avec cue \s \\
4NT \> Quantitatif\-\-\\
4X \> Cue, TDC fit \h \-\\
3NT \> 9-10 plis en \d\ avec au moins 2 grosses pièces extérieures\-
\end{bidtable}

\begin{bidtable}
2\d\+\\
2\s \> Naturel 5 cartes avec deux gros honneurs ou 5/5 correct\\
3\c\d\h\s \> Transferts pour unicolores Q+ avec 2 gros honneurs\\
\>Le soutien ou la rectification du texas montre un intérêt pour la couleur du répondant\\
2NT \> 5/5 Mineur FM\-
\end{bidtable}

\subsection{Après intervention en 2e}

\subsubsection{Contre}

\begin{itemize}
\item Punitif si intervention au palier de 2

\item Appel si intervention au palier de 3 ou plus

\end{itemize}

\subsubsection{Passe RAS}

Ensuite, contre de l'ouvreur est punitif au niveau de 2 et montre le bicolore majeur ou une demande d'arrêt au niveau de 3.

\subsubsection{3\pdfc\pdfd\pdfh\pdfs\ Transferts (Transfert cue - Tricolore court)}

\subsubsection{2NT Bicolore mineur}

\subsubsection{Après 2\pdfd\ - X}

\begin{itemize}
\item Passe : envie de jouer

\item Tout le reste sauf 2\h\ : Naturel

\item XX : Naturel (4 belles cartes, pas 5 ou 6)

\end{itemize}

\subsection{Après intervention en 4e}

\begin{itemize}
\item PASSE, CUE-BID et ENCHERES A SAUT montrent le BICOLORE MAJEUR JUSQU’À 3\s .

\end{itemize}

\begin{itemize}
\item CONTRE demande l’arrêt

\end{itemize}

\subsection{Corollaires (à vérifier avec Gazilli !)}

\begin{bidtable}
1\s-1NT\\
3\h \> montre un 5=/4= dans la zone 18-21H
\end{bidtable}

\begin{bidtable}
1\s-1NT\\
4\h \> montre un 6/5 de 14 à 17H
\end{bidtable}

\begin{bidtable}
1\h-1NT\\
3\s \> montre un 6/5 de 14 à 17H, c’est NF !
\end{bidtable}

\section{Ouverture de 2\pdfh\ et 2\pdfs}

Montre une main de 5 à 10PH comportant une couleur 6ème (qualité de la couleur dépend de la position et de la vulnérabilité).

\begin{bidtable}
2M\+\\
3/4M \> Prolongation de barrage, 4M est une main faible ou forte qui espère gagner la manche.\\
2/3X \> Forcing un tour, l'ouvreur donne le fit (xxx ou Hx), répète sa couleur ou dit 3SA.\\
2NT \> Relais fort\+\\
2/3X \> FM, Max avec gros honneur.\+\\
3M \> TDC\+\\
4Y \> singleton Y\\
4M \> pas de singleton\-\-\\
3M \> Mini\\
3NT \> ARDxxx\\
4x \> Belle main, singleton X\-\\
4X \> Rencontre ?\-
\end{bidtable}

Attention, ``La force prime sur le courte''

\section{Ouverture de 2NT}

Cette ouverture peut contenir une majeure 5e.

\begin{bidtable}
2NT---\\
3\c \> Puppet Stayman\\
3\d \> Texas \h \\
3\h \> Texas \s \\
3\s \> Texas pour 3N\\
3NT \> 5\s\ et 4\h \\
4\c \> Bicolore mineur TDC\\
4\d \> Bicolore majeur, limité à la manche ou certitude de chelem\\
4NT \> Quantitatif
\end{bidtable}

\subsection{Le Puppet Stayman}

\begin{bidtable}
2NT-3\c \> ---\\
3\d \> 4\h\ et/ou 4\s \+\\
3\h \> 4 cartes à \s \+\\
3\s \> fit \s \+\\
4X \> envie de chelem\-\\
3NT \> 4 cartes à \h\ (il est donc nécessaire de passer par 3\h\ avec\\
\>les deux majeures 4e et des envies de chelem)\\
\>4X fit avec envie de chelem\-\\
3\s \> 4 cartes à \h\ et pas 4 cartes à \s \\
3NT \> 3-\h\ et 3-\s \\
4\d \> 4\h\ et 4\s\ sans envie de chelem\-\\
3\h \> 5\h \\
3\s \> 5\s \\
3NT \> 3-\h\ et 3-\s 
\end{bidtable}

\subsection{Les Texas}

\begin{bidtable}
2NT---\\
3\d\+\\
3\h \> 2\h \\
3\s \> 5\s\ et 2\h \+\\
4\c/4\d \> ctrl \c /\d\ et fit \s \\
4\h/4\s \> Arrêt\-\\
3NT \> 3+\h\ et ctrl \s \\
4\c \> 3+\h\ et ctrl \c\ (sans ctrl \s )\\
4\d \> 3+\h\ et ctrl \d \\
4\h \> 3+\h\ et tous les ctrl\-
\end{bidtable}

\begin{bidtable}
3\h\+\\
3\s \> 2\s \\
3NT \> 5\h\ et 2\s \+\\
4\c/4\d \> ctrl \c /\d\ et fit \h \\
4\h/4\s \> Arrêt\-\\
4\c \> 3+\s\ et ctrl \c \\
4\d \> 3+\s\ et ctrl \d\ (sans ctrl \c )\\
4\h \> 3+\s\ et ctrl \h \\
4\s \> 3+\s\ et tous les ctrl\-
\end{bidtable}

\begin{bidtable}
3\s\+\\
3NT\+\\
4\c\d \> Envie ou certitude de chelem à \c /\d . L'ouvreur décourage par 4SA.\\
4\h\s \> Envie de chelem à \h /\s . L'ouvreur passe ou demande les clés.\+\\
Avec \> certitude de chelem, il faut passer par un texas.\-\-\-
\end{bidtable}

\subsection{Autres réponses}

\begin{bidtable}
4\c\+\\
4\d \> fit \d \\
4\h \> fit \c\ et ctrl \h \\
4\s \> fit \c , ctrl \s , pas de ctrl \h \-
\end{bidtable}

\begin{bidtable}
4\d \> L'ouvreur choisit sa majeure
\end{bidtable}

\begin{bidtable}
4NT \> Minimum, l'ouvreur passe.\\
\>Maximum, les réponses sont les suivantes :\+\\
5\c \> 1 ou 4 As\\
5\d \> 0 ou 3 As\\
5\h \> 2 As et 5\c \\
5\s \> 2 As et 5\d \\
5NT \> 2 As et pas de min 5e\-
\end{bidtable}

\section{Défense sur les ouvertures Multi}

Principe général sur les ouvertures de type Multifonction :

\begin{itemize}
\item X montre soit 12-15 semi-balancé, soit 16+ si irrégulier ou 20+ si balancé.

\item Passe puis X est d'appel.

\item L'annonce de la couleur adverse (si une seule couleur connue) est d'appel (style tricolore, 4441/5431) ou bicolore cher).

\item 2X est naturel avec l'ouverture.

\item 3X est naturel avec 7-8 levées.

\item 2SA est balancé 16-19 avec arrêt(s).

\item Ensuite, sur un X d'appel et une enchère adverse en 3è, X est d'appel (du saut et transformable).

\end{itemize}

En pratique :

\subsection{2\pdfc\ multi (les deux M faibles ou fort)}

\begin{itemize}
\item X 12-15 semi-balancé ou fort irrégulier ou 20-22 balancé

\item 2\d\ naturel

\item 2M semi-saturel (4+ cartes, voire 3 belles) 13-17

\item 2NT semi-balancé 16-19 avec arrêts majeurs

\item 3X NAT 7/8 levées

\item 3SA pour jouer

\item Passe puis 2SA avec les mineures (5+ 4+)

\item Passe puis 3SA bicolore m très distribué

\end{itemize}

\subsection{2\pdfd\ Multi (unicolore M faible ou fort)}

\begin{itemize}
\item X 12-15 semi balancé, en principe 3+ 3+ dans les M, ou fort

\item 2M naturel

\item 2NT semi-BAL 16-19 avec les arrêts majeurs

\item 3X naturel, 7-8 levées

\item Passe puis X d'appel

\item Passe puis 2NT avec les mineures (5+ 4+)

\end{itemize}

Après 2\d\ - 2M, en 4è :

\begin{itemize}
\item X est d'appel sur la M et 3\c\ d'appel sur la M'

\item 2NT ou 3SA naturel

\item 3M bicolore m 

\item 4m bicolore mM'

\end{itemize}

\subsection{2\pdfh\ Ekren (les deux M faible)}

\begin{itemize}
\item X d'appel (transformable) avec 4+ \h\ 

\item 2\s\ d'appel court à \h\ (ensuite 2NT modérateur)

\item 2NT naturel 16/18 (ensuite 3m NF et 3M FM \c /\d\ respectivement)

\item 3m naturel

\item 3M (semi-)bicolore mineur fort, court M

\item 3NT naturel

\end{itemize}

\subsection{2\pdfh\ Multi (Faible \pdfs\ ou \pdfs\ + m ou fort)}

\begin{itemize}
\item X ouverture avec 5+\h\ 

\item 2\s\ contre d'appel court \s\ (ensuite 2NT modérateur)

\item 2NT naturel 16/18

\item 3x naturel

\item 3NT naturel

\end{itemize}

\subsection{2\pdfs\ Multi (les deux mineures faible ou fort)}

\begin{itemize}
\item X ouverture avec 5+\s\ 

\item 2NT naturel 16/18

\item 3\c\ contre d'appel avec + de \h\ ou égalité de longueur

\item 3\d\ contre d'appel avec + de \s\ 

\item 3M naturel

\item 3NT naturel

\end{itemize}

\subsection{Ouverture de 3X en Texas}

\begin{itemize}
\item X d'appel 12-17, court de la couleur longue annoncée

\item 3X+1 main forte avec un bicolore cher

\end{itemize}

\section{Défense sur 1SA}

\subsection{Sur le SA fort}

\begin{bidtable}
(1NT)\+\\
Dbl \> Texas \c\ ou Landy (BIC M 4+/4+) si suivi de 2\d \\
\>2C Avec une préférence M (ou sans préférence avec un unicolore \c )\+\\
2\d \> Landy\+\\
2\h/2\s \> Nat NF\\
2NT \> Relais\+\\
3\c \> mini\\
3\d \> Max 5+\h /4\s \\
3\h \> Max 5+\s /4\h \\
4\c \> 5/5 court \c \\
4\d \> 5/5 court \d \-\\
3\c \> NF\\
3\d \> F1\\
3M \> Invit\-\\
2\d \> Sans préférence M (au mieux 2/2)\\
\>Le répondant dit 2\d\ avec un unicolore \d\ ou un semi-bicolore m. \\
\>L'intervenant passe, nomme une majeure cinquième ou 3\c\ naturel NF. Les autres enchères sont naturelles avec du \c .\\
2\h \> Unicolore\\
2\s \> Unicolore\\
2NT \> F1\-\\
2\c \> Texas \d . Si fort, suivi par 2M (6\d 4M), 3\c\ (5\d 4+\c ), 3\d\ unicolore\\
2\d \> Unicolore M\+\\
2\h\s \> P/C. Ensuite si maxi :\+\\
2NT \> Invit avec des \h \\
3\c\d \> Invit avec des \s \-\\
2NT \> Relais\+\\
3\c \> Max \h \+\\
3\h\s \> Invit\-\\
3\d \> Max \s \+\\
3\h\s \> Invit\-\\
3\h \> Min \h \\
3\s \> Min \s \-\\
3\c \> NF\\
3\d \> F1\\
3\h\s \> P/C\\
4\h\s \> Clôture !!\-\\
2\h \> 5\h 4m\+\\
2NT \> Relais - demande la mineure\+\\
3\c \> Min \c \\
3\d \> Min \d \\
3\h \> Max \c \\
3\s \> Max \d \\
3NT \> Max 0544\-\\
3\c \> P/C\\
3\d \> Invit à \h \\
3\h \> Pour jouer\\
3NT \> pour jouer\-\\
2\s \> 5\s 4m\+\\
2NT \> Relais - demande la mineure\+\\
3\c \> Min \c \\
3\d \> Min \d \\
3\h \> Max \c \\
3\s \> Max \d \\
3NT \> Max 5044\-\\
3\c \> P/C\\
3\h \> Invit à \s \\
3\s \> Pour jouer\\
3NT \> pour jouer\-\\
2NT \> Bicolore mineur ou bicolore Fm indéterminé\+\\
3\c\d \> Préférence\\
3\h\s \> Naturel F1, sauf si X sur 2NT\-\\
3x \> Barrage\\
3NT \> Bicolore mineur très distribué.\-
\end{bidtable}

En réveil :

\begin{bidtable}
(1NT)-Pass-(Pass)\+\\
Dbl \> 13+ et semi-régulier (au moins 44 ou 5+)\\
2\c \> Landy\\
2X \> Naturel\\
3X \> Naturel + fort\-
\end{bidtable}

\subsection{Sur le SA faible}

\begin{itemize}
\item Contre : tendance punitive. Au minimum le maximum du NT, main +- balancée. Le répondant passe avec une main plate à partir de 8H.
    2\c\ est stayman, 2\d , 2\h , 2\s\ et 2NT sont des Texas.

\item 2\c\ : Landy - Bicolore majeur au moins 5/4.

\item 2\d , 2\h , 2\s , 2NT : Texas.

\item 3X : Belle couleur 6e - environ 13-15H

\end{itemize}

En réveil : Idem.

\subsection{Réponses à l'intervention d'1NT}

\begin{bidtable}
(1\h)-1NT-(Pass)\+\\
2\c \> Texas \d \\
2\d \> Stayman\\
2\h \> Texas \s \\
2\s \> Texas \c \\
2NT \> Naturel 7+ - 9-H\\
3\c \> Texas \d \-
\end{bidtable}

\begin{bidtable}
(1\s)-1NT-(Pass)\+\\
2\c \> Texas \d \\
2\d \> Texas \h \\
2\h \> Stayman\\
2\s \> Texas \c \\
2NT \> Naturel 7+ - 9-H\\
3\c \> Texas \d \-
\end{bidtable}

\begin{bidtable}
(1\c\d)-1NT-(Pass)\+\\
2\c \> Stayman\\
2\d \> Texas \h \\
2\h \> Texas \s \\
2\s \> Texas \c \\
2NT \> Naturel 7+ - 9-H\\
3\c \> Texas \d \\
4\d \> Bicolore majeur\-
\end{bidtable}

\subsection{En réveil}

\begin{itemize}
\item 1NT : 11-14H

\item Dbl puis 1NT : 15-17H

\item 2NT : 18-19H

\item Dbl puis 2NT : 20-21

\end{itemize}

\section{Les barrages et ouvertures à haut palier}

\subsection{Namyats 4\pdfc/4\pdfd}

Cette ouverture montre une couleur \h /\s\ 8e fermée ou 7e fermée avec un as.
La rectification à la manche est un arrêt.

L'enchère juste au-dessus montre un intérêt pour le chelem et demande à l'ouvreur de:

\begin{itemize}
\item Nommer la manche sans as annexe

\item Nommer son as annexe

\end{itemize}

\subsection{3SA}

Mineure 8e fermée sans rien à côté. 
4/5\c \d\ PoC

\section{Enchères après passe}

\begin{bidtable}
1\s\+\\
1NT \> 6-11H, misfit\\
2\c \> Drury, 3\s\ 10-12 DH/ 4\s\ jeu plat, (2\s\ avec GH et 11 points H)\+\\
2\d/2\h \> naturel avec une ouverture correcte\\
2\s \> Main n'ayant pas d'espoir de manche en face d'un passe initial\\
2NT \> 14+ 17-, jeu régulier avec des arrêts un peu partout\\
3\c/3\d/3\h \> Beau 5-5, 16+\\
3\s \> 6\s , 16/18\\
3NT \> 17+ 19, jeu régulier\-\\
2\d/2\h \> Naturel. Belle couleur souvent 6ème, 8-11 H.\\
2\s \> 6-9HL\\
2NT \> Fit 4e et un singleton\+\\
3\c \> Relais\+\\
3\d \> Singleton \d \\
3\h \> Singleton \h\ (ou \c\ si fit \h )\\
3\s \> Singleton \c\ (ou \s\ si fit \h )\-\-\\
3\c/3\d/3\h \> Rencontre : 5 belles cartes, 4+ \s , 9/11H : L'ouvreur revient à la majeur au palier qu'il souhaite ou nomme un contrôle si TDC. Si double fit, BW 6 clés ?\\
3\s \> Barrage\\
4\c/4\d/4\h \> Splinter avec 5 atouts\\
4\s \> Barrage\-
\end{bidtable}

\subsection{Après intervention}

Le contre remplace le Drury (en plus de garder sa signification habituelle). Le cue-bid montre 4 cartes.

Les enchères de rencontre restent d'application, le Splinter s'effectue uniquement dans la couleur d'intervention.

\section{Enchères de chelem}

L'enchère de 4NT est un Blackwood à 5 clés incluant le roi d'atout. L'atout est la couleur explicitement ou implicitement (via cue-bid) fittée ou à défaut la dernière couleur nommée naturellement

\begin{bidtable}
4NT\+\\
5\c \> 4 ou 1 clé(s)\\
5\d \> 3 ou 0 clé(s)\\
5\h \> 2 clés sans dame d'atout\\
5\s \> 2 clés avec dame d'atout\\
5NT \> nombre impair de clé et une chicane\\
6x \> nombre pair de clés et chicane x utile (ou chicane au-dessus si x = atout)\-
\end{bidtable}

Sur une réponse 5m, le palier immédiatement supérieur en dehors de l'atout est un relais pour la demande de la dame d'atout et des rois spécifiques.
Avec la dame d'atout, on nomme économiquement la couleur du premier roi ou 5SA déniant un roi.

Le retour à l'atout au palier le plus économique dénie la dame d'atout. Ensuite, le palier au-dessus demande de nommer les rois dans un ordre économique.

Une autre enchère non-ambigüe explore pour le grand chelem, demandant un complément.

Blackwood d'exclusion : Double saut anormal avec fit précisé ! 
Réponses : 1, 3/0, 2, 2+Q.

Une réponse de 5NT avec une zone initiale de 2 pts (par exemple l'ouverture de 2NT) propose de découvrir un fit 4-4 au palier de 6.

En cas de bicolore et sans fit précisé par manque de place, le Blackwood est à 6 clés :

\begin{itemize}
\item 5\h\ : pas de dame

\item 5\s\ : la dame la moins chère

\item 5NT : la plus chère

\item 6\c\ : les deux

\end{itemize}

Après le BW, 5NT est une demande de roi :

\begin{bidtable}
5NT\+\\
Retour \> à l'atout : pas de roi.\\
6X \> : Roi X ou les deux autres.\\
6NT \> : Tous les rois.\-
\end{bidtable}

Sur un 4NT quantitatif :

\begin{bidtable}
4NT\+\\
Passe \> mini\\
5\c \> 4 ou 1 As\\
5\d \> 3 ou 0 As\\
5\h \> 2 As et 5\c \\
5\s \> 2 As et 5\d \-
\end{bidtable}

D0PI/R0PI

Après intervention adverse sur le Blackwood :
Double 30 Passe 41, ensuite 2 et 2+Q

Après le contre d'une couleur du Blackwood :
Redouble 30 Passe 41, ensuite 2 et 2+Q

Si intervention adverse au-dessus de 5 de notre couleur :
DEP0/REP0 - Double Even, Pass Odd.

Intervention adverse par contre sur les cue-bids :

\begin{itemize}
\item Passe : sans contrôle

\item XX : contrôle du premier tour

\item Autre : contrôle du second tour (retour à l'atout = second tour sans autre cue).

\end{itemize}

Intervention adverse en couleur sur les cue-bids :

\begin{itemize}
\item X : punitif

\item Passe : sans contrôle

\item Autre : souvent court dans la couleur

\end{itemize}

Intervention adverse par contre sur un cue-bid dans une courte connue (Splinter par exemple) :

\begin{itemize}
\item Passe : pas de contrôle

\item XX : contrôle du premier tour

\item Autre : cue-bid + contrôle du second tour

\end{itemize}

\section{Enchères compétitives}

\section{Jeu de la carte}

\subsection{L'entame}

\subsubsection{Entame à SA}

Principe de la quatrième meilleure :

\begin{itemize}
\item Quatrième carte d'une couleur contenant un honneur (le 10 n'est pas considéré comme un honneur, sauf 109xx ou 10 cinquième. )

\item Tête de séquence d'une séquence d'au moins trois cartes dont un honneur

\item Séquence brisée, le plus gros honneur du début de séquence (A/R/D 109xx - le 10, jusqu'au 9 - D98x le 9.)

\item Top of nothing : \textbf{x}xx - la plus grosse; x \textbf{x} x x

\item Séquence = 3 cartes avec au moins 1 honneur (àpd T).

\item Si le partenaire ouvre d'1\c\ et que les adversaires jouent à NT, 4e meilleure même à \c .

\end{itemize}

Sur l'entame :

\begin{itemize}
\item Roi : Déblocage, Parité. Si le mort est court - appel par les petites

\item As : Appel avec des couleurs longues (le partenaire est court) et pas des honneurs

\item Dame : Attitude - on dit qu’on aime quand on a des honneurs car son entame provient souvent d’une longue

\item Valet : Parité - on confirmera par la suite si intérêt dans la couleur

\end{itemize}

\subsubsection{Entame à la couleur}

Tête de séquence de 2 cartes ou séquence brisée.

Parité :

\begin{itemize}
\item Dans 4 cartes, la 2e si pas d'honneur, la 3e si un honneur.

\item Dans 6 cartes, le 3e si pas d'honneur, la 5e si un honneur.

\end{itemize}

Jeu des honneurs :

\begin{itemize}
\item En 3e position dans la levée, on met l'honneur le moins élevé.

\item Si le partenaire entame le Roi - la Dame promet le Valet.

\end{itemize}

\subsubsection{Dans le cours du jeu}

\begin{itemize}
\item Switch petit prometteur - en fonction du mort.

\end{itemize}

\begin{itemize}
\item Quand le partenaire entame d'un honneur et que le mort est court - italien.

\end{itemize}

\begin{itemize}
\item Confirmation d'entame par les petites - si nécessaire.

\end{itemize}

\begin{itemize}
\item Première défausse Lavinthal.

\end{itemize}

\end{document}
