\documentclass[a4paper]{article}
\usepackage[T1]{fontenc}
\usepackage[utf8]{inputenc}
\usepackage{newcent}
\usepackage{helvet}
\usepackage{graphicx}
\usepackage[pdftex, pdfborder={0 0 0}]{hyperref}
\frenchspacing

\include{bml}
\title{Système Monticelli's}
\author{Thomas et Romain Monticelli}
\begin{document}
\maketitle
\tableofcontents

\section{Introduction}

\section{Ouverture de 1m}

\subsection{Le Walsh}

À partir de 2 cartes. Le Walsh, basé sur le principe ``la majeure d'abord'', consistant à ne présenter les D
K que dans les mains fortes ou sans majeures, est utilisé.
En cas de fit fort sans majeure, le fit mineur inversé est utilisé.

\begin{bidtable}
1\c---\\
1\d \> 3 possibilités :\\
\>4+\d\ et inv+\\
\>5+\d , 3-\h \s\ et 6-10H\\
\>BAL 5-7H et 3-\h \s \\
1\h \> \h /4+ (possibilité de \d\ plus longs que \h\ si <11h)\\
1\s \> \s /4+ (possibilité de \d\ plus longs que \s\ si <11h)\\
1NT \> 8-10 3-\h \s \\
2\c \> FMI : fit \c , 10+ (voir développements plus loin)\\
2\d \> 6+\d\ (et 2GH), 18H+\\
2\h \> 5+\s\ 4+\h , 5-8H, idem sur 1\d \\
2\s \> 5+\s\ 4+\h , 9-10H, idem sur 1\d \\
2NT	 \> 11-12 BAL (3\c\ NF, reste FM)\\
3\c	 \> 5\c , main irrégulière ~5-9 H\\
3X	 \> Barrage 7 cartes\\
3NT	 \> 12-14 BAL
\end{bidtable}

\subsection{La séquence 1\pdfc\ - 1\pdfd}

\begin{bidtable}
1\c-1\d;\\
1M \> 12-17 H, M/4 Main irrégulière (5\c )\\
1\h\+\\
1\s \> 4eme couleur forcing\\
\>(L’ouvreur réagit comme sur une enchère NAT mais ne peut pas dire 4\s ) \+\\
2\s \> 12-14\\
3\s \> 15+\+\\
3NT \> SO\\
4X \> cue fit \s \-\-\\
2\s \> 5\d 4\s\ FM\-\\
1NT \> 12-14 H, BAL\+\\
2\c \> checkback stayman avec des mains INV+\+\\
2X\+\\
3\c \> 5\d 5\c\ FM (fit différé)\\
3\d \> 6+\d\ FM\-\\
2\d \> Pas de M/4\\
2\h \> 4\h\ (et peut être 4\s )\\
2\s \> 4\s \-\\
3\c \> 5\d 4\c\ INV\\
3\d \> 6\d\ INV\-
\end{bidtable}

\subsection{La séquence 1\pdfc\ - 1M}

\begin{bidtable}
1\c-1M;\\
1NT\+\\
2\d \> 5\d 4M faible to play\-\\
2\c\+\\
2\d \> 3eme couleur forcing\-
\end{bidtable}

\subsection{La séquence 1m - 1M}

\begin{bidtable}
1m-1M;\\
2NT \> pas de fit\+\\
3\c \> réponses style Baron\+\\
3\d \> 5m\\
3M \> 3m\\
3M' \> 4M'\\
3NT \> aucun des cas précédents\-\\
3\d \> fit dans la m (FM ou TDC)\\
3M \> NAT NF\\
3M' \> BIC (55 ou 65) FM\\
3NT \> SO\\
4m/M' \> cue avec atout M\\
4\h \> (sur 1\s ) BIC NF\\
4M \> SO\\
4NT \> quantitatif\-\\
3NT \> fit main régulière\\
4(3)M’ \> splinter\\
4m’ \> splinter\\
4m \> 5+m 4+M\\
4M \> main extrêmement irrégulière avec peu de points d’honneurs
\end{bidtable}

\subsection{La séquence 1m - 2\pdfh}

\subsection{La séquence 1m - 2\pdfs}

\subsection{La séquence 1\pdfd\ - 2\pdfc}

\begin{bidtable}
1\d-2\c;\\
2\d \> irrégulier 11-15, BAL 12-13\\
2M \> inversée NAT\\
2NT \> BAL 14 ou 18-19\+\\
3\c \> relai\+\\
3\d \> 18-19, le reste NAT 14\-\-\\
3\c \> 16+ PJS Fit 4ème ou 18-19 et 3 cartes\\
3\d \> 6+\d , 16+ problème dans une ou deux M\\
3M \> 65 sans inversée\\
3NT \> 6+\d , 16+ avec les arrêts M
\end{bidtable}

\subsection{Le fit mineur inversé}

\subsection{Enchère du répondant après une inversée}

\section{Ouverture de 1M}

% Séquence 1S - 2S - 4H : Splinter ?
% 1x - X - 2y (au-dessus) : faible ou rencontre ?
% Cue-bid de l'ouverture majeure : FM ? Développements à discuter

\section{Ouverture de 1NT}

\begin{bidtable}
1NT\+\\
2\c \> Stayman\\
2\d \> Texas H\\
2\h \> Texas S\\
2\s \> ambigu : 8-9 H reg ou Texas C\\
2NT \> Texas D\\
3\c \> Puppet Stayman\\
3\d \> 6 belles D, 6-7 H sans singleton, NF.\\
3\h \> 5/4m court H, FM\\
3\s \> 5/4m court S, FM\\
3NT \> Pour jouer\\
4\c \> Bic mineur\+\\
4\d \> fit D\\
4\h/ \> S fit C\\
4NT \> Coup de frein\-\\
4\d \> Bic M : certitude de manche ou de chelem\\
4\h \> /S  Pour jouer\\
4NT \> Quantitatif\\
5\c\d \> Pour jouer\-
\end{bidtable}

\subsection{Stayman}

Pas utilisé avec des main régulières, FM, 1 seule majeure 4e.
Utilisé avec une majeure 5e dans la zone 8-9H.

\begin{bidtable}
2\c\+\\
2\d \> Pas de majeure 4e\+\\
2\h \> Misère dorée : 8-9H et 5H\\
2\s \> Misère dorée : 8-9H et 5S\\
2NT \> 8-9HL, NF\\
3\c/D \> 5C/D, FM, problème pour 3SA ou TDC\\
3\h \> 5S/4H, FM\\
3\h \> 5H/4S, FM\\
4m \> BIC M TDC, court m\+\\
4\h \> coup de frein\\
5NT \> BW H\-\\
4NT \> Quantitatif\-\-
\end{bidtable}

\begin{bidtable}
2\h\+\\
2\s \> Misère dorée : 8-9H et 5S\\
2NT \> 8-9HL, NF\\
3\c/D \> 5C/D, FM, problème pour 3SA ou TDC\\
3\h \> fit H NF\\
3\s \> Fit H TDC\\
4\c/D \> Splinter\\
4NT \> Quantitatif\-\\
2\s\+\\
2NT \> 8-9HL, NF. L'ouvreur dit 3H max avec 3H pour retrouver le fit en cas de misère dorée\\
3\c/D \> 5C/D, FM, problème pour 3SA ou TDC\\
3\h \> fit S TDC\\
3\s \> Fit S NF\\
4\c/D \> Splinter\\
4NT \> Quantitatif\-\\
2NT\+\\
3\c \> Texas H\\
3\d \> Texas S\\
3\h \> TDC\\
3\s \> TDC\\
4\c \> Texas C % Quelle différence ?\\
4\d \> Texas S\\
4\c/S \> to play\-
\end{bidtable}

\section{Ouverture de 2\pdfc}

\subsection{Définition}

\begin{itemize}
\item FAIBLE en \d\ (constructif VUL et en 1ère et 2ème NV) et 11-13 en 4ème

\item 22-24 BAL

\item GAMBLING solide en \c\ 

\item FM sauf Bicolores Majeurs

\end{itemize}

\begin{bidtable}
2\c--\+\\
2\d \> relais\+\\
Pass \> FAIBLE \d \\
2\h \> 24+ balancé\\
\>Bicolore avec au moins 4\h \\
\>Unicolore \h\ balancé 22-25\+\\
2\s \> relais\+\\
2NT \> 25H et + (développements comme sur 2N)\\
3\c \> 5+\c , =4\h \\
3\d \> 5+\d , =4\h \\
3\h \> 5+\h , 4+\c \\
3\s \> 5+\h , 4+\d \\
3NT \> Unicolore \h\ Gambling (style régulier avec 6/7\h\ plein et 2 As)\\
4\c\d \> 6\h /5m moins fort que 3\h\ suivi de 4\c\ (5+/5)\\
\>Ex: \s\ 2 \h\ ADV987 \d\ 8 \c\ ARDT2\-\-\\
2\s \> Bicolore avec 4+\s \\
\>Unicolore \s\ balancé 22-25\+\\
2NT \> relais\+\\
3\c \> 5+\c , =4\s \\
3\d \> 5+\d , =4\s \\
3\h \> 5+\s , 4+\c \\
3\s \> 5+\s , 4+\d \\
3NT \> Unicolore \s\ Gambling\\
4\c\d \> 6\s /5m moins fort que 3\s\ suivi de 4\c\ (5+/5)\-\-\\
2NT \> 22+ - 24H\\
3\c\d \> Unicolores ou bicolore mineur (la collante est ambigue)\\
3\h\s \> Unicolores irréguliers\\
3NT \> Gambling \c \\
4\c\d \> Bicolores 6/5m \h\ (6\h -5\c /\d\ + faible que 2\c -2\d -2\h -2\s -4\c /\d )\\
4\h\s \> Bicolores 6/5m \s\ (pareil)\-\\
2\h\s3\c \> F1T (pas forcing en paires !)\\
\>Nouvelle couleur GF\+\\
2NT \> Max misfit\\
3\d \> Min\\
3x/4x \> Fit, Splinters mains faibles\-\\
2NT \> Relais F1 avec espoir de manche en face d’un faible à \d \+\\
3\c \> faible \d\ et une courte\+\\
3\d \> NF courte s’annonce par paliers si max\\
\>(=> Passe si min du faible, annonce la courte par palier si max du faible)\-\\
3\d \> faible \d\ sans courte\\
3\h \> Max pièce \c /\h , pas de courte\+\\
3P \> relais\+\\
3NT \> court \c \\
4\c \> court \h \-\-\\
3\s \> Max force sans courte\\
3NT \> beauc \d\ ARD ou ARVT sans courte\\
4X \> GF\-\\
3\d \> 7-11H (Forcing > 4NT si main forte, mêmes dvpts)\\
3\h\s4\c \> NAT + Fit\-
\end{bidtable}

\subsection{Avec les mains TRICOLORES, on reparle toujours sur 3NT.}

Ex : \s\ - \h\ ADV8 \d\ ARD98 \c\ ARV10
(2\c -2\d -2\h -2\s -3\d -3NT-4\c -...)

\subsection{Après interventions}

\begin{itemize}
\item Contre punitif

\item Toute enchère F1T

\item 2NT R fort

\item Cue demande d'arret

\end{itemize}

\subsection{Après contre}

\begin{itemize}
\item Passe des \c !

\item Surcontre FORT ou UNICOLORE

\item 2\d\ ambigu

\item Nouvelle couleur F1T

\end{itemize}

\section{Ouverture de 2\pdfd}

\subsection{Définition}

\begin{itemize}
\item Unicolore FI

\item Bicolore Majeur FI ou FM

\end{itemize}

\begin{bidtable}
2\d-2\h\\
2\s \> Montre un bicolore majeur \textbf{+ long à \s }\+\\
2NT \> Relais\+\\
3\h \> 5/5 MAJ FI 18-21 PH\\
3\s \> 6\s /4\h\ FI 18-21 PH\\
3NT \> 5422 FM\\
4\c\d \> Tricolore parfait\\
4\h \> 65 FM\\
4\s \> 74 FM\\
3\d \> 5431 FM\\
3\c \> FM ambigu\+\\
3\d \> Relais\+\\
3\h \> 55 FM\\
3\s \> 64 FM\\
3NT \> 5413 FM\-\-\-\\
3\c\d \> Aucun intérêt pour les majeures, naturel (pas belle 6e)\\
3\h\s \> TDC\\
4\h\s \> NF\-
\end{bidtable}

\begin{bidtable}
2\d-2\h\\
2NT \> Montre un bicolore majeur \textbf{+ long à \h }\+\\
3\c \> Relais\+\\
3\d \> Résidu mineur 31 ou 30\+\\
3\h \> Relais\+\\
3\s \> 4513\\
3NT \> 4531\\
4X \> 4603 ou 4630\-\-\\
3\h\\
3\s \> 65 FM\\
3NT \> 5422 FM\\
4\c\d \> Tricolore parfait\\
4\h \> 74 FM\-\-
\end{bidtable}

\begin{bidtable}
2\d-2\h\\
3\c\d\h\s \> ACOL naturel (=FI) Collante ambigue, éventuellement pour tenter de jouer 3NT de la main forte\+\\
3\c\+\\
3\s \> Ambigu\+\\
3NT \> pour jouer\+\\
4\h \> TDC avec cue \s \\
4NT \> Quantitatif\-\-\\
4X \> Cue, TDC fit \h \-\\
3NT \> 9-10 plis en \d\ avec au moins 2 grosses pièces extérieures\-
\end{bidtable}

\begin{bidtable}
2\d\+\\
2\s \> Naturel 5 cartes avec deux gros honneurs ou 5/5 correct\\
3\c\d\h\s \> Transferts pour unicolores Q+ avec 2 gros honneurs\\
\>Le soutien ou la rectification du texas montre un intérêt pour la couleur du répondant\\
2NT \> 5/5 Mineur FM\-
\end{bidtable}

\subsection{Après intervention en 2e}

\subsubsection{Contre}

\begin{itemize}
\item Punitif si intervention au palier de 2

\item Appel si intervention au palier de 3 ou plus

\end{itemize}

\subsubsection{Passe RAS}

Ensuite, contre de l'ouvreur est punitif au niveau de 2 et montre le bicolore majeur ou une demande d'arrêt au niveau de 3.

\subsubsection{3\pdfc\pdfd\pdfh\pdfs\ Transferts (Transfert cue - Tricolore court)}

\subsubsection{2NT Bicolore mineur}

\subsubsection{Après 2\pdfd\ - X}

\begin{itemize}
\item Passe : envie de jouer

\item Tout le reste sauf 2\h\ : Naturel

\item XX : Naturel (4 belles cartes, pas 5 ou 6)

\end{itemize}

\subsection{Après intervention en 4e}

\begin{itemize}
\item PASSE, CUE-BID et ENCHERES A SAUT montrent le BICOLORE MAJEUR JUSQU’À 3\s .

\end{itemize}

\begin{itemize}
\item CONTRE demande l’arrêt

\end{itemize}

\subsection{Corollaires (à vérifier avec Gazilli !)}

\begin{bidtable}
1\s-1NT\\
3\h \> montre un 5=/4= dans la zone 18-21H
\end{bidtable}

\begin{bidtable}
1\s-1NT\\
4\h \> montre un 6/5 de 14 à 17H
\end{bidtable}

\begin{bidtable}
1\h-1NT\\
3\s \> montre un 6/5 de 14 à 17H, c’est NF !
\end{bidtable}

\section{Ouverture de 2\pdfh\ et 2\pdfs}

A compléter, pour cette année => Voir livre rouge

\section{Ouverture de 2NT}

Cette ouverture peut contenir une majeure 5e.

\begin{bidtable}
2NT---\\
3\c \> Puppet Stayman\\
3\d \> Texas \h \\
3\h \> Texas \s \\
3\s \> Texas pour 3N\\
3NT \> 5\s\ et 4\h \\
4\c \> Bicolore mineur TDC\\
4\d \> Bicolore majeur, limité à la manche ou certitude de chelem
\end{bidtable}

\subsection{Le Puppet Stayman}

\begin{bidtable}
2NT-3\c \> ---\\
3\d \> 4\h\ et/ou 4\s \+\\
3\h \> 4 cartes à \s \+\\
3\s \> fit \s \+\\
4X \> envie de chelem\-\\
3NT \> 4 cartes à \h\ (il est donc nécessaire de passer par 3\h\ avec\\
\>les deux majeures 4e et des envies de chelem)\\
\>4X fit avec envie de chelem\-\\
3\s \> 4 cartes à \h\ et pas 4 cartes à \s \\
3NT \> 3-\h\ et 3-\s \\
4\d \> 4\h\ et 4\s\ sans envie de chelem\-\\
3\h \> 5\h \\
3\s \> 5\s \\
3NT \> 3-\h\ et 3-\s 
\end{bidtable}

\subsection{Les Texas}

\begin{bidtable}
2NT---\\
3\d\+\\
3\h \> 2\h \\
3\s \> 5\s\ et 2\h \+\\
4\c/4\d \> ctrl \c /\d\ et fit \s \\
4\h/4\s \> Arrêt\-\\
3NT \> 3+\h\ et ctrl \s \\
4\c \> 3+\h\ et ctrl \c\ (sans ctrl \s )\\
4\d \> 3+\h\ et ctrl \d \\
4\h \> 3+\h\ et tous les ctrl\-
\end{bidtable}

\begin{bidtable}
3\h\+\\
3\s \> 2\s \\
3NT \> 5\h\ et 2\s \+\\
4\c/4\d \> ctrl \c /\d\ et fit \h \\
4\h/4\s \> Arrêt\-\\
4\c \> 3+\s\ et ctrl \c \\
4\d \> 3+\s\ et ctrl \d\ (sans ctrl \c )\\
4\h \> 3+\s\ et ctrl \h \\
4\s \> 3+\s\ et tous les ctrl\-
\end{bidtable}

\begin{bidtable}
3\s\+\\
3NT\+\\
4\c/4\d \> Envie ou certitude de chelem à \c /\d . L'ouvreur décourage par 4SA.\\
4\h/S \> Envie de chelem à \h /\s . L'ouvreur passe ou demande les clés.\\
\>Avec certitude de chelem, il faut passer par un texas.\-\-
\end{bidtable}

\subsection{Autres réponses}

\begin{bidtable}
4\c\+\\
4\d \> fit \d \\
4\h \> fit \c\ et ctrl \h \\
4\s \> fit \c , ctrl \s , pas de ctrl \h \-
\end{bidtable}

\begin{bidtable}
4\d \> L'ouvreur choisit sa majeure
\end{bidtable}

\begin{bidtable}
4NT \> Minimum, l'ouvreur passe.\\
\>Maximum, les réponses sont les suivantes :\+\\
5\c \> 1 ou 4 As\\
5\d \> 0 ou 3 As\\
5\h \> 2 As et 5C\\
5\s \> 2 As et 5D\\
5NT \> 2 As et pas de min 5e\-
\end{bidtable}

\section{Les barrages et ouvertures à haut palier}

\subsection{Namyats}

\begin{bidtable}
4\c\d \> Cette ouverture montre une couleur \h /\s\ 8e fermée ou 7e fermée avec un as.
\end{bidtable}

La rectification à la manche est un arrêt et l'enchère juste au-dessus montre un intérêt pour le chelem et demande à l'ouvreur de

\begin{itemize}
\item Nommer la manche sans as annexe

\item Nommer son as annexe

\end{itemize}

\subsection{3SA}

Mineure 8e fermée sans rien à côté.

4/5\c \d\ PoC

\section{Enchères après passes}

\section{Enchères de chelem}

\section{Défense}

\subsection{Sur le SA fort}

\subsection{Sur le SA faible}

Mohan : Le contre montre une bonne main, meilleure que l'ouverture d'1NT adverse (15+ HCP, à adapter aux vulnérabilités et aux atouts de la main). En réveil, la force du contre doit être équivalente.

Ensuite, le partenaire enchérit comme si le partenaire avait ouvert d'1NT (en incluant le Stayman faible ? \textbf{à discuter}).

Après le contre, le camp adverse ne peut jouer un contrat non contré en-dessous de 2S. De plus, le premier contre par chaque joueur en situation forcing est d'appel.

Sur 1SA - X - 2x :

\begin{itemize}
\item contre : des points et court dans la couleur adverse

\item passe : des points, mais 3+ cartes dans la couleur adverse

\item 2y : faible, pas une main pour défendre

\end{itemize}

Autres réponses :

\begin{bidtable}
2\c \> Landy\+\\
2\d \> Choisis ta majeure\+\\
2M\+\\
3\d \> fit la majeure nommée, FM\-\-\\
2\h/S \> Forcing passe\\
2NT \> Nat NF\\
3\c/D \> Nat NF\-\\
2\d \> Texas H\+\\
2\s/3\c \> Nat NF\\
2NT \> Invit\-\\
2\h \> Texas S\\
2\s \> S et une Mineure\+\\
2NT \> Demande la mineure\-\\
2NT \> Les deux mineures
\end{bidtable}

\section{Enchères compétitives}

\section{Jeu de la carte}

\subsection{L'entame}

\subsubsection{Entame à SA}

Principe de la quatrième meilleure :

\begin{itemize}
\item Quatrième carte d'une couleur contenant un honneur (le 10 n'est pas considéré comme un honneur, sauf, 109xx ou 10 cinquième)

\end{itemize}

\begin{itemize}
\item Tête de séquence d'une séquence d'au moins trois cartes dont un honneur

\end{itemize}

\begin{itemize}
\item Séquence brisée, le plus gros honneur du début de séquence (A/R/D 109xx - le 10)

\end{itemize}

\begin{itemize}
\item Top of nothing : \textbf{x}xx - la plus grosse; x \texttt{x} x \emph{x}

\end{itemize}

\begin{itemize}
\item Séquence = 3 cartes avec au moins 1 honneur (àpd T).

\end{itemize}

\paragraph{Entame dans sa propre couleur longue:}

\begin{itemize}
\item Séquence complète : tête de séquence

\item Séquence incomplète : tête de séquence

\item Séquence brisée RVT-DT9

\end{itemize}

\paragraph{Couleur de 3 cartes avec mini séquence:}

\begin{itemize}
\item Tête de séquence

\end{itemize}

\paragraph{Entame dans la couleur des adversaires (pas le choix) :}

\begin{itemize}
\item Si celle du déclarant : la couleur doit être longue et belle

\item Si celle du mort : si on a peu de cartes et pas d’honneur dans la couleur (doubleton avec honneur OK)

\end{itemize}

\paragraph{Pas de séquence dans couleur longue:}

\begin{itemize}
\item Entame 4ème meilleure (promet au moins 1 honneur)

\end{itemize}

\paragraph{Si 2 couleurs longues:}

\begin{itemize}
\item Prendre la plus belle (= la plus facilement affranchissable)

\end{itemize}

\paragraph{Si couleur longue mais pas belle:}

\begin{itemize}
\item On entame la 2ème (9753 : on prend le 7)

\end{itemize}

\paragraph{Dans la couleur du partenaire:}

\begin{itemize}
\item Avec 2 cartes qui se suivent avec honneur, tête de séquence SINON pair-impaire

\item Si couleur 4ème du partenaire, on entame la 2ème (Alain entame la 3ème)

\end{itemize}

\paragraph{Couleur de 3 cartes sans séquence:}

\begin{itemize}
\item pas d’honneur : la + haute

\item si 1 honneur : celle du milieu
  Remarque : Qd on entame un honneur c’est toujours une tête de séquence, jamais une 4ème meilleure

\end{itemize}

\paragraph{Entames particulières à SA}

\begin{itemize}
\item AS : (on promet le roi et peu de cartes) Qd on estime qu’on n’a pas d’entame ailleurs évidente : on regarde le mort, ce qu’il se passe et la carte du partenaire.

\item ROI : on a au moins 5 cartes et au moins 3 honneurs, demande le déblocage de la part du partenaire.

\end{itemize}

Attention exception : couleur 5ème sans reprise et qu’on espère une reprise du partenaire, on peut entamer petit

\begin{itemize}
\item Dame (DVT, DV9, ou si on a le roi = RD98)

\end{itemize}

\subsubsection{Entame à la couleur}

\paragraph{Bonne entame:}

\begin{enumerate}
\item AS R

\item Singleton

\item Couleur du partenaire

\item Tête de séquence

\end{enumerate}

\paragraph{Entame à éviter:}

\begin{enumerate}
\item Sous l’as

\item L’as sans le roi

\item Singleton atout

\item Honneur doubleton (V second à la rigueur)

\item Fourchette d’honneur

\end{enumerate}

\paragraph{Entame neutre:}

\begin{enumerate}
\item Atout (quand on a des petits atouts) : entame bonne si fit 44 chez les adversaires (le Roi 3e peut par exemple être une bonne entame dans ce cas). Faire attention à la séquence d’enchères (exemple : si un fit différé a été annoncé, on a probablement une longue et belle couleur chez les adversaires, pouvant générer beaucoup de levées, une entame agressive peut du coup être plus indiquée qu’une entame neutre)

\item Tête de mini séquence

\item Parité

\end{enumerate}

\paragraph{Entame particulière à la couleur}

\begin{enumerate}
\item AS : promet le roi et couleur courte

\item ROI : promet AS ou Dame et une couleur plutôt longue

\end{enumerate}

\subsubsection{Carte en 3ème à SA}

\paragraph{Carte obligatoire:}

\begin{itemize}
\item La plus haute mais la plus basse des équivalentes (en regardant les cartes du mort)

\item Déblocage sur entame du roi

\end{itemize}

Remarque : si mort a un singleton on n’est pas obligés de débloquer et on appelle par une petite

\subsubsection{Signalisation:}

\paragraph{Attitude:}

\begin{itemize}
\item J’aime bien ou je n’aime pas. On appelle par des petites (sur entame d’As ou de D), si pas de carte obligatoire à fournir.

\end{itemize}

Ex : Entame Dame et il y 3 petit au mort et j’ai T82, je dois appeler du 2 !!!
Ex : T, R du mort, J’ai D3ème alors je mets une petite carte
Attention, s’il faut jeter une grosse carte elle doit être sans intérêt !!!

\paragraph{Parité}

Ex : entame R, je n’ai rien à débloquer alors je montre ma parité

\paragraph{Sur l’entame d’une petite carte}

\begin{itemize}
\item honneur absent : la plus haute

\item honneur fourni : la plus haute (si on entame d’une petite et qu’on fournit un gros honneur du mort, avec une séquence d’honneur on fournit la + haute)

\item honneur non fourni : garder un honneur, carte équivalente
Attention remarque… Mais quoi ?

\end{itemize}

\paragraph{Sur l’entame d’un honneur}

\begin{itemize}
\item ROI : Déblocage, Parité

\item AS : on appelle avec des couleurs longues (le partenaire est court) et pas des honneurs

\item DAME, VALET : Attitude (on dit qu’on aime quand on a des honneurs car son entame provient souvent d’une longue)

\end{itemize}

\paragraph{Le mort est maitre}

\begin{itemize}
\item Attitude

\item la + haute des équivalentes

\end{itemize}

\paragraph{Le mort est court}

\begin{itemize}
\item attitude

\item le plus haut des honneurs équivalent

\end{itemize}

\paragraph{Carte en 3ème position à la couleur}

\begin{itemize}
\item Carte obligatoire : la plus haute

\item Signalisation : attitude (on appelle avec un honneur), parité, préférence (en Lavinthal, concerne les autres couleurs, utilisée si singleton au mort)

\item Autre possibilité de préférence (attention, délicat à utiliser) : si le partenaire sait qu’on possède beaucoup de cartes dans la couleur (vu la séquence d’enchères), alors on montre la préférence avec des cartes paires. Avec les impaires, soit pas de préférence, soit un intérêt pour raccourcir le mort. Attention aux choix forcés !

\end{itemize}

\paragraph{Sur entame d’une petite carte}

\begin{itemize}
\item Honneur absent : la + haute

\item Honneur fourni : la plus haute

\item Honneur non fourni : attitude

\end{itemize}

\paragraph{Sur l’entame d’un honneur}

\begin{itemize}
\item Roi : Parité, préférence

\item As : Attitude (on appelle si on a la dame), préférence

\item Dame valet : attitude, préférence (si le mort est court)

\end{itemize}

\paragraph{Si le mort est maitre}

\begin{itemize}
\item Attitude

\item le + haut des honneurs équivalents

\end{itemize}

\paragraph{Si le mort est court}

\begin{itemize}
\item Préférence

\end{itemize}

\subsubsection{Analyse de la première levée}

\begin{itemize}
\item Enchère

\item Carte du partenaire

\item Carte obligatoire

\item Signaux

\item Déclarant

\item Plan général de la défense
Remarque : Il faut se faire une idée du jeu de partenaire dès la première carte fournie et ne pas attendre la 2ème carte, déjà réfléchir à son plan de défense pendant que le déclarant fait son plan de jeu (ainsi que se décider sur quelle carte mettre lorsqu’il va attaquer certaines couleurs)…

\end{itemize}

\end{document}
