\documentclass[a4paper]{article}
\usepackage[T1]{fontenc}
\usepackage[utf8]{inputenc}
\usepackage{newcent}
\usepackage{helvet}
\usepackage{graphicx}
\usepackage[pdftex, pdfborder={0 0 0}]{hyperref}
\frenchspacing

\include{bml}
\title{Système Bigfish}
\author{Le bigfish et le bon sens}
\begin{document}
\maketitle
\tableofcontents

\section{Introduction}

\subsection{Double-Deux}

Après toute séquence 1X (sauf 1\c )- 1Y - 1Z, 2\c , 2\d\ et 2NT sont artificiels. 2\c\ montre
une main faible à carreau ou des mains invites ou une main de manche avec 5 cartes
dans sa majeure. 2\d\ est un relais FM et demande à l'ouvreur de se décrire.
2NT montre des mains faibles avec des trèfles ou des mains chelemisantes avec plus de
trèfles que de cartes dans la première couleur.
Les enchères au niveau de 3 sont chelemisantes et naturelles, et un bicolore cher
au niveau de 2 est FM, a priori avec un 6-5

\subsubsection{L'ouvreur dit 1NT}

\begin{bidtable}
1\c-1\h-1NT\+\\
2\c \> Relais pour 2\d , invit ou mieux\\
2\d \> Relais FM\\
2\h \> Pour jouer\\
2\s \> TDC 6\h\ 5\s \\
2NT \> Faible avec des trèfles ou FM fitté trèfle avec exactement 4\h .\+\\
Force \> le partenaire à dire 3\c \-\\
3\c \> TDC fitté trèfles avec 5 coeurs\\
3\d \> TDC 5-5. 3\h\ montre le fit, 3NT est négatif, le reste est contrôle fitté carreaux\\
3\h \> TDC 6 mauvaises cartes. Avec 6 belles cartes on aurait dit 2\h\ sur 1\c \\
3\s \> 4 piques 5 coeurs, pour laisser le choix entre 3SA, 4C et 4P\\
3NT \> Pour les jouer\\
4\h \> Pour les jouer\\
5T \> Pour les jouer\\
4NT \> Invit\\
5NT \> Invit pour 7\-
\end{bidtable}

Après un 2\c\ - 2\d , les réponses du répondant sont naturelles:

\begin{bidtable}
1\c-1\h-1NT-2\c-2\d\+\\
2\h \> Invit avec 5\h . Une nouvelle couleur de la part de l'ouvreur montre une force et\\
\>est une enchère d'essais. 2NT est enchère d'essais généralisée fittée.\\
\>3NT et 4\h\ sont pour jouer.\\
2\s \> Invit avec 6\h\ 5\s \\
2NT \> Invit sans 5\h \\
3\c \> Invit fitté trèfles\\
3\d \> Invit 5-5\\
3\h \> Invit 6 cartes\\
3NT \> 5 coeurs, pour laisser le choix entre 3NT et 4\h \-
\end{bidtable}

\begin{bidtable}
1\d-1\s-1NT-2\c-2\d\+\\
Pass \> Pour les jouer\\
2\h \> 5\s\ 4\h\ invit\\
2\s \> Invit avec 5 cartes\\
2NT \> Invit sans 5 cartes\\
3\c \> 5-5 invit\\
3\d \> Invit avec 4+ carreaux\\
3\h \> 5-5 invit\\
3\s \> Invit avec 6 cartes\\
3NT \> 5\s , pour laisser le choix entre 3NT et 4\s \-
\end{bidtable}

Après un 2\d , les réponses de l'ouvreur sont naturelles, avec l'enchère impossible
qui montre une main avec une courte qu'on ne peut pas annoncer autrement:

\begin{bidtable}
1\c-1\h-1NT-2\d\+\\
2\h \> Trois cartes\\
2\s \> 3145\\
2NT \> 3244\\
3\c \> 3235\\
3\d \> 2245\-
\end{bidtable}

\begin{bidtable}
1\c-1\s-1NT-2\d\+\\
2\h \> Quatre cartes\\
2\s \> Trois cartes\\
2NT \> 2344\\
3\c \> 2335\\
3\d \> 2245\\
3\h \> 1345\-
\end{bidtable}

Après un 2NT, l'ouvreur doit dire 3\c . Le répondant passe avec une main faible
et reparle avec une main chelemisante. Les réponses sont naturelles et les
continuations aussi:

\begin{bidtable}
1\d-1\s-1NT-2NT-3\c\+\\
Pass \> Pour les jouer\\
3\d \> Quelque chose comme 4135. L'ouvreur dit 3NT si pas intéressé, 3\s\ si\\
\>trois beaux piques, 4\c\ pour chelemisant à trèfles, ou 3\h\ ou 4\d\ pour\\
\>chelemisant à carreaux.\\
3\h \> Quelque chose comme 4315. De même, l'ouvreur donne sa couleur de préférence\\
\>ou 3NT si pas intéressé\\
3\s \> 6\c\ 5\s \\
3NT \> 4225 non forcing\\
3\c \> 6 trèfles, demande à l'ouvreur de nommer les contrôles.\\
4NT \> 4225 invit\\
5NT \> 4225 invit pour 7\-
\end{bidtable}

\subsubsection{L'ouvreur ne dit pas 1NT}

L'ouvreur peut encore avoir 18-19 points, donc il n'est pas obligé de rectifier
les transferts de 2\c\ ou de 2NT s'il a peur que le partenaire passe.

\begin{bidtable}
1\c-1\h-1\s\+\\
1NT \> 6-10\\
2\c \> Invit ou 12-14 (15) avec 5 coeurs. Ici l'ouvreur n'a pas peur que le partenaire\\
\>passe la réponse de 2\d . En effet, il a promis plus de coeurs que de carreaux.\\
2\d \> Relais FM\\
2\h \> 6-9 (10) avec 6 cartes\\
2\s \> 6-9 (10) avec 4 cartes\\
2NT \> 5+\c , plus de trèfles que de coeurs, faible ou TDC\\
3\c \> TDC 5\h\ 5\c\ (Il faut 5 trèfles car l'ouvreur n'en a promis que 3)\\
3\d \> TDC 5\h\ 5\d \\
3\s \> TDC 4 cartes\\
3NT \> 12-14 (15) sans 5 coeurs\-
\end{bidtable}

\begin{bidtable}
1\c-1\h-1\s-2\c-2\d\+\\
2\h \> Invit 5 cartes\\
2\s \> Invit 4 cartes\\
2NT \> Invit sans 5\h\ ni 4\s \\
3\c \> Invit avec 5 trèfles et 4 coeurs\\
3\d \> Invit 5-5\\
3\h \> Invit 6 cartes\\
3\s \> Invit 4 piques et 5 coeurs\\
3NT \> 12-14 (15) avec 5\h \-
\end{bidtable}

\begin{bidtable}
1\c-1\h-1\s-2\d\+\\
2\h \> 3 cartes\\
2\s \> 6\c\ 5\s \\
2NT \> 4\s\ 5\c\ 15-17 sans 3 coeurs\\
3\c \> 4\s\ 5\c\ 12-14 sans 3 coeurs\\
3\d \> 4045\\
3\h \> TDC 3 cartes\\
3\s \> TDC 6\c\ 5\s \\
3NT \> 4234 12-14\-
\end{bidtable}

\begin{bidtable}
1\c-1\h-1\s-2NT
\end{bidtable}

Ici l'ouvreur peut ne pas rectifier le transfert s'il est 18-19. Surtout que si c'est le cas, il est irrégulier
et il y a donc au moins un fit 10e à trèfle. La 4e couleur demande l'arrêt.

\begin{bidtable}
1\d-1\h-1\s-2\c
\end{bidtable}

De nouveau si l'ouvreur est 18-19 il peut ne pas rectifier le transfert pour ne pas jouer 2\d\ lorsque 3NT gagne.

\section{Ouverture de 1NT}

L’ouverture de 1SA se fait avec une main régulière de 15-17H. 
Les mains 5m-4\h 22 et 54m22 de 15-17H sont également recommandées pour cette 
ouverture.
Elle peut comporter une majeure cinquième de 15-16H dans une main régulière. Il est 
conseillé alors que le doubleton comporte un honneur. L’autre majeure troisième est un plus 
également

\begin{bidtable}
1NT\+\\
2\c \> Stayman\\
2\d \> Texas \h \\
2\h \> Texas \s \\
2\s \> ambigu : 8-9 H REG ou Texas \c \\
2NT \> Texas \d \\
3\c \> Puppet Stayman\\
3\d \> 6 belles \d , 6-7 H sans singleton, NF.\\
3\h \> 5/4m court \h , FM\\
3\s \> 5/4m court \s , FM\\
3NT \> Pour jouer\\
4\c \> Bic mineur\+\\
4\d \> fit \d \\
4\h\s \> fit \c \\
4NT \> Coup de frein\-\\
4\d \> Bic M : certitude de manche ou de chelem\\
4\h\s \> Pour jouer\\
4NT \> Quantitatif\\
5\c\d \> Pour jouer\-
\end{bidtable}

\subsection{Stayman}

Pas utilisé avec des main régulières, FM, 1 seule majeure 4e.
Utilisé avec une majeure 5e dans la zone 8-9H.

\begin{bidtable}
1NT-2\c\\
2\d \> Pas de majeure 4e\+\\
2\h \> Misère dorée : 8-9H et 5\h \\
2\s \> Misère dorée : 8-9H et 5\s \\
2NT \> 8-9HL, NF\\
3\c\d \> 5\c /\d , FM, problème pour 3SA ou TDC\\
3\h \> 5\s /4\h , FM\\
3\s \> 5\h /4\s , FM\\
4m \> BIC M TDC, court m\+\\
4M \> coup de frein\\
4NT \> BW \h \-\\
4NT \> Quantitatif\-
\end{bidtable}

\begin{bidtable}
2\h\+\\
2\s \> Misère dorée : 8-9H et 5\s \\
2NT \> 8-9HL, NF\\
3\c\d \> 5\c /\d , FM, problème pour 3SA ou TDC\\
3\h \> fit \h\ NF\\
3\s \> Fit \h\ TDC\\
4\c\d \> Splinter\\
4NT \> Quantitatif\-\\
2\s\+\\
2NT \> 8-9HL, NF. L'ouvreur dit 3\h\ max avec 3\h\ pour retrouver le fit en cas de misère dorée\\
3\c\d \> 5\c /\d , FM, problème pour 3SA ou TDC\\
3\h \> fit \s\ TDC\\
3\s \> Fit \s\ NF\\
4\c\d \> Splinter\\
4NT \> Quantitatif\-\\
2NT\+\\
3\c \> Texas \h \\
3\d \> Texas \s \\
3\h \> TDC\\
3\s \> TDC\\
4\c \> Texas \h \\
4\d \> Texas \s \\
4\h\s \> to play\-\\
3\c \> 5\s\ 4\h 
\end{bidtable}

\subsection{1NT - 2\pdfd\ : Texas \pdfh}

Remarques :

\begin{itemize}
\item Avec 4 cartes à \h\ et minimum, 4333 on rectifie le plus souvent à 3\h\ 

\item avec 4 cartes à \h\ et maximum, on rectifie à 2NT

\end{itemize}

\begin{bidtable}
1NT-2\d\\
2\h\+\\
2\s \> Bicolore 5\h /5x dans la zone 6-7H pour trouver une manche miracle\+\\
2NT \> Relais\+\\
3\c \> \h /\c \\
3\d \> \h /\d \\
3\h \> \h /\s \-\-\\
2NT \> Relais FM. Soit BIC 5\h /4m, soit Unicolore \h\ TDC\+\\
3\c \> 2 cartes à \h \+\\
3\d \> Singleton \s \+\\
3\h\+\\
3\s \> les \c \\
3NT \> les \d\ NF\\
4\c \> les \d\ Forcing\-\-\\
3\h \> Singleton \c \\
3\s \> singleton \d \\
3NT \> 5\h 4m22 honneurs concentrés\\
4\c\d\h \> contrôle, unicolore \h\ TDC\-\\
3\d \> Fit \h , max\+\\
3NT \> enclenche les contrôles\\
3\s4\c\d \> court\-\\
3\h \> Fit \h , min\+\\
3NT \> enclenche les contrôles\\
3\s4\c\d \> court\-\\
3\s \> 5 cartes à \s\ sans 3 cartes à \h \-\\
3m \> BIC 5+\h /5+m, TDC (au moins 11H concentrés).\\
\>Sans ambition de chelem, le 5/5 sera traité comme un 5/4\\
3\s/4\c/4\d \> Splinter\\
4NT \> Quantitatif 16-17 HL - 5332\-
\end{bidtable}

\subsection{1NT - 2\pdfh\ : Texas \pdfs}

Remarques :

\begin{itemize}
\item avec 4 cartes à \s\ et maximum, on rectifie à 2NT

\item Avec 4 cartes à \s\ et maximum 2SA, ensuite 4 \h\ est Texas.

\end{itemize}

\begin{bidtable}
1NT-2\h\\
2\s\+\\
2NT \> Relais FM. Soit BIC 5\s /4m, soit Unicolore \s\ TDC\+\\
3\c \> 2 cartes à \s \+\\
3\d \> Singleton \h \+\\
3\h\+\\
3\s \> les \c \\
3NT \> les \d\ NF\\
4\c \> les \d\ Forcing\-\-\\
3\h \> Singleton \c \\
3\s \> singleton \d \\
3NT \> 5\s 4m22 honneurs concentrés\\
4\c\d\h \> contrôle, unicolore \s\ TDC\-\\
3\d \> Fit \s , max\+\\
3NT \> enclenche les contrôles\\
3\s4\c\d \> court\-\\
3\h \> 5 cartes à \h\ sans 3 cartes à \s \\
3\s \> Fit \s , min\+\\
3NT \> enclenche les contrôles\\
4\c\d\h \> court\-\-\\
3m \> BIC 5+\s /5+m, TDC (au moins 11H concentrés).\\
\>Sans ambition de chelem, le 5/5 sera traité comme un 5/4\\
3\h \> BIC majeur fort\\
4\c\d\h \> Splinter\\
4NT \> Quantitatif 16-17 HL - 5332\-
\end{bidtable}

\subsection{1NT-2\pdfs\ : Texas \pdfc\ ou 8-9H régulier}

\begin{bidtable}
1NT-2\s\\
2NT \> mini\+\\
Pass \> 8-9H réguliers\\
3\c \> SO\\
3\d \> 5/5 mineur FM\\
3\h \> singleton \s \\
3\s \> singleton \h \\
3NT \> singleton \d \\
4\d \> singleton \d\ ambition de chelem\\
4\h \> 6\c /5\h\ NF\\
4\s \> 6\c /5\s\ NF\-\\
3\c \> maxi\+\\
3NT \> 8-9 réguliers\\
3\d \> singleton \d\ ou 5/5 mineur\+\\
3\h \> relais\+\\
3\s \> 5/5 mineur\\
3NT \> singleton \d \\
4\d \> singleton \d\ ambition de chelem\-\-\\
3\h \> singleton \s \\
3\s \> singleton \h \\
4\h \> 6\c /5\h\ NF\\
4\s \> 6\c /5\s\ NF\-
\end{bidtable}

\subsection{1NT-2NT : Texas \pdfd}

Remarque : L'ouvreur peut moduler son soutien, 3\c\ max fitté et 3\d\ RAS.
Il est préférable que l'ouvreur ait une tenue à \c\ lorsqu'il répond 3\c .

\begin{bidtable}
1NT-2NT\\
3\d\+\\
Pass \> 0-6 HL\\
3\h \> singleton \s \\
3\s \> singleton \h \\
3NT \> singleton \c \\
4\c \> TDC \d , singleton \c \\
4\d \> TDC \d\ sans singleton\\
4\h \> 6\d /5\h\ NF\\
4\s \> 6\d /5\s\ NF\-\\
3\c \> maxi\+\\
3\d \> 0-6HL\\
3\h \> singleton \s \\
3\s \> singleton \h \\
3NT \> singleton \c \\
4\c \> TDC \d , singleton \c \\
4\d \> TDC \d\ sans singleton\\
4\h \> 6\d /5\h\ NF\\
4\s \> 6\d /5\s\ NF\-
\end{bidtable}

\subsection{1NT-3\pdfc\ : Puppet Stayman}

Remarque importante

\begin{itemize}
\item Forcing manche

\item Ne s'emploie pas avec des mains comportant un singleton 

\item Ne s'emploie pas avec les 2 majeures 4e

\end{itemize}

\begin{bidtable}
1NT-3\c\\
3\d \> 0, 1 ou 2 majeures 4e\+\\
3\h \> 4\s . L'ouvreur avec 4\s\ fit en disant 3\s\ (3SA = camouflage)\\
3\s \> 4\h \\
3NT \> pas de majeures 4e, pour jouer\\
4\c\d \> 5 cartes 15+, 5332\-\\
3\h \> 5 cartes à \h \+\\
3\s \> ambition de chelem à \h \\
3NT \> pour jouer\\
4\c\d \> 15+, 5332\-\\
3\s \> 5 cartes à \s \+\\
4\h \> ambition de chelem à \s \\
3NT \> pour jouer\\
4\c\d \> 15+, 5332\-\\
3NT \> 5\h /4\s 
\end{bidtable}

\subsection{1NT-3\pdfh/\pdfs\ : 5/4 mineurs court \pdfh/\pdfs}

Bicolore mineur 5/4 avec une courte dans la couleur annoncée, forcing manche avec au 
moins 9H. L’ouvreur donne 3SA avec au moins 1 arrêt et demi dans la couleur de la courte.
Sinon, il se décrit de manière naturelle. Il peut même fitter le résidu s’il possède cette majeure 5ème.

\section{Défense sur 1SA}

\subsection{Sur le SA fort}

\begin{bidtable}
(1NT)\+\\
Dbl \> Texas \c\ ou Landy (BIC M 4+/4+) si suivi de 2\d \\
\>2C Avec une préférence M (ou sans préférence avec un unicolore \c )\+\\
2\d \> Landy\+\\
2\h/2\s \> Nat NF\\
2NT \> Relais\+\\
3\c \> mini\\
3\d \> Max 5+\h /4\s \\
3\h \> Max 5+\s /4\h \\
4\c \> 5/5 court \c \\
4\d \> 5/5 court \d \-\\
3\c \> NF\\
3\d \> F1\\
3M \> Invit\-\\
2\d \> Sans préférence M (au mieux 2/2)\\
\>Le répondant dit 2\d\ avec un unicolore \d\ ou un semi-bicolore m. \\
\>L'intervenant passe, nomme une majeure cinquième ou 3\c\ naturel NF. Les autres enchères sont naturelles avec du \c .\\
2\h \> Unicolore\\
2\s \> Unicolore\\
2NT \> F1\-\\
2\c \> Texas \d . Si fort, suivi par 2M (6\d 4M), 3\c\ (5\d 4+\c ), 3\d\ unicolore\\
2\d \> Unicolore M\+\\
2\h\s \> P/C. Ensuite si maxi :\+\\
2NT \> Invit avec des \h \\
3\c\d \> Invit avec des \s \-\\
2NT \> Relais\+\\
3\c \> Max \h \+\\
3\h\s \> Invit\-\\
3\d \> Max \s \+\\
3\h\s \> Invit\-\\
3\h \> Min \h \\
3\s \> Min \s \-\\
3\c \> NF\\
3\d \> F1\\
3\h\s \> P/C\\
4\h\s \> Clôture !!\-\\
2\h \> 5\h 4m\+\\
2NT \> Relais - demande la mineure\+\\
3\c \> Min \c \\
3\d \> Min \d \\
3\h \> Max \c \\
3\s \> Max \d \\
3NT \> Max 0544\-\\
3\c \> P/C\\
3\d \> Invit à \h \\
3\h \> Pour jouer\\
3NT \> pour jouer\-\\
2\s \> 5\s 4m\+\\
2NT \> Relais - demande la mineure\+\\
3\c \> Min \c \\
3\d \> Min \d \\
3\h \> Max \c \\
3\s \> Max \d \\
3NT \> Max 5044\-\\
3\c \> P/C\\
3\h \> Invit à \s \\
3\s \> Pour jouer\\
3NT \> pour jouer\-\\
2NT \> Bicolore mineur ou bicolore Fm indéterminé\+\\
3\c\d \> Préférence\\
3\h\s \> Naturel F1, sauf si X sur 2NT\-\\
3x \> Barrage\\
3NT \> Bicolore mineur très distribué.\-
\end{bidtable}

En réveil :

\begin{bidtable}
(1NT)-Pass-(Pass)\+\\
Dbl \> 13+ et semi-régulier (au moins 44 ou 5+)\\
2\c \> Landy\\
2X \> Naturel\\
3X \> Naturel + fort\-
\end{bidtable}

\subsection{Sur le SA faible}

\begin{itemize}
\item Contre : tendance punitive. Au minimum le maximum du NT, main +- balancée. Le répondant passe avec une main plate à partir de 8H.
    2\c\ est stayman, 2\d , 2\h , 2\s\ et 2NT sont des Texas.

\item 2\c\ : Landy - Bicolore majeur au moins 5/4.

\item 2\d , 2\h , 2\s , 2NT : Texas.

\item 3X : Belle couleur 6e - environ 13-15H

\end{itemize}

En réveil : Idem.

\subsection{Réponses à l'intervention d'1NT}

\begin{bidtable}
(1\h)-1NT-(Pass)\+\\
2\c \> Texas \d \\
2\d \> Stayman\\
2\h \> Texas \s \\
2\s \> Texas \c \\
2NT \> Naturel 7+ - 9-H\\
3\c \> Texas \d \-
\end{bidtable}

\begin{bidtable}
(1\s)-1NT-(Pass)\+\\
2\c \> Texas \d \\
2\d \> Texas \h \\
2\h \> Stayman\\
2\s \> Texas \c \\
2NT \> Naturel 7+ - 9-H\\
3\c \> Texas \d \-
\end{bidtable}

\begin{bidtable}
(1\c\d)-1NT-(Pass)\+\\
2\c \> Stayman\\
2\d \> Texas \h \\
2\h \> Texas \s \\
2\s \> Texas \c \\
2NT \> Naturel 7+ - 9-H\\
3\c \> Texas \d \\
4\d \> Bicolore majeur\-
\end{bidtable}

\subsection{En réveil}

\begin{itemize}
\item 1NT : 11-14H

\item Dbl puis 1NT : 15-17H

\item 2NT : 18-19H

\item Dbl puis 2NT : 20-21

\end{itemize}

\section{Ouverture de 2\pdfc}

\subsection{Définition}

\begin{itemize}
\item FAIBLE en \d\ (constructif VUL et en 1ère et 2ème NV) et 11-13 en 4ème

\item 22-24 BAL

\item GAMBLING solide en \c\ 

\item FM sauf Bicolores Majeurs

\end{itemize}

\begin{bidtable}
2\c--\+\\
2\d \> relais\+\\
Pass \> FAIBLE \d \\
2\h \> 24+ balancé\\
\>Bicolore avec au moins 4\h \\
\>Unicolore \h\ balancé 22-25\+\\
2\s \> relais\+\\
2NT \> 25H et + (développements comme sur 2N)\\
3\c \> 5+\c , =4\h \\
3\d \> 5+\d , =4\h \\
3\h \> 5+\h , 4+\c \\
3\s \> 5+\h , 4+\d \\
3NT \> Unicolore \h\ Gambling (style régulier avec 6/7\h\ plein et 2 As)\\
4\c\d \> 6\h /5m moins fort que 3\h\ suivi de 4\c\ (5+/5)\\
\>Ex: \s\ 2 \h\ ADV987 \d\ 8 \c\ ARDT2\-\-\\
2\s \> Bicolore avec 4+\s \\
\>Unicolore \s\ balancé 22-25\+\\
2NT \> relais\+\\
3\c \> 5+\c , =4\s \\
3\d \> 5+\d , =4\s \\
3\h \> 5+\s , 4+\c \\
3\s \> 5+\s , 4+\d \\
3NT \> Unicolore \s\ Gambling\\
4\c\d \> 6\s /5m moins fort que 3\s\ suivi de 4\c\ (5+/5)\-\-\\
2NT \> 22+ - 24H\\
3\c\d \> Unicolores ou bicolore mineur (la collante est ambigue)\\
3\h\s \> Unicolores irréguliers\\
3NT \> Gambling \c \\
4\c\d \> Bicolores 6/5m \h\ (6\h -5\c /\d\ + faible que 2\c -2\d -2\h -2\s -4\c /\d )\\
4\h\s \> Bicolores 6/5m \s\ (pareil)\-\\
2\h\s3\c \> F1T (pas forcing en paires !)\\
\>Nouvelle couleur GF\+\\
2NT \> Max misfit\\
3\d \> Min\\
3x/4x \> Fit, Splinters mains faibles\-\\
2NT \> Relais F1 avec espoir de manche en face d’un faible à \d \+\\
3\c \> faible \d\ et une courte\+\\
3\d \> NF courte s’annonce par paliers si max\\
\>(=> Passe si min du faible, annonce la courte par palier si max du faible)\-\\
3\d \> faible \d\ sans courte\\
3\h \> Max pièce \c /\h , pas de courte\+\\
3P \> relais\+\\
3NT \> court \c \\
4\c \> court \h \-\-\\
3\s \> Max force sans courte\\
3NT \> beaux \d\ ARD ou ARVT sans courte\\
4X \> GF\-\\
3\d \> 7-11H (Forcing > 4NT si main forte, mêmes dvpts)\\
3\h\s4\c \> NAT + Fit\-
\end{bidtable}

Avec les mains TRICOLORES, on reparle toujours sur 3NT.

Ex : \s\ - \h\ ADV8 \d\ ARD98 \c\ ARV10
(2\c -2\d -2\h -2\s -3\d -3NT-4\c -...)

\subsection{Après intervention}

\begin{itemize}
\item Contre punitif

\item Toute enchère F1T

\item 2NT R fort

\item Cue demande d'arret

\end{itemize}

\subsection{Après contre}

\begin{itemize}
\item Passe des \c !

\item Surcontre FORT ou UNICOLORE

\item 2\d\ ambigu

\item Nouvelle couleur F1T

\end{itemize}

\section{Ouverture de 2\pdfd}

\subsection{Description}

Quatrième main : Idem mais zone faible 9-13.
1/2/3e main :

\begin{itemize}
\item Acol fort indeterminé à \c\ ou \d\ 

\item Unicolore 6e à \h\ 5-10 HCP 

\item Bicolore 5/5 \h\ + mineure, 5-10 HCP

\item Bicolore majeur FI ou FM

\end{itemize}

\subsection{Développements}

1/2/3e main

\begin{bidtable}
2\d-2\h \> Pour jouer si faible à \h\ (< 15 HCP)\+\\
Pass \> Unicolore \h\ faible ou 5/5 faible avec de beaux \h\ (rare)\\
2\s \> 4+\s /4+\h\ GF\+\\
2NT \> Relais\+\\
3\c \> Montre des \h\ + longs (5+)\+\\
3\d \> relais\+\\
3\h \> Résidu \c\ ou solo chelem 6+\\
3\s \> Résidu \d \\
3NT \> 4522\\
4\c\d \> 4504/4540\\
4\h \> 6/5 mini\-\-\\
3\d \> 5+ \s \+\\
3\h \> relais\+\\
3\s \> Résidu \c \\
3NT \> résidu \d \-\-\\
3\h \> 5+/5\\
3\s \> 6/4\\
3NT \> 5422/5431 22H+\\
4\c\d \> 5404/5440\\
4\h \> 6/5 Mini\-\\
3\c\d \> 6 très belles cartes\\
3\h\s \> 4 cartes 5DH+\\
3NT \> 5/5 ou 6/5 m faible\\
4\c\d \> Chicane\\
4\h\s \> Naturel < 5DH\-\\
2NT \> Bicolore 5+/5+ \h /\c\ ou \h /\d \+\\
3\c \> PoC\\
3\d \> PoC\\
3\s \> Relais le + fort\\
3NT \> Proposition naturelle\\
4/5\c \> PoC\\
4\h \> Chicane \h\ - TdC dans la mineure de l'ouvreur\-\\
3\c/3\d \> Acol naturel NF\\
3\h \> 5/5 Majeur FI\\
3\s \> 6\s /4\h\ FI\\
3NT \> Gambling à \d \-
\end{bidtable}

\begin{bidtable}
2\d\+\\
2\s \> Relais + misfit \h\ (max Hx), n'envisage normalement pas 4\h\ en face d'un 5/5\+\\
2NT \> 5+\c\ mini\+\\
3\c \> NF encourageant\-\\
3\c \> 5+\d\ mini\+\\
3\d \> idem\-\\
3\d \> 6\h \+\\
3\h \> NF encourageant\\
3\s \> 6 cartes à \s\ fort\-\\
3\h \> 5+\c\ max\\
3\s \> 5+/4+ M FM\\
3NT \> 5+\d\ max\\
4\c/4\d \> Acol \c /\d \\
4\h \> 5/5 M FI\\
4\s \> 6\s /4\h\ FI\-\\
2NT \> Relais fitté \h\ au moins 3 cartes, pas FM\+\\
3\c \> 5\h /5\d \+\\
3\h \> encourageant NF\-\\
3\d \> 6\h \+\\
3\h \> encourageant NF\-\\
3\h \> 5\h /5\c\ mini NF\\
3\s \> 4+/4+ M FI/FM\\
3NT \> 5\h /5\c\ max\\
4\c/4\d \> Acol \c /\d \\
4\h \> Bic M 5/5 FI\\
4\s \> 6\s /4\h\ FI\-\\
3\c/3\d/3\s \> Texas \d , \s\ et \c\ - Ce sont des enchères positives en Texas mais pas nécessairement FM.\\
\>L'ouvreur se décrit naturellement en privilégiant le fit. La rectification est négative.\+\\
3\d \> (sur 3\c ) négatif\\
3\h \> Naturel + fit \d\ (positif)\\
3\s \> Bic M\\
3NT \> Acol \c\ (sans fit \d\ ou sans ambition)\\
4\d \> Fit setting fort (Acol \c\ ou Bic M avec 3\d\ ou Hx)\-\\
3\h \> 3/4 \h\ et 7-13 HCP\+\\
Pass \> L'enchère la plus fréquente de l'ouvreur faible\\
3\s \> Bic M fort\+\\
3NT \> Négatif à \s\ et à \h , minimum\\
4\c/4\d \> Naturel F1\-\\
3NT \> Montre un Acol mineur - NF\\
4\h \> Montre une main faible avec un supplément de distribution (6/5, 6/4, ...)\\
4\c/4\d \> Acol mineur très excentré\-\\
3\s \> Pour jouer 3NT en face d'un Acol à \c\ (! double sens)\\
3NT \> Pour jouer en face d'un FI à \d \\
4\c/4\d \> Naturel, bon fit \h , 10+ HCP, l'ouvreur peut décider d'être compétitif au palier supérieur\\
4\h \> Pour jouer 4\h\ si ouvreur faible mais du jeu pour supporter le chelem si ouvreur fort\\
4\s \> Pour jouer en face d'un 2 faible ou au moins 5\h\ en face d'un jeu fort\-
\end{bidtable}

\subsection{Après intervention}

\begin{bidtable}
2\d\+\\
(Dbl)\+\\
Pass \> si tu es 5\h /5\d , je veux jouer 2\d \+\\
(Pass)\+\\
Rdbl \> 6\h \\
2\h \> 5\h /5\c \\
2\s \> Bic M fort\\
2NT \> 5\h /6\c \\
3\c \> Acol \c \-\-\\
2\h \> si tu es 5\h /5\d , je veux jouer 2\h \+\\
(Dbl)\+\\
Pass \> 6\h\ (Principe : faire jouer le répondant le + souvent possible !)\\
2\h \> 5\h /5\c \\
2\s \> Bic M fort\\
2NT \> 5\h /6\c \\
3\c \> Acol \c \-\-\\
Rdbl \> Unicolore autonome soit misfit \h\ soit semi-fit \h\ avec tolérance pour les mineures\+\\
(Pass)\+\\
2\h \> Version faible\\
2\s \> Bic M Fort\-\-\\
2\s/3\c/3\d \> Naturel F1 mais pas auto-forcing\\
3\h \> permet de dire 4\h\ (suivre le bon sens)\\
2NT \> fitté\\
3X/4\h \> System on\-\\
(Pass)\+\\
2\h\+\\
(Dbl)\+\\
Pass \> 6\h\ Principe : Faire jouer le répondant le plus souvent possible\\
Rdbl \> 5\h /5m\\
2\s \> Bic M ou 6\s /4\h\ FI (NF)\\
2NT \> 5\h\ + 6\c /6\d \\
3\c/3\d \> Acol\\
3M \> Bic M FM (et ne veut pas passer sur 2Cx)\-\-\\
2\s\+\\
(Dbl)\+\\
System \> on\-\-\-\-
\end{bidtable}

\section{Ouverture de 2\pdfh}

\subsection{Description}

1/2/3e main

\begin{itemize}
\item Acol fort indeterminé à \h\ 

\item Unicolore 6e à \s\ 5-10 HCP 

\item Bicolore 5/5 \s\ + mineure, 5-10 HCP

\end{itemize}

\begin{bidtable}
4e \> main : Ideam mais zone faible 9-13 HCP
\end{bidtable}

\subsection{Développements}

\begin{bidtable}
2\h\+\\
Pass\\
2\s \> Pour jouer si faible à \s\ (< 15 HCP)\+\\
Pass \> Unicolore \s\ faible ou 5/5 faible avec de beaux \s\ (rare)\\
2NT \> Bicolore 5+/5+ \s /\c\ ou \s /\d \+\\
3\c \> Pour jouer avec des \c\ ou corriger avec des \d \\
3\d \> Pour jouer avec des \d\ ou proposition de manche avec des \c \\
3NT \> Proposition naturelle\\
4/5\c \> PoC\\
4\h/4\s \> Chicane - TdC\-\\
3\c\d\s \> 2e couleur ou résidu 3e si la main possède une courte\\
3\h \> 6\h\ tendance 6322 18-20\\
3NT \> 6\h\ tendance régulier 21-22\-\\
2NT \> Relais fort\+\\
3\c \> 5\c\ mini\+\\
3\d \> Tendance naturel FM\\
3\h \> Fit \s\ Forcing\\
3\s \> Fit \s\ Invit\\
4\c \> Fit \c\ Forcing\-\\
3\d \> 5\d\ mini\+\\
3\h \> Fit \s\ Forcing\\
3\s \> Fit \s\ Invit\\
4\c \> Fit \d\ + contrôle \c \\
4\d \> Fit \d\ Forcing, pas de ctrl \c \-\\
3\h \> 6\s \+\\
3\s \> NF (oriente la main)\-\\
3\s \> 5\c\ maxi\\
3NT \> 5\d\ maxi\\
4\c/4\d \> Acol \h\ - Premier contrôle disponible\\
4\h \> Acol \h \-\\
3\c/3\d/3\h \> Texas \d , \h\ et \c\ - Ce sont des enchères positives en Texas mais pas nécessairent FM.\\
\>L'ouvreur se décrit naturellement en privilégiant le fit. La rectification est négative.\\
3\s \> 3/4 \s\ et 7-13 HCP\+\\
Pass \> L'enchère la plus fréquente de l'ouvreur faible\\
3NT \> Proposition de jouer 3NT sur base d'un Acol \h \\
4\s \> Montre une main faible avec un supplément de distribution (6/5, 6/4, ...)\\
4\c/4\d \> Tentative à base d'un Acol \h\ (généralement avec une courte)\-\\
3NT \> Pour jouer\\
4\c/4\d \> Naturel, bon fit \s , 10+ HCP\\
4\h \> !! Pour jouer en face d'un Acol Fort ou 4\s\ en face du 2 faible.\\
4\s \> Pour jouer en face d'un 2 faible ou au moins 5\h\ en face d'un jeu fort\-
\end{bidtable}

\subsection{Après intervention}

\begin{bidtable}
2\h\+\\
(Dbl)\+\\
Pass \> 2- \s \+\\
(Pass)\+\\
Rdbl \> 6\s\ (Principe : faire jouer le répondant le + souvent possible !)\\
2\s \> 5\s\ + 5m\-\-\\
Rdbl \> Unicolore ou fort\+\\
(Pass)\+\\
Pass \> Acol \h \\
2\s \> 6\s \\
2NT/3\c \> 5\c /5\d\ pour faire jouer le partenaire dans son unicolore.\-\-\\
2\s \> au moins 2\s \+\\
(Dbl)\+\\
Pass \> 6\s\ Toujours selon le même principe\\
Rdbl \> 5\s\ + 5m  (pas obligé sur Dbl mais bien 2\h -P-2\s )\\
2NT \> Acol \h\ + arrêt \s \\
3\c \> 5\s\ + 6\c\ faible\\
3\d \> 5\s\ + 6\d\ faible\\
3\h \> Acol \h\ pour jouer à la couleur (peu/pas d'arrêt \s )\-\-\\
2NT \> fitté\\
3X/4\h \> System on\-\-
\end{bidtable}

\section{Ouverture de 2\pdfs}

\subsection{Description}

Quatrième main : Bicolore mineur 5+/5+ 9-13 HCP ou Acol fort indeterminé à \s . 1/2/3e main

\begin{itemize}
\item Acol fort indeterminé à \s\ 

\item Bicolore mineur 5+/5+, 5-10 HCP

\end{itemize}

\subsection{Développements}

\begin{bidtable}
2\s\+\\
Pass\\
2NT \> Relais\+\\
3\c \> Bic mineur minimum\+\\
Pass\\
3\d \> Pour jouer\\
3\h/3\s \> Naturel F1\\
3NT \> Pour jouer\-\\
3\d \> Bic mineur maximum\+\\
3\h/3\s \> Naturel F1\\
3NT \> Pour jouer\-\\
3\h \> 3 cartes d'un honneur (dans la version forte)\+\\
3\s \> Fit \s \\
3NT \> Forcing\\
4\c/4\d/4\h \> Naturel (a priori misfit \s )\+\\
4\s \> Pas fitté\\
4NT \> KBB\-\\
4\s \> Pour jouer\\
4NT \> KBB\-\\
3\s \> Acol \s\ 6322 min (18-19)\+\\
3NT \> Forcing (a priori misfit \s )\\
4\c/4\d/4\h \> Naturel (a priori misfit \s , recherche de fit)\+\\
4\s \> Pas fitté\\
4NT \> KBB\-\\
4\s \> Forcing\\
4NT \> KBB à \s \-\\
3NT \> Acol \s\ 6322 max (20-21)\+\\
4\c/4\d/4\h \> Naturel (a priori misfit \s , recherche de fit)\+\\
4\s \> Pas fitté\\
4NT \> KBB\-\\
4\s \> Forcing\\
4NT \> KBB à \s \-\\
4\c \> 1156\\
4\d \> 1165\\
4\h/4\s \> 6/5 + chicane\-\\
3\c \> Pour jouer\+\\
3\d \> Acol \s\ + couleur \d\ 3+ cartes\\
3\h \> Acol \s\ + couleur \h\ 3+ cartes (forte probabilité de courte \s )\\
3\s \> Acol \s\ NF\\
3NT \> 6\s\ (3-2-2) 20-21\-\\
3\d \> Pour jouer\+\\
Cfr \> 3\c \-\\
3\h \> Naturel NF\\
3\s \> Naturel NF\\
3NT \> Pour jouer\\
4\c \> "Barrage" 4/5 cartes 7-12 HCP\\
4\d \> "Barrage" 4/5 cartes 7-12 HCP\\
4\h \> Pour jouer (aussi en face de la main faible)\\
4\s \> Pour jouer en face de la main forte\-
\end{bidtable}

\subsection{Après intervention}

\begin{bidtable}
2\s\+\\
(Dbl)\+\\
Pass \> Propose d'en rester là sur base de sa propre couleur (faible + \s )\\
Rdbl \> Propose de jouer 2\s\ XX (plus punitif éventuel ultérieur) sauf si ouvreur a 1- \s \\
2NT \> Game try en mineure ou SA, l'ouvreur développe comme sur séquence de base\\
3\c/3\d \> préférence\\
3\h \> Naturel NF\\
3\s \> N'existe pas\\
3NT \> Naturel\\
4\c/4\d \> Barrage\\
4\h \> Pour jouer\-\\
(Pass)\+\\
3\c/3\d\+\\
(Dbl)\+\\
Pass \> Main faible, le répondant peut annoncer une longue Maj\\
\>ou XX = Bic M \\
Rdbl \> Acol \s\ mini\\
3\h \> 6\s\ + 3\h \\
3\s \> Acol \s\ max\-\\
(Pass)-Pass\+\\
(Dbl)\+\\
Rdbl \> Misfit total, Bic M\\
3\h/3\s \> Longue personnelle\-\-\-\-\\
3X/4X\+\\
Pass \> Forcing pour la main forte\\
Dbl \> Punitif face à la main faible\\
4\c/4\d \> Compétitif, fit face à la main faible\\
Couleur \> Naturel F1\\
3NT/4NT \> Naturel face à la main faible\\
5\c/5\d \> Pour jouer\-\-
\end{bidtable}

\section{Ouverture de 2NT}

Cette ouverture peut contenir une majeure 5e.

\begin{bidtable}
2NT---\\
3\c \> Puppet Stayman\\
3\d \> Texas \h \\
3\h \> Texas \s \\
3\s \> Texas pour 3N\\
3NT \> 5\s\ et 4\h \\
4\c \> Bicolore mineur TDC\\
4\d \> Bicolore majeur, limité à la manche ou certitude de chelem\\
4NT \> Quantitatif
\end{bidtable}

\subsection{Le Puppet Stayman}

\begin{bidtable}
2NT-3\c \> ---\\
3\d \> 4\h\ et/ou 4\s \+\\
3\h \> 4 cartes à \s \+\\
3\s \> fit \s \+\\
4X \> envie de chelem\-\\
3NT \> 4 cartes à \h\ (il est donc nécessaire de passer par 3\h\ avec\\
\>les deux majeures 4e et des envies de chelem)\\
\>4X fit avec envie de chelem\-\\
3\s \> 4 cartes à \h\ et pas 4 cartes à \s \\
3NT \> 3-\h\ et 3-\s \\
4\d \> 4\h\ et 4\s\ sans envie de chelem\-\\
3\h \> 5\h \\
3\s \> 5\s \\
3NT \> 3-\h\ et 3-\s 
\end{bidtable}

\subsection{Les Texas}

\begin{bidtable}
2NT---\\
3\d\+\\
3\h \> 2\h \\
3\s \> 5\s\ et 2\h \+\\
4\c/4\d \> ctrl \c /\d\ et fit \s \\
4\h/4\s \> Arrêt\-\\
3NT \> 3+\h\ et ctrl \s \\
4\c \> 3+\h\ et ctrl \c\ (sans ctrl \s )\\
4\d \> 3+\h\ et ctrl \d \\
4\h \> 3+\h\ et tous les ctrl\-
\end{bidtable}

\begin{bidtable}
3\h\+\\
3\s \> 2\s \\
3NT \> 5\h\ et 2\s \+\\
4\c/4\d \> ctrl \c /\d\ et fit \h \\
4\h/4\s \> Arrêt\-\\
4\c \> 3+\s\ et ctrl \c \\
4\d \> 3+\s\ et ctrl \d\ (sans ctrl \c )\\
4\h \> 3+\s\ et ctrl \h \\
4\s \> 3+\s\ et tous les ctrl\-
\end{bidtable}

\begin{bidtable}
3\s\+\\
3NT\+\\
4\c\d \> Envie ou certitude de chelem à \c /\d . L'ouvreur décourage par 4SA.\\
4\h\s \> Envie de chelem à \h /\s . L'ouvreur passe ou demande les clés.\+\\
Avec \> certitude de chelem, il faut passer par un texas.\-\-\-
\end{bidtable}

\subsection{Autres réponses}

\begin{bidtable}
4\c\+\\
4\d \> fit \d \\
4\h \> fit \c\ et ctrl \h \\
4\s \> fit \c , ctrl \s , pas de ctrl \h \-
\end{bidtable}

\begin{bidtable}
4\d \> L'ouvreur choisit sa majeure
\end{bidtable}

\begin{bidtable}
4NT \> Minimum, l'ouvreur passe.\\
\>Maximum, les réponses sont les suivantes :\+\\
5\c \> 1 ou 4 As\\
5\d \> 0 ou 3 As\\
5\h \> 2 As et 5\c \\
5\s \> 2 As et 5\d \\
5NT \> 2 As et pas de min 5e\-
\end{bidtable}

\section{Défense sur les ouvertures Multi}

Principe général sur les ouvertures de type Multifonction :

\begin{itemize}
\item X montre soit 12-15 semi-balancé, soit 16+ si irrégulier ou 20+ si balancé.

\item Passe puis X est d'appel.

\item L'annonce de la couleur adverse (si une seule couleur connue) est d'appel (style tricolore, 4441/5431) ou bicolore cher).

\item 2X est naturel avec l'ouverture.

\item 3X est naturel avec 7-8 levées.

\item 2SA est balancé 16-19 avec arrêt(s).

\item Ensuite, sur un X d'appel et une enchère adverse en 3è, X est d'appel (du saut et transformable).

\end{itemize}

En pratique :

\subsection{2\pdfc\ multi (les deux M faibles ou fort)}

\begin{itemize}
\item X 12-15 semi-balancé ou fort irrégulier ou 20-22 balancé

\item 2\d\ naturel

\item 2M semi-saturel (4+ cartes, voire 3 belles) 13-17

\item 2NT semi-balancé 16-19 avec arrêts majeurs

\item 3X NAT 7/8 levées

\item 3SA pour jouer

\item Passe puis 2SA avec les mineures (5+ 4+)

\item Passe puis 3SA bicolore m très distribué

\end{itemize}

\subsection{2\pdfd\ Multi (unicolore M faible ou fort)}

\begin{itemize}
\item X 12-15 semi balancé, en principe 3+ 3+ dans les M, ou fort

\item 2M naturel

\item 2NT semi-BAL 16-19 avec les arrêts majeurs

\item 3X naturel, 7-8 levées

\item Passe puis X d'appel

\item Passe puis 2NT avec les mineures (5+ 4+)

\end{itemize}

Après 2\d\ - 2M, en 4è :

\begin{itemize}
\item X est d'appel sur la M et 3\c\ d'appel sur la M'

\item 2NT ou 3SA naturel

\item 3M bicolore m 

\item 4m bicolore mM'

\end{itemize}

\subsection{2\pdfh\ Ekren (les deux M faible)}

\begin{itemize}
\item X d'appel (transformable) avec 4+ \h\ 

\item 2\s\ d'appel court à \h\ (ensuite 2NT modérateur)

\item 2NT naturel 16/18 (ensuite 3m NF et 3M FM \c /\d\ respectivement)

\item 3m naturel

\item 3M (semi-)bicolore mineur fort, court M

\item 3NT naturel

\end{itemize}

\subsection{2\pdfh\ Multi (Faible \pdfs\ ou \pdfs\ + m ou fort)}

\begin{itemize}
\item X ouverture avec 5+\h\ 

\item 2\s\ contre d'appel court \s\ (ensuite 2NT modérateur)

\item 2NT naturel 16/18

\item 3x naturel

\item 3NT naturel

\end{itemize}

\subsection{2\pdfs\ Multi (les deux mineures faible ou fort)}

\begin{itemize}
\item X ouverture avec 5+\s\ 

\item 2NT naturel 16/18

\item 3\c\ contre d'appel avec + de \h\ ou égalité de longueur

\item 3\d\ contre d'appel avec + de \s\ 

\item 3M naturel

\item 3NT naturel

\end{itemize}

\subsection{Ouverture de 3X en Texas}

\begin{itemize}
\item X d'appel 12-17, court de la couleur longue annoncée

\item 3X+1 main forte avec un bicolore cher

\end{itemize}

\section{Les barrages et ouvertures à haut palier}

\subsection{Namyats 4\pdfc/4\pdfd}

Cette ouverture montre une couleur \h /\s\ 8e fermée ou 7e fermée avec un as.
La rectification à la manche est un arrêt.

L'enchère juste au-dessus montre un intérêt pour le chelem et demande à l'ouvreur de:

\begin{itemize}
\item Nommer la manche sans as annexe

\item Nommer son as annexe

\end{itemize}

\subsection{3SA}

Mineure 8e fermée sans rien à côté. 
4/5\c \d\ PoC

\section{Enchères après passe}

\begin{bidtable}
1\s\+\\
1NT \> 6-11H, misfit\\
2\c \> Drury, 3\s\ 10-12 DH/ 4\s\ jeu plat, (2\s\ avec GH et 11 points H)\+\\
2\d/2\h \> naturel avec une ouverture correcte\\
2\s \> Main n'ayant pas d'espoir de manche en face d'un passe initial\\
2NT \> 14+ 17-, jeu régulier avec des arrêts un peu partout\\
3\c/3\d/3\h \> Beau 5-5, 16+\\
3\s \> 6\s , 16/18\\
3NT \> 17+ 19, jeu régulier\-\\
2\d/2\h \> Naturel. Belle couleur souvent 6ème, 8-11 H.\\
2\s \> 6-9HL\\
2NT \> Fit 4e et un singleton\+\\
3\c \> Relais\+\\
3\d \> Singleton \d \\
3\h \> Singleton \h\ (ou \c\ si fit \h )\\
3\s \> Singleton \c\ (ou \s\ si fit \h )\-\-\\
3\c/3\d/3\h \> Rencontre : 5 belles cartes, 4+ \s , 9/11H : L'ouvreur revient à la majeur au palier qu'il souhaite ou nomme un contrôle si TDC. Si double fit, BW 6 clés ?\\
3\s \> Barrage\\
4\c/4\d/4\h \> Splinter avec 5 atouts\\
4\s \> Barrage\-
\end{bidtable}

\subsection{Après intervention}

Le contre remplace le Drury (en plus de garder sa signification habituelle). Le cue-bid montre 4 cartes.

Les enchères de rencontre restent d'application, le Splinter s'effectue uniquement dans la couleur d'intervention.

\section{Enchères de chelem}

L'enchère de 4NT est un Blackwood à 5 clés incluant le roi d'atout. L'atout est la couleur explicitement ou implicitement (via cue-bid) fittée ou à défaut la dernière couleur nommée naturellement

\begin{bidtable}
4NT\+\\
5\c \> 4 ou 1 clé(s)\\
5\d \> 3 ou 0 clé(s)\\
5\h \> 2 clés sans dame d'atout\\
5\s \> 2 clés avec dame d'atout\\
5NT \> nombre impair de clé et une chicane\\
6x \> nombre pair de clés et chicane x utile (ou chicane au-dessus si x = atout)\-
\end{bidtable}

Sur une réponse 5m, le palier immédiatement supérieur en dehors de l'atout est un relais pour la demande de la dame d'atout et des rois spécifiques.
Avec la dame d'atout, on nomme économiquement la couleur du premier roi ou 5SA déniant un roi.

Le retour à l'atout au palier le plus économique dénie la dame d'atout. Ensuite, le palier au-dessus demande de nommer les rois dans un ordre économique.

Une autre enchère non-ambigüe explore pour le grand chelem, demandant un complément.

Blackwood d'exclusion : Double saut anormal avec fit précisé ! 
Réponses : 1, 3/0, 2, 2+Q.

Une réponse de 5NT avec une zone initiale de 2 pts (par exemple l'ouverture de 2NT) propose de découvrir un fit 4-4 au palier de 6.

En cas de bicolore et sans fit précisé par manque de place, le Blackwood est à 6 clés :

\begin{itemize}
\item 5\h\ : pas de dame

\item 5\s\ : la dame la moins chère

\item 5NT : la plus chère

\item 6\c\ : les deux

\end{itemize}

Après le BW, 5NT est une demande de roi :

\begin{bidtable}
5NT\+\\
Retour \> à l'atout : pas de roi.\\
6X \> : Roi X ou les deux autres.\\
6NT \> : Tous les rois.\-
\end{bidtable}

Sur un 4NT quantitatif :

\begin{bidtable}
4NT\+\\
Passe \> mini\\
5\c \> 4 ou 1 As\\
5\d \> 3 ou 0 As\\
5\h \> 2 As et 5\c \\
5\s \> 2 As et 5\d \-
\end{bidtable}

D0PI/R0PI

Après intervention adverse sur le Blackwood :
Double 30 Passe 41, ensuite 2 et 2+Q

Après le contre d'une couleur du Blackwood :
Redouble 30 Passe 41, ensuite 2 et 2+Q

Si intervention adverse au-dessus de 5 de notre couleur :
DEP0/REP0 - Double Even, Pass Odd.

Intervention adverse par contre sur les cue-bids :

\begin{itemize}
\item Passe : sans contrôle

\item XX : contrôle du premier tour

\item Autre : contrôle du second tour (retour à l'atout = second tour sans autre cue).

\end{itemize}

Intervention adverse en couleur sur les cue-bids :

\begin{itemize}
\item X : punitif

\item Passe : sans contrôle

\item Autre : souvent court dans la couleur

\end{itemize}

Intervention adverse par contre sur un cue-bid dans une courte connue (Splinter par exemple) :

\begin{itemize}
\item Passe : pas de contrôle

\item XX : contrôle du premier tour

\item Autre : cue-bid + contrôle du second tour

\end{itemize}

\section{Enchères compétitives}

\section{Jeu de la carte}

\subsection{L'entame}

\subsubsection{Entame à SA}

Principe de la quatrième meilleure :

\begin{itemize}
\item Quatrième carte d'une couleur contenant un honneur (le 10 n'est pas considéré comme un honneur, sauf 109xx ou 10 cinquième. )

\item Tête de séquence d'une séquence d'au moins trois cartes dont un honneur

\item Séquence brisée, le plus gros honneur du début de séquence (A/R/D 109xx - le 10, jusqu'au 9 - D98x le 9.)

\item Top of nothing : \textbf{x}xx - la plus grosse; x \textbf{x} x x

\item Séquence = 3 cartes avec au moins 1 honneur (àpd T).

\item Si le partenaire ouvre d'1\c\ et que les adversaires jouent à NT, 4e meilleure même à \c .

\end{itemize}

Sur l'entame :

\begin{itemize}
\item Roi : Déblocage, Parité. Si le mort est court - appel par les petites

\item As : Appel avec des couleurs longues (le partenaire est court) et pas des honneurs

\item Dame : Attitude - on dit qu’on aime quand on a des honneurs car son entame provient souvent d’une longue

\item Valet : Parité - on confirmera par la suite si intérêt dans la couleur

\end{itemize}

\subsubsection{Entame à la couleur}

Tête de séquence de 2 cartes ou séquence brisée.

Parité :

\begin{itemize}
\item Dans 4 cartes, la 2e si pas d'honneur, la 3e si un honneur.

\item Dans 6 cartes, le 3e si pas d'honneur, la 5e si un honneur.

\end{itemize}

Jeu des honneurs :

\begin{itemize}
\item En 3e position dans la levée, on met l'honneur le moins élevé.

\item Si le partenaire entame le Roi - la Dame promet le Valet.

\end{itemize}

\subsubsection{Dans le cours du jeu}

\begin{itemize}
\item Switch petit prometteur - en fonction du mort.

\end{itemize}

\begin{itemize}
\item Quand le partenaire entame d'un honneur et que le mort est court - italien.

\end{itemize}

\begin{itemize}
\item Confirmation d'entame par les petites - si nécessaire.

\end{itemize}

\begin{itemize}
\item Première défausse Lavinthal.

\end{itemize}

\end{document}
