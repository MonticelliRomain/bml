\documentclass[a4paper]{article}
\usepackage[T1]{fontenc}
\usepackage[utf8]{inputenc}
\usepackage{newcent}
\usepackage{helvet}
\usepackage{graphicx}
\usepackage[pdftex, pdfborder={0 0 0}]{hyperref}
\frenchspacing

\include{bml}
\title{Système Monticelli's}
\author{Thomas et Romain Monticelli}
\begin{document}
\maketitle
\tableofcontents

\section{Introduction}

\section{Ouverture de 1m}

\subsection{Le Walsh}

À partir de 2 cartes. Le Walsh, basé sur le principe ``la majeure d'abord'', consistant à ne présenter les \d\ que dans les mains fortes ou sans majeures, est utilisé.
En cas de fit fort sans majeure, le fit mineur inversé est utilisé.

\begin{bidtable}
1\c---\\
1\d \> 3 possibilités :\\
\>5+\d\ et inv+\\
\>5+\d , 3-\h \s\ et 6-10H\\
\>BAL 5-7H et 3-\h \s \\
1\h \> \h /4+ (possibilité de \d\ plus longs que \h\ si <11h)\\
1\s \> \s /4+ (possibilité de \d\ plus longs que \s\ si <11h)\\
1NT \> 8-10 3-\h \s \\
2\c \> FMI : fit \c , 10+ (voir développements plus loin)\\
2\d \> 6+\d\ (et 2GH), 18H+\\
2\h \> 5+\s\ 4+\h , 5-9H, idem sur 1\d \\
2\s \> 5+\s\ 4+\h , 10-11H, idem sur 1\d \\
2NT	 \> 11-12 BAL (3\c\ NF, reste FM)\\
3\c	 \> 5\c , main irrégulière ~5-9 H\\
3X	 \> Barrage 7 cartes\\
3NT	 \> 12-14 BAL
\end{bidtable}

\subsection{La séquence 1\pdfc\ - 1\pdfd}

\begin{bidtable}
1\c-1\d;\\
1M \> 12-17 H, M/4 Main irrégulière (5\c )\\
1\h\+\\
1\s \> 4eme couleur forcing\\
\>(L’ouvreur réagit comme sur une enchère NAT mais ne peut pas dire 4\s ) \+\\
2\s \> 12-14\\
3\s \> 15+\+\\
3NT \> SO\\
4X \> cue fit \s \-\-\\
2\s \> 5\d 4\s\ FM\-\\
1NT \> 12-14 H, BAL\+\\
2\c \> checkback stayman avec des mains INV+\+\\
2X\+\\
3\c \> 5\d 5\c\ FM (fit différé)\\
3\d \> 6+\d\ FM\-\\
2\d \> Pas de M/4\\
2\h \> 4\h\ (avec ou sans 4\s )\\
2\s \> 4\s \-\\
3\c \> 5\d 4\c\ INV\\
3\d \> 6\d\ INV\-
\end{bidtable}

\subsection{La séquence 1\pdfc\ - 1M}

\begin{bidtable}
1\c-1M;\\
1NT\+\\
2\d \> 5\d 4M faible to play\-\\
2\c\+\\
2\d \> 3eme couleur forcing\-
\end{bidtable}

\subsection{La séquence 1m - 1M}

\begin{bidtable}
1m-1M;\\
2NT \> pas de fit\+\\
3\c \> réponses dans l'ordre, style Baron\+\\
3\d \> 5m\\
3M \> 3cM\\
3M' \> 4cM'\\
3NT \> aucun des cas précédents\-\\
3\d \> fit dans la m (FM ou TDC)\\
3M \> NAT NF\\
3M' \> BIC (55 ou 65) FM\\
3NT \> SO\\
4m/M' \> cue avec atout M\\
4\h \> (sur 1\s ) BIC NF\\
4M \> SO\\
4NT \> quantitatif\-\\
3NT \> fit main régulière\\
4(3)M’ \> splinter\\
4m’ \> splinter\\
4m \> 5+m 4+M\\
4M \> main extrêmement irrégulière avec peu de points d’honneurs
\end{bidtable}

\subsection{La séquence 1m - 2\pdfh}

\begin{bidtable}
1m-2\h \> 5+\s\ 4+\h\ 5-9\+\\
2\s \> Nat NF\\
2NT \> Relay\+\\
3\c \> Mini\+\\
3\d \> Relay\+\\
3\h \> 54\\
3\s \> 55\\
3NT \> 64\-\-\\
3\d \> Max 54\\
3\h \> Max 55\\
3\s \> Max 64\-\\
3\c \> Nat NF\\
3\d \> Nat NF\\
3M \> Invit\\
3NT \> Nat\-
\end{bidtable}

\subsection{La séquence 1m - 2\pdfs}

Cfr 1m-2\h 

\subsection{La séquence 1\pdfd\ - 2\pdfc}

\begin{bidtable}
1\d-2\c;\\
2\d \> irrégulier 11-15, BAL 12-13\\
2M \> inversée NAT\\
2NT \> BAL 14 ou 18-19\+\\
3\c \> relais\+\\
3\d \> 18-19, le reste NAT 14\-\-\\
3\c \> 16+ PJS Fit 4ème ou 18-19 et 3 cartes\\
3\d \> 6+\d , 16+ problème dans une ou deux M\\
3M \> 65 sans inversée\\
3NT \> 6+\d , 16+ avec les arrêts M
\end{bidtable}

\subsection{Le fit mineur inversé}

\begin{bidtable}
1\c-2\c \> --\\
2NT \> 12-14- BAL\+\\
3\c \> SO\\
3\d \> arrêt \d \+\\
3\s \> arrêt \s \+\\
4\c \> SO, pas d'arrêt \h \-\-\\
Autre \> : arrêt\\
Sauts \> : courtes\+\\
4NT \> : to play (des points dans les courtes)\-\-\\
2\d \> 14+, arrêt \d , arrêts \h /\s\ possibles\+\\
2\h \> arrêt \h\ et pas mini\\
2\s \> arrêt \s , pas d'arrêt \h\ et pas mini\\
2NT \> arrêts des autres couleurs mais mini\\
3\c \> pas d'arrêts des autres couleurs et mini\\
3\d \> singleton (ou chicane) \d \-\\
2\h \> 14+, arrêt \h , pas d'arrêt \d , arrêt \s\ possible\\
2\s \> 14+, arrêt \s , pas d'arrêts \h /\d \\
3\c \> 12-14- irrégulier\+\\
Sauts \> : courtes\+\\
4NT \> : to play (des points dans les courtes)\-\-\\
3\d \> 14+, court \d , \c\ mieux que Vxxx\\
3\h \> 14+, court \h , \c\ mieux que Vxxx\\
3\s \> 14+, court \s , \c\ mieux que Vxxx\\
3NT \> 18-19 Bal\\
4\d \> 15-18, chicane \d , beaux \c \\
4\h \> 15-18, chicane \h , beaux \c \\
4\s \> 15-18, chicane \s , beaux \c 
\end{bidtable}

\begin{bidtable}
1\d-2\d \> --\\
2NT \> 12-14- BAL\+\\
3\c \> arrêt \c \+\\
3\s \> arrêt \s \+\\
4\d \> SO, pas d'arrêt \h \-\-\\
3\d \> SO\\
Autre \> : arrêt\\
Sauts \> : courtes\+\\
4NT \> : to play (des points dans les courtes)\-\-\\
2\h \> 14+, arrêt \h , arrêts \s /\c\ possibles\+\\
2\s \> arrêt \s\ et pas mini\\
2NT \> arrêts des autres couleurs mais mini\\
3\c \> arrêt \c , pas d'arrêt \s\ et pas mini\\
3\d \> pas d'arrêts des autres couleurs et mini\\
3\h \> singleton (ou chicane) \h \-\\
2\s \> 14+, arrêt \s , pas d'arrêt \h , arrêt \c\ possible\\
3\c \> 14+, arrêt \c , pas d'arrêts \h /\s \\
3\d \> 12-14- irrégulier\+\\
Sauts \> : courtes\+\\
4NT \> : to play (des points dans les courtes)\-\-\\
3\h \> 14+, court \h , \d\ mieux que Vxxx\\
3\s \> 14+, court \s , \d\ mieux que Vxxx\\
3NT \> 18-19 Bal\\
4\c \> 14+, court \c , \d\ mieux que Vxxx\\
4\h \> 15-18, chicane \h , beaux \d \\
4\s \> 15-18, chicane \s , beaux \d \\
5\c \> 15-18, chicane \c , beaux \d 
\end{bidtable}

\subsubsection{Soutien au niveau de 3}

\begin{bidtable}
1m\+\\
3m \> NF, on peut jouer ce contrat\+\\
4m \> prolongation de barrage\\
3x \> demande d'arrêt dans cette couleur\\
4x \> courte\-\-
\end{bidtable}

\subsubsection{Après interventions}

Le FMI ne s'applique plus (passer par le surcontre)

\begin{bidtable}
1\d-X\+\\
2\d \> 6-9\\
2NT \> 10-11 fitté (Truscott)\\
3\d \> barrage 6+\-
\end{bidtable}

\begin{bidtable}
1\c-1\s\+\\
2\c \> fit 6-10\\
2\s \> 11+, pas 4\h , souvent fitté \c \\
2NT \> naturel 10-12 avec arrêt\\
3\c \> barrage\\
4\c \> barrage\-
\end{bidtable}

\subsection{Enchère du répondant après une inversée}

Pour s'arrêter à une partielle après une inversée de l'ouvreur, le répondant dispose de deux enchères :

\begin{itemize}
\item La répétition de sa couleur au niveau de 2 montrant 6 cartes (parfois 5, en cas de distribution 5521 ou belle couleur).
  La plupart du temps, l'ouvreur passera.

\end{itemize}

Cependant, il dispose de trois enchères non forcing :
1. 2SA avec l'arrêt dans la dernière couleur (et probablement un complément dans la couleur du répondant).
2. Le soutien au palier de 3, invitationnel.
3. La répétition de sa couleur d'ouverture.
Toutes les autres enchères sont FM.

\begin{itemize}
\item L'enchère conventionnelle de 2NT (Lebensohl, transfert pour 3\c ) : main faible ou une main avec 5 cartes en majeure si celle-ci est répétable au niveau de 2.
L'ouvreur rectifiera le transfert sauf s'il est maximum. Il se décrira alors le plus naturellement possible, rendant la séquence FM.
Après la rectification du transfert, le répondant :

\end{itemize}

\begin{enumerate}
\item Passe s'il souhaite jouer 3\c\ 

\item Corrige à 3\d , 3\h\ ou 3\s\ SO (sauf répétition de la couleur de réponse)

\item Répète, le cas échéant, sa couleur au niveau de 3 avec 5 cartes pour laisser l'ouvreur choisir la manche adéquate (voir exemple)

\item Enchérit la manche avec des mains ayant eu de très ténus espoirs de chelem

\end{enumerate}

\dealdiagramenw
{}
{\vhand{AQ642}{T52}{K86}{J5}}
{}
{N}

\begin{bidtable}
1\c-1\s\\
2\h-2NT\\
3\c-3\s
\end{bidtable}

\subsection{Double-Deux}

Après toute séquence 1X - 1Y - 1Z, 2\c , 2\d\ et 2NT sont artificiels. 2\c\ montre
une main faible à carreau ou des mains invites ou une main de manche avec 5 cartes
dans sa majeure. 2\d\ est un relais FM et demande à l'ouvreur de se décrire.
2NT montre des mains faibles avec des trèfles ou des mains chelemisantes avec plus de
trèfles que de cartes dans la première couleur.
Les enchères au niveau de 3 sont chelemisantes et naturelles, et un bicolore cher
au niveau de 2 est FM, a priori avec un 6-5

\subsubsection{L'ouvreur dit 1NT}

\begin{bidtable}
1\c-1\h-1NT\+\\
2\c \> Relais pour 2\d , invit ou mieux\\
2\d \> Relais FM\\
2\h \> Pour jouer\\
2\s \> TDC 6\h\ 5\s \\
2NT \> Faible avec des trèfles ou FM fitté trèfle avec exactement 4\h .\+\\
Force \> le partenaire à dire 3\c \-\\
3\c \> TDC fitté trèfles avec 5 coeurs\\
3\d \> TDC 5-5. 3\h\ montre le fit, 3NT est négatif, le reste est contrôle fitté carreaux\\
3\h \> TDC 6 mauvaises cartes. Avec 6 belles cartes on aurait dit 2\h\ sur 1\c \\
3\s \> 4 piques 5 coeurs, pour laisser le choix entre 3SA, 4C et 4P\\
3NT \> Pour les jouer\\
4\h \> Pour les jouer\\
5T \> Pour les jouer\\
4NT \> Invit\\
5NT \> Invit pour 7\-
\end{bidtable}

Après un 2\c\ - 2\d , les réponses du répondant sont naturelles:

\begin{bidtable}
1\c-1\h-1NT-2\c-2\d\+\\
2\h \> Invit avec 5\h . Une nouvelle couleur de la part de l'ouvreur montre une force et\\
\>est une enchère d'essais. 2NT est enchère d'essais généralisée fittée.\\
\>3NT et 4\h\ sont pour jouer.\\
2\s \> Invit avec 6\h\ 5\s \\
2NT \> Invit sans 5\h \\
3\c \> Invit fitté trèfles\\
3\d \> Invit 5-5\\
3\h \> Invit 6 cartes\\
3NT \> 5 coeurs, pour laisser le choix entre 3NT et 4\h \-
\end{bidtable}

\begin{bidtable}
1\d-1\s-1NT-2\c-2\d\+\\
Pass \> Pour les jouer\\
2\h \> 5\s\ 4\h\ invit\\
2\s \> Invit avec 5 cartes\\
2NT \> Invit sans 5 cartes\\
3\c \> 5-5 invit\\
3\d \> Invit avec 4+ carreaux\\
3\h \> 5-5 invit\\
3\s \> Invit avec 6 cartes\\
3NT \> 5\s , pour laisser le choix entre 3NT et 4\s \-
\end{bidtable}

Après un 2\d , les réponses de l'ouvreur sont naturelles, avec l'enchère impossible
qui montre une main avec une courte qu'on ne peut pas annoncer autrement:

\begin{bidtable}
1\c-1\h-1NT-2\d\+\\
2\h \> Trois cartes\\
2\s \> 3145\\
2NT \> 3244\\
3\c \> 3235\\
3\d \> 2245\-
\end{bidtable}

\begin{bidtable}
1\c-1\s-1NT-2\d\+\\
2\h \> Quatre cartes\\
2\s \> Trois cartes\\
2NT \> 2344\\
3\c \> 2335\\
3\d \> 2245\\
3\h \> 1345\-
\end{bidtable}

Après un 2NT, l'ouvreur doit dire 3\c . Le répondant passe avec une main faible
et reparle avec une main chelemisante. Les réponses sont naturelles et les
continuations aussi:

\begin{bidtable}
1\d-1\s-1NT-2NT-3\c\+\\
Pass \> Pour les jouer\\
3\d \> Quelque chose comme 4135. L'ouvreur dit 3NT si pas intéressé, 3\s\ si\\
\>trois beaux piques, 4\c\ pour chelemisant à trèfles, ou 3\h\ ou 4\d\ pour\\
\>chelemisant à carreaux.\\
3\h \> Quelque chose comme 4315. De même, l'ouvreur donne sa couleur de préférence\\
\>ou 3NT si pas intéressé\\
3\s \> 6\c\ 5\s \\
3NT \> 4225 non forcing\\
3\c \> 6 trèfles, demande à l'ouvreur de nommer les contrôles.\\
4NT \> 4225 invit\\
5NT \> 4225 invit pour 7\-
\end{bidtable}

\subsubsection{L'ouvreur ne dit pas 1NT}

L'ouvreur peut encore avoir 18-19 points, donc il n'est pas obligé de rectifier
les transferts de 2\c\ ou de 2NT s'il a peur que le partenaire passe.

\begin{bidtable}
1\c-1\h-1\s\+\\
1NT \> 6-10\\
2\c \> Invit ou 12-14 (15) avec 5 coeurs. Ici l'ouvreur n'a pas peur que le partenaire\\
\>passe la réponse de 2\d . En effet, il a promis plus de coeurs que de carreaux.\\
2\d \> Relais FM\\
2\h \> 6-9 (10) avec 6 cartes\\
2\s \> 6-9 (10) avec 4 cartes\\
2NT \> 5+\c , plus de trèfles que de coeurs, faible ou TDC\\
3\c \> TDC 5\h\ 5\c\ (Il faut 5 trèfles car l'ouvreur n'en a promis que 3)\\
3\d \> TDC 5\h\ 5\d \\
3\s \> TDC 4 cartes\\
3NT \> 12-14 (15) sans 5 coeurs\-
\end{bidtable}

\begin{bidtable}
1\c-1\h-1\s-2\c-2\d\+\\
2\h \> Invit 5 cartes\\
2\s \> Invit 4 cartes\\
2NT \> Invit sans 5\h\ ni 4\s \\
3\c \> Invit avec 5 trèfles et 4 coeurs\\
3\d \> Invit 5-5\\
3\h \> Invit 6 cartes\\
3\s \> Invit 4 piques et 5 coeurs\\
3NT \> 12-14 (15) avec 5\h \-
\end{bidtable}

\begin{bidtable}
1\c-1\h-1\s-2\d\+\\
2\h \> 3 cartes\\
2\s \> 6\c\ 5\s \\
2NT \> 4\s\ 5\c\ 15-17 sans 3 coeurs\\
3\c \> 4\s\ 5\c\ 12-14 sans 3 coeurs\\
3\d \> 4045\\
3\h \> TDC 3 cartes\\
3\s \> TDC 6\c\ 5\s \\
3NT \> 4234 12-14\-
\end{bidtable}

\begin{bidtable}
1\c-1\h-1\s-2NT
\end{bidtable}

Ici l'ouvreur peut ne pas rectifier le transfert s'il est 18-19. Surtout que si c'est le cas, il est irrégulier
et il y a donc au moins un fit 10e à trèfle. La 4e couleur demande l'arrêt.

\begin{bidtable}
1\d-1\h-1\s-2\c
\end{bidtable}

De nouveau si l'ouvreur est 18-19 il peut ne pas rectifier le transfert pour ne pas jouer 2\d\ lorsque 3NT gagne.

\subsection{Le Gazzili}

\begin{bidtable}
1\h-1\s--\\
1NT \> BAL 12-15 (2\c\ Roudi, Reste Nat)\\
2\c \> Gazilli : Soit NAT, soit Fort\+\\
2\d \> 8+PH, FM si ouvreur fort. Tout le reste est plus faible, sauf 3M.\+\\
2\h \> 5\h 4\c\ 12-16 NF. Ensuite, Passe ou NAT Inv+\\
2\s \> 3 cartes à \s \+\\
2NT \> Relais\+\\
3m \> 5\h 4m3\s \\
3\h \> 6\h 3\s\ 6 levées\\
3\s \> BAL 15-17 avec 3\s \\
3NT \> BAL 18-19 avec 3\s \-\-\\
2NT \> BAL (54m22 / 5332) 15-16 sans 3\s \\
3m \> 5\h 4m 17+ (ou 5\h 5m 5 perdantes)\\
3\h \> 6\h\ sans 3\s\ 6 levées\\
3\s \> 5+\h 4\s\ 15-17\+\\
3NT \> demande de courte\\
4x \> cue\-\\
3NT \> BAL (54m22 / 5332) 16+-17 sans 3\s \-\\
2\h \> Préférence\\
2\s \> 5\s\ court \h\ ou 6+\s \\
2NT \> 5\d 4\s\ court \h\ (ensuite 3m préférence NF).\\
3\c \> Préférence\\
3\d \> 6+\d \\
3\h \> Fit différé\\
3\s \> INV\-\\
2x \> NAT (Sur la répétition à 2\h , 3\h\ est Inv)\\
2NT \> 6\h /4x ou 6\h /3\s \+\\
3\c \> Relais\+\\
3\d \> 6\h 4\d \\
3\h \> 6\h 3\s \\
3\s \> 6\h 4\s \\
3NT \> 6\h 4\c \-\-\\
3m \> 55 4 perdantes\\
3\h \> 6+\h\ sans 3\s , 7 levées\\
3\s \> 5\h 4\s\ 18-20\+\\
3NT \> demande de courte\\
4x \> cue\-\\
3NT \> BAL 18-19 sans 3\s \\
4m \> Fit 20+PJS, Splinter\\
4\h \> Sign-off\\
4\s \> Fit distributionnel
\end{bidtable}

\begin{bidtable}
1\h-1NT--\\
2\c \> Gazilli : Nat ou 17+\+\\
2\d \> 8+ PH, FM si ouvreur fort. Reste plus faible\+\\
2\h \> 5\h 4\c\ 12-16\\
2\s \> 5\h 4m irrégulier 17+\+\\
2NT \> Relais\+\\
3\c \> 5\h 4\c \\
3\d \> 5\h 4\d \\
3\h \> 5\h 4\d 4\c \-\-\\
2NT \> Bal (54m22 / 5332) 15-16\\
3m \> 55 5 perdantes\\
3\h \> 6+\h\ 6 levées\\
3\s \> 6\h 5\s\ 5 perdantes\\
3NT \> Bal (54m22 / 5332) 16+-17\-\\
2\h \> Préférence\\
2\s \> 5\c\ 4\d \\
2NT \> 5\d , 4/5\c \\
3\c \> Préférence\\
3\d \> 6+\d \-\\
2\d\h \> Nat 12-16\\
2\s \> Inversée (2NT modérateur)\\
2NT \> 6\h 4m 17+\+\\
3\c \> Relais\+\\
3\d \> 6\h 4\d \\
3\h \> 6\h 4\c \-\-\\
3m \> 55 4 perdantes\\
3\h \> 6+\h , 7 levées\\
3\s \> 65 4 perdantes\\
3NT \> Bal 18-19\\
4m \> 65 sans TDC\\
4\h \> Sign-off
\end{bidtable}

\begin{bidtable}
1\s-1NT--
\end{bidtable}

\begin{bidtable}
2\c \> Gazilli : Naturel ou Fort\+\\
2\d \> 8+ PH, FM si ouvreur fort. Reste plus faible\+\\
2\h \> 3+\h\ (4\h\ 17+, inversée avec 3\h , 6\s 3\h\ 6 levées ou Bal 15+ avec 3\h )\+\\
2\s \> Relais\+\\
2NT \> Bal 15-17 avec 3\h \\
3\c\d \> 5\s 4m3\h \\
3\h \> 5\s 4\h \\
3\s \> 6\s 3\h \\
3NT \> Bal 18-19 avec 3\h \-\-\\
2\s \> 5\s 4\c\ 12-14 NF\\
2NT \> Bal 15-16 sans 3\h \\
3m \> 54 17+ sans 3\h\ ou 55 5 perdantes\\
3\h \> 55 5 perdantes\\
3\s \> 6+\s\ sans 3\h , 6 levées\\
3NT \> Bal 16+-17 sans 3\h \-\\
2\h \> 5\h\ court \s\ ou 6+\h \\
2\s \> Préférence\\
2NT \> 5\d 4\h\ court \s\ (ensuite 3m préférence NF)\\
3\c \> Préférence\\
3\d \> 6+\d \-\\
2\d\h\s \> Nat\\
2NT \> 6\s 4x / 6\s 3\h\ 17+\+\\
3\c \> Relais\+\\
3\d \> 6\s 4\d \\
3\h \> 6\s 4\h \\
3\s \> 6\s 3\h \\
3NT \> 6\s 4\c \-\-\\
3\c\d\h \> 55 4 perdantes\\
3\s \> 6+\s\ sans 3\h , 17+\\
3NT \> Bal 18-19 sans 3\h \\
4x \> 65 sans TDC\\
4\s \> Sign-off
\end{bidtable}

Après une réponse faible sur un Gazilli, 3x est 5M4x 17+ ou 5M5x 5 perdantes (sauf 3m sur la réponse de 2NT et sur la réponse de 2\s\ après 1\h -1NT), 3M Inv, 2NT Inv avec réponse naturelle et non forcing, 4M clôture.

Sur le soutien simple à 2M, l'ouvreur utilise les enchères d'essai dans les courtes (2NT court \s\ sur 2\h ) et généralisé à 2\s\ (sur 2\h\ : nommer des forces, 2NT force \s ) ou 2NT (sur 2\s , nommer des forces), saut court et TDC, 2NT puis saut long et TDC, courte puis répétition chicane et TDC.

Après un contre de 2T, 2\d\ montre 8+ et l'arrêt. Sans arrêt, le répondant passe ou montre une préférence faible.

\section{Le 2 sur 1 FM}

Notre étude est limitée à l’ouverture en majeure. 
Note : 
L’ancien système reste évidemment après une ouverture en 3ième puisque le 2/1 FM promet 
l’ouverture et que le n°1 a passé

\subsection{Introduction}

Les enchères non forcing en standard francais

\begin{bidtable}
1\s-2\c\\
2\d-2\s
\end{bidtable}

\begin{bidtable}
1\s-2\c\\
2\h-2NT
\end{bidtable}

\begin{bidtable}
1\s-2\c\\
2\d-3\c
\end{bidtable}

Une constante : le répondant a enchéri exactement 11H(L)

Les changements:

\begin{enumerate}
\item 1SA va jusque 11H

\item Le changement de couleur 2 sur 1 promet la valeur d’une ouverture

\item Le changement de couleur à saut au niveau de 3 est limite

\end{enumerate}

Remarques:

\begin{enumerate}
\item Si on considère que l’on aurait ouvert une main ne comportant que 11H, il faut 
l’assimiler à un 2/1 FM

\item A l’inverse, l’évaluation de la main est « rétroactive ». On aurait ouvert certaines mains 
mais lorsque le partenaire ouvre, des chicanes peuvent être dévaluées et donc il faut les 
considérer comme une non-ouverture et dire 1SA.

\end{enumerate}

\subsection{1SA va jusque 11}

La réponse de 1SA est agrandie à la zone 6-11H. 
Ses conditions d’emploi sont inchangées. 
L’enchère est définie comme « semi-forcing ».

\subsubsection{Attitude de l'ouvreur}

Une seule modification : il ne passe pas avec 14H régulier et nomme sa 
première mineure de 3 cartes (voir corrolaires gazzili ?)

\subsubsection{Attitude du répondant}

Quand le répondant a répondu 1SA dans la zone 10/11H, il doit ensuite 
le signaler

\begin{enumerate}
\item Par une enchère impossible

\item Par l’enchère de 2SA

\end{enumerate}

\subsection{Le saut au niveau de 3}

Nommer sa couleur avec saut avec une belle couleur 6ième et 11H(L). 
L’expression de ce saut est non forcing. Il dénie l’autre majeure 4ième
Rem : si on a passé d’entrée, ces enchères sont des rencontres.

Conséquence : répétition de la couleur du répondant
Minimum 14H avec 2 gros honneurs dans une couleur au moins 6ième

\subsection{Les nouveaux soutiens directs}

\begin{bidtable}
1\h--\\
2\h \> 8-10DH (dans la zone 5-7DH répondre 1 \s\ ou 1SA)\\
3\h \> 8-10DH - 4 atouts (évitez 8H – 4333)\\
2NT \> 11-14DH – 3 ou 4 atouts, en-dessous du 2/1 FM\\
3\s \> Splinter chicane : 7-10H; 5 atouts - 2 cartes clés. Ensuite 3NT demande de courte : \c , \d , \s \\
3NT/4\c\d \> Splinter singleton : 7-10H - 2 cartes clés.
\end{bidtable}

\begin{bidtable}
1\s--\\
3NT \> Splinter chicane : 7-10H; 5 atouts - 2 cartes clés. Ensuite 4\c\ demande de courte : \d , \h , \c \\
4\c/4\d/4\h \> Splinter singleton : 7-10H - 2 cartes clés.
\end{bidtable}

\subsubsection{La séquence 1M – 2SA}

Enchère charnière où il est possible de s’arrêter avant la manche mais également de la déclarer en 
face d’un ouvreur minimum.

\paragraph{Attitude de l’ouvreur après la séquence 1M-2SA}

\begin{enumerate}
\item Il freine à 3M avec 12/13H réguliers ou 11/13H avec honneur singleton.
Rem : le répondant avec 13/14DH peut nommer l’autre majeure 4ième pour retrouver un fit 4-4
 ex : 1 \h\ - 2SA – 3 \h\ - 3 \s\ ou 1 \s\ - 2SA – 3 \s\ - 4 \h\ 

\item Il saute à 4M dans la zone au-dessus de 3M, également avec 6 cartes en M ou bicolore 5-5 sans 
espoir de chelem

\item 3SA avec 5332 de 14 à 18H (attention à un petit doubleton) 

\item Nomme l’autre majeure de 4 cartes, toute zone (sauf 11/12H et 5422)

\item Nomme une mineure de 4 cartes à partir de 17H, ambitions de chelem (peut-être dans la 
mineure. Celle-ci doit donc être de bonne qualité.
Cas rare : 3 \c\ avec 18/19H réguliers (donc même dans 2 cartes !)

\end{enumerate}

\subsubsection{Le 2 sur 1 FM}

\paragraph{Le fit différé au niveau de 2}

\begin{enumerate}
\item Exprime un soutien d’au moins 3 cartes dans la majeure 
d’ouverture dans une zone 16DH+ (15+
DH).

\item Le plus souvent avec une distribution régulière ou un 
problème de qualité ou de longueur, soit dans la mineure soit 
à l’atout.

\item 2 \c\ peut se faire avec 2 cartes.
2 \d\ avec au moins 4 cartes, plutôt cinq.
2 \h\ toujours 5 cartes mais à éviter avec un soutien \s . (si 
l’ouvreur soutient les \h , on jouera à \h )

\end{enumerate}

Demande des renseignements essentiellement distributionnels sur 
la main de l’ouvreur.

\paragraph{Séquences après 1M – 2 \pdfc}

\begin{bidtable}
1\s-2\c\\
2\d-2\s--\\
2NT \> 14H+ avec les mains qui ne correspondent pas aux autres enchères\+\\
3\c \> Relais\+\\
3\d \> singleton le moins cher\\
3\h \> singleton le plus cher\\
3\s \> 17H+, 5422\\
3NT \> 14-16H, 5422\-\-\\
3\c \> ctrl d’honneur, As ou le Roi à \c , singleton \h \\
3\d \> 5 \s\ et 5 \d\ 12-14H, avec 15H+ dire 3 \d\ sur 2 \c \\
3\h \> ctrl d’honneur, As ou le Roi à \h , singleton \c \\
3\s \> 6 \s\ et 4 \d \\
3NT \> 5242 – 11/13H\\
4\c \> Splinter \c , sans ctrl \h\ - 11/13H\\
4\h \> Splinter \h , sans ctrl \c\ - 11/13H
\end{bidtable}

\begin{bidtable}
1\s-2\c\\
2\h-2\s--\\
2NT \> 14H+ avec les mains qui ne correspondent pas aux autres enchères\+\\
3\c \> Relais\+\\
3\d \> singleton le moins cher\\
3\h \> singleton le plus cher\\
3\s \> 17H+, 5422\\
3NT \> 14-16H, 5422\-\-\\
3\c \> ctrl d’honneur, As ou le Roi à \c , singleton \d \\
3\d \> ctrl d’honneur, As ou le Roi à \d , singleton \c \\
3\h \> 5 \s\ et 5 \h \\
3\s \> 6 \s\ et 4 \h \\
3NT \> 5422 – 11/13H\\
4\c \> Splinter \c , sans ctrl \d\ - 11/13H\\
4\h \> Splinter \d , sans ctrl \c\ - 11/13H
\end{bidtable}

\begin{bidtable}
1\h-2\c\\
2\d-2\h--\\
2NT \> 14H+ avec les mains qui ne correspondent pas aux autres enchères\+\\
3\c \> Relais\+\\
3\d \> singleton \c \\
3\h \> singleton \s \\
3\s \> 17H+, 5422\\
3NT \> 14-16H, 5422\-\-\\
2\s \> ctrl d’honneur, As ou le Roi à \s , singleton \c \\
3\c \> ctrl d’honneur, As ou le Roi à \c , singleton \s \\
3\d \> 5 \h\ et 5 \d \\
3\h \> 6 \h\ et 4 \d \\
3\s \> Splinter \s , sans ctrl \c\ - 11/13H\\
3NT \> 2542 – 11/13H\\
4\c \> Splinter \c , sans ctrl \s\ - 11/13H
\end{bidtable}

\begin{bidtable}
1\s-2\c\\
2x--\\
2NT \> Une certitude, le répondant n’a pas 3 cartes à \s .\+\\
L’enchère \> exprime une distribution régulière ou semi-régulière (voir 2 séquences plus bas)\-
\end{bidtable}

\begin{bidtable}
1\s-2\c\\
2\d--\\
2NT \> Sans tenue \h , il utilise la 4ième couleur forcing à 2 \h .\+\\
3\c \> 5143\\
3\d \> 55\\
3\h \> 5341\\
3\s \> 64\\
3NT \> 5242 limité à 16H\\
4NT \> 5242 17/18H\\
5NT \> 5242 19/20H\-
\end{bidtable}

\begin{bidtable}
1\s-2\c\\
2\h--\\
2NT \> Pas nécessairement l’arrêt \d \+\\
3\c \> 5413\\
3\d \> 5431\\
3\h \> 55\\
3\s \> 64\\
3NT \> 5422 limité à 16H\\
4NT \> 5422 17/18H\\
5NT \> 5422 19/20H\-
\end{bidtable}

\begin{bidtable}
1\s-2\c--\\
2\s \> L’enchère de 2 \s\ ne peut exprimer que 2 distributions : 5332 ou 6 cartes à \s .\\
\>Avec 6 très beaux \s , à partir de 14H, l’ouvreur saute à 3 \s .\\
\>Si il est 5332, il est limité a 14H.\+\\
2NT \> Relais avec au moins 2 cartes à \s .\+\\
L’enchère \> exprime une distribution régulière ou semi-régulière\\
3\c \> 6 \s\ et singleton \c \\
3\d \> 6 \s\ et singleton \d \\
3\h \> 6 \s\ et singleton \h \\
3\s \> 6322\\
3NT \> 5332\-\\
3\c \> Unicolore, ambition de chelem\\
3\d\h \> Tendance naturelle, bicolore ou sans tenue dans la 4ième, singleton \s\ très fréquent.\\
3\s \> Soutien forcing, beaux \c\ et au moins 2 honneurs à \s \\
3NT \> 12-15H, singleton \s , les 2 autres couleurs sont tenues.\\
4\c \> 5 \c\ 4 \s\ 13-15DH et singleton moins cher\\
4\d \> 5 \c\ 4 \s\ 13-15DH et singleton plus cher\\
4\s \> 5 \c\ 4 \s\ 13-15DH, honneurs concentrés régulier\-\\
2NT \> 17-20 5332 ou 2524 (15-16 on ouvre d'1NT) - Rappel 3T limité à 16.\+\\
3\c \> Naturel\\
3\h \> Fit \h \\
3NT \> Minimum. 4\c\ de l'ouvreur 2524, 4NT 19-20H\-
\end{bidtable}

\begin{bidtable}
1\s-2\c\\
2\d--\\
3\h \> 6-5 forcing\\
4\h \> 6-5 non forcing
\end{bidtable}

\begin{bidtable}
1\s-2\c--\\
3\c \> L’enchère de 3 \c\ est obligatoire avec 4 cartes à \c .\+\\
3\s \> L’enchère de 3 \s\ impose l’atout \s\ et montre une plus belle main que 4 \s .\\
\>L’ouvreur précise sa distribution.\+\\
3NT \> 5431, pas de ctrl du résidu.\+\\
4\c \> relais\+\\
4\d \> singleton moins cher\\
4\h \> singleton plus cher\-\-\\
4\c \> 5422, 14H+\\
4\d \> ctrl d’honneur à \d , courte à \h \\
4\h \> ctrl d’honneur à \h , courte à \d \\
4\s \> 5422, 11-13H\-\-
\end{bidtable}

Rem:

\begin{bidtable}
1\s-2\c\\
3NT \> 6 beaux \s\ et 4 beaux \c 
\end{bidtable}

\paragraph{Séquences après 1 \pdfs\ – 2 \pdfh}

\begin{bidtable}
1\s-2\h\\
2\s-2NT--\\
3\c \> 5 \s\ 4+ \c \\
3\d \> 5 \s\ 4+ \d \\
3\h \> 6 \s\ et singleton \h \\
3\s \> 6 \s\ régulier\\
3NT \> 5233 exactement\\
4\c \> 6 \s\ et singleton \c \\
4\d \> 6 \s\ et singleton \d 
\end{bidtable}

\begin{bidtable}
1\s-2\h--\\
3\h \> 11H+, 3 ou 4 cartes à \h , ne garantit pas un Honneur à \h , fort ou faible.\\
4\h \> 5422, 11-13H, plutôt des points perdus dans les mineures (très faible)\\
4\c\d \> Splinter atténué, environ 3 cartes clés\\
3NT \> Splinter fort, 4+ cartes clés\+\\
4\c \> relais\+\\
4\d \> singleton - cher (\c )\\
4\h \> singleton + cher (\d )\-\-\\
2NT \> 18-20H\+\\
()\+\\
4\c\d \> 5422 ctrl \c \\
4\d \> 5422 ctrl \d \\
4\h \> 5332\-\-
\end{bidtable}

\paragraph{Séquences dérivées après 1\pdfh\ d’ouverture}

\begin{bidtable}
1\h-2\c--\+\\
2\s \> S’exprime dans la zone 11-20H. Par conséquent, 1 \h\ - 2 \c\ - 2 \h\ dénie 4 cartes à \s .\+\\
2NT \> main régulière avec au moins 2 cartes à \h\ (peut cacher un soutien \h )\+\\
3\c \> 5413\\
3\d \> 5431\\
3\h \> 64\\
3\s \> 5422 15H+ : seule enchère conventionnelle !\\
3NT \> 5422 12-14H\-\\
3\s \> 14h+\+\\
3NT \> OUI mais : pas minimum avec un problème pour nommer un ctrl.\\
4\c\d\h \> Ctrl\\
4\s \> minimum\-\\
4\s \> 12-13H+\-\\
2\h\+\\
2\s \> Troisième couleur affirmative, jeu irrégulier (sinon 2SA). L’atout \s\ n’est plus recherché.\+\\
2NT \> promet une tenue à \d .\\
3\h \> 6 cartes.\\
3\c \> 3 cartes ou un honneur second sans tenue à \d \\
3\d \> enchère coopérative : je n’ai pas l’arrêt mais 3 petites cartes (ou l’As pour jouer de la bonne main\-\\
2NT \> main régulière avec au moins 2 cartes à \h\ (relais qui ne promet pas d'arrêt)\+\\
3\c \> 6 \h\ et singleton \c \\
3\d \> 6 \h\ et singleton \d \\
3\h \> 6322\\
3NT \> 5332\-\\
3\h \> Une perdante maximum en face d'un singleton\+\\
3\s \> vers le chelem à \h . L'ouvreur dit 3N avec ctrl \s\ et le reste en ctrl.\\
3NT \> naturel : pas de fit \h\ et mini.\\
4\c \> naturel : très breaux \c \\
4\d \> ctrl \d , court \s\ et ctrl \c \\
4\h \> frein\\
4\s \> à discuter : soit 6/5 naturel soit BW exclusion à \h \-\-\\
2NT \> 17-20 5332 ou 2524 (15-16 on ouvre d'1NT) - Rappel 3\c\ limité à 16.\+\\
3\c \> Naturel\\
3\h \> Fit \h \\
3NT \> Minimum. 4\c\ de l'ouvreur 2524, 4NT 19-20H\-\\
3NT \> 17+ 1534 ou 3514\+\\
4\c \> demande singleton pour jouer à \c\ (court bas-court haut)\\
4\d \> demande singleton pour jouer à \h\ (court bas-court haut)\-\-
\end{bidtable}

\paragraph{Développements après 1\pdfd-2\pdfc\ FM}

\begin{bidtable}
1\d-2\c--\\
2\d \> 6\d\ ou 5 beaux \d\ et un petit doubleton\+\\
2\h \> (force à \h )\+\\
2\s \> 5332 (6322) : pas de tenue \s\ et 2\c \\
2NT \> 5332 (6322) : tenue \s \\
3\c \> 5332 (6322) : pas de tenue \s\ et 3\c \\
3\d \> 6\d , singleton \c\ sans tenue \s \\
3\h \> 6\d , singleton \h\ avec ou sans tenue \s\ (3\s\ demande)\\
3NT \> 6\d , singleton \c\ et tenue \s \-\\
2\s \> (force à \s )\+\\
2NT \> 5332 (6322) : tenue \h \\
3\c \> 5332 (6322) : pas de tenue \h\ et 3\c \\
3\h \> 5332 (6322) : pas de tenue \h\ et 2\c \\
3\d \> 6\d , singleton \c\ sans tenue \h \\
3\s \> 6\d , singleton \s\ avec ou sans tenue \h \\
3NT \> 6\d , singleton \c\ et tenue \h \-\-\\
2\h \> 5\d\ 4\h\ 11-20H\\
2\s \> 5\d\ 4\s\ 11-20H\\
2NT \> 12-14 ou 18-19H - Ne promet pas les arrêts majeurs, main régulière (4441 toléré)\+\\
3\c \> Relais pour 3\d\ avec un unicolore \c \+\\
3\d\+\\
3\h \> singleton \s \\
3\s \> singleton \h \\
3NT \> singleton \d \\
4\c \> singleton \d , forcing\-\-\\
3\d \> Texas \h \+\\
3\h \> pas de tenue \s , pas 4\h \\
3\s \> fit \h , ctrl \s , 14h\\
3NT \> tenue \s , pas 4\h \\
4\c \> 4\c , 18-19H\\
4\d \> fit \h , ctrl \d , 14H\\
4\h \> fit \h , 12-13H\\
4\s \> fit \h , 18-19H (BW \h\ ?)\\
4NT \> 18-19H, 4342\-\\
3\h \> Texas \s \+\\
3\s \> pas de tenue \h \\
3NT \> tenue \h , pas 4\s \\
4\c \> 4\c , 18-19H\\
4\d \> fit \s , ctrl \d , 14H\\
4\h \> fit \s , 18-19H (BW \s\ ?)\\
4\s \> fit \s , 12-13H\\
4NT \> 18-19H, 3442 ou 3352\-\\
3\s \> Texas \d\ : 6 \c /4 \d\ TDC\-\\
3\c \> 5\d\ 4\c\ 11-20H\\
3NT \> 15-17H, 4441\+\\
4\c \> Texas \d \\
4\d \> Texas \h \\
4\h \> Texas \s \\
4\s \> Texas \c\ Unicolore TDC. L'ouvreur freine à 4NT sans honneur \c \-
\end{bidtable}

\section{Cachalot}

\subsection{Conditions d'emploi et définition}

\begin{enumerate}
\item Ouverture d’1 \c\ ou d’1 \d\ 

\item Intervention par 1 \d\ ou 1 \h\ 

\item Réponse avec au moins 4 cartes dans une majeure

\item Modification de l’enchère de 1 \s 

\end{enumerate}

Le but est de faire jouer le contrat par l’ouvreur et donc le joueur 
qui est intervenu devra entamer.

\begin{bidtable}
1\c-1\d\\
X \> remplace l'enchère d'1\h \\
1\h \> remplace l'enchère d'1\s \\
1\s \> remplace le contre d'appel\\
1NT \> 8-10H avec tenue\\
2\c \> soutien \c \\
2\d \> Texas 6\h\ toutes zones\\
2\h \> Texas 6\s\ toutes zones\\
2\s \> 5+\c , FM\\
2NT \> Naturel\\
3\c \> mixed raised \c \\
3\d \> 5\h\ + 5\s\ 8-10H\\
3\h \> Barrage \h \\
3\s \> Barrage \s \\
4\d \> 5\h\ + 5\s\ 11-12H
\end{bidtable}

\begin{bidtable}
1\c\d-1\h\\
X \> remplace l'enchère d'1\s \\
1\s \> remplace le contre d'appel\\
1NT \> 8-10H avec tenue\\
2\c\d \> soutien \c /\d \\
2\d\c \> naturel \d /\c \\
2\h \> Texas 6\s\ toutes zones\\
2\s \> 5+ \c /\d , FM\\
2NT \> Naturel\\
3\c\d \> mixed raised \c /\d \\
3\h \> Barrage \s \\
3\s \> L'enchère de 3NT avec Axx à \h 
\end{bidtable}

\begin{bidtable}
1\c \> X\\
XX \> 11H+ punitif\\
1\d \> Texas \h \\
1\h \> Texas \s \\
1\s \> 8+ régulier ou du \d \\
2\c \> 5 \s /4+ \h\ 5-9H\+\\
2NT \> Relay\+\\
3\c \> Mini\+\\
3\d \> Relay\+\\
3\h \> 54\\
3\s \> 55\\
3NT \> 64\-\-\\
3\d \> Max 54\\
3\h \> Max 55\\
3\s \> Max 64\-\-\\
2\d \> Texas 6\h\ toutes zones\\
2\h \> texas 6\s\ toutes zones\\
2\s \> Soutien FM \c \\
2NT \> Soutien limite \c , supporte 3NT\\
3\c \> Barrage \c \\
3\d\h\s \> Splinter, vers la manche
\end{bidtable}

Les développements après 1\d\ et 1\h\ sont ceux du Cachalot.

\begin{bidtable}
1\d \> X\\
XX \> Texas \h \\
1\h \> Texas \s \\
1\s \> 8+ régulier ou soutien \d \\
2\c \> 5 \s /4+ \h\ 5-9H\+\\
2NT \> Relay\+\\
3\c \> Mini\+\\
3\d \> Relay\+\\
3\h \> 54\\
3\s \> 55\\
3NT \> 64\-\-\\
3\d \> Max 54\\
3\h \> Max 55\\
3\s \> Max 64\-\-\\
2\d \> Texas 6\h\ toutes zones\\
2\h \> texas 6\s\ toutes zones\\
2\s \> Soutien FM \d \\
2NT \> Soutien limite \d , supporte 3NT\\
3\d \> Barrage \d \\
3\c\h\s \> Splinter, vers la manche
\end{bidtable}

Passe puis contre : punitif, remplace le surcontre
Passe puis une enchère : 11+, se préparait à contrer punitif, donc deux couleurs 4eme.
Les développements après XX et 1\h\ sont ceux du Cachalot.

\subsection{Attitude de l’ouvreur. Le répondant a 
nommé une majeure}

\begin{enumerate}
\item Avec 4 cartes dans la majeure, l’ouvreur exprime son soutien 
au palier qui convient. (*)

\item Avec 12-14H régulier, l’ouvreur redemande à 1SA, avec ou 
sans tenue, avec 2 ou 3 cartes dans la majeure du répondant 
mais sans 4 cartes à \s\ (comme sans cachalot).

\item La rectification au niveau de 1 montre précisément 3 cartes. 
Elle est forcing, donc illimitée. Ne s’emploie pas si l’ouvreur 
peut répondre 1SA ou 2SA (2SA promet une tenue)

\end{enumerate}

(*)Rem :

\begin{enumerate}
\item 3SA n’est plus un fit 18-19 régulier lorsqu’il y a une intervention. On passe
par un cue-bid puis on fit.

\item Les splinters existent mais attention de bien faire jouer le contrat de la 
bonne main.

\item l’inversée existe mais dénie 3 cartes dans la majeure du répondant.

\end{enumerate}

\begin{bidtable}
1\c-1\d-X-P\\
1\h \> garantit 3 cartes à \h\ et s’utilise dans 3 situations
\end{bidtable}

\begin{enumerate}
\item Avec 4 cartes à \s\ 

\item Avec une distribution irrégulière ou 2326

\item Avec 18-19H régulier sans tenue à \d\ 
Rem : par inférence, 3 \c\ dénie 3 cartes à \h 

\end{enumerate}

\begin{bidtable}
1\c-1\d-1\h-P\\
1\s
\end{bidtable}

\begin{bidtable}
1\c\d-1\h-X-P\\
1\s
\end{bidtable}

Guarantissent 3 cartes à \s\ et s’utilisent dans 2 situations:

\begin{enumerate}
\item Avec une distribution irrégulière ou 3226

\item Avec 18-19H régulier sans tenue dans la couleur d’intervention

\end{enumerate}

\subsection{Développement après une redemande à 1SA}

\begin{enumerate}
\item Le Roudi fixe reste d’application comme sans intervention

\item Le Texas qui rallonge la majeure, rectification obligatoire

\end{enumerate}

Avec 5 \s\ et 4 \h\ nommer les \h\ en Texas qui est faible ou FM. 
Propositionnel, il faut passer par le Roudi fixe.

\begin{bidtable}
1\c-1\d-1\h-P\\
1NT-P-2\d
\end{bidtable}

L’ouvreur ne dira 2 \h\ qu’avec 4 cartes à \h , sinon 2 \s 

Rem : le Texas est également employé sur la redemande de 2SA de l’ouvreur

La répétition de la majeure montre exactement 4 cartes et un soutien 
de 5 cartes forcing de manche dans la mineure d’ouverture

\begin{bidtable}
1\c-1\d-X-P\\
1NT-P-2\h
\end{bidtable}

\begin{bidtable}
1\c-1\d-1\h-P\\
1NT-P-2\s
\end{bidtable}

\subsection{Développements après 1\pdfc-1\pdfd-X-P-1\pdfh}

\subsubsection{Retrouver un soutien \pdfs\ de la bonne main}

\begin{bidtable}
1\c-1\d-X-P\\
1\h--\\
1NT \> 4 cartes à \s , illimité donc forcing\+\\
2\c \> 12-13H le plus souvent 6 cartes\+\\
2\d \> demande des précisions (pas obligatoire)\+\\
2\s \> singleton \d\ (on donne son résidu)\\
2NT \> pas de singleton mais tenue à \d \\
2\h \> pas de singleton 2326\\
3\c \> singleton \s\ sans tenue à \d \-\-\\
2\d \> fit 18-19H reg ou irreg 17H+\\
2\s \> 12-14H\\
2NT \> 18-19H sans tenue\\
3\c \> 14-16H le plus souvent 6 cartes\+\\
3\d \> demande des précisions (pas obligatoire)\+\\
3\s \> résidu donc singleton \d \\
3NT \> tenue à \d\ singleton \s \\
3\h \> singleton \s\ sans tenue à \d \-\-\\
3\s \> 14-16H, court \d \\
4\s \> 16+-18H, court \d \\
3\d \> 18H+, court \d \-
\end{bidtable}

\subsubsection{Le répondant a 5 cartes à \pdfh\ (sans 4 cartes à \pdfs)}

\begin{bidtable}
1\c-1\d-X-P\\
1\h\+\\
2\d \> 14H+\\
2\h \> 6-9H\\
3\h \> 9+-11H\\
4\h \> 11+-13H\-
\end{bidtable}

Rem : L’enchère de 2 \d\ ne garantit pas 5 cartes à \h , on l’apprendra par la suite

\subsubsection{Le répondant exprime son soutien à \pdfc}

Il promet 5 cartes à \c\ et dénie 5 cartes à \h 

\begin{bidtable}
1\c-1\d-X-P\\
1\h\+\\
2\c \> 6-10H\\
3\c \> 11+H FM\+\\
3NT \> tenue à \d\ régulier (donc 4 cartes à \s )\\
3\d \> même distribution mais sans tenue à \d \\
3\s \> résidu donc singleton à \d \\
3\h \> singleton à \s\ (par inférence)\-\-
\end{bidtable}

\subsubsection{Le répondant utilise le relais à 1 \pdfs}

Cette enchère dénie 5 cartes à \h , elle dénie 4 cartes à \s\ et elle dénie un soutien à \c\ faible ou fort.
Elle peut être utilisée avec un soutien limite 10+/11-H à \c\ 
Demande de renseignements sur la force et la distribution de l’ouvreur.

\begin{bidtable}
1\c-1\d-X-P\\
1\h-P-1\s-P--\\
1NT \> 12-14H régulier avec ou sans l’arrêt (et donc il possède 4 \s\ et 3 \h )\\
2\c \> 12-14H 5 ou 6 cartes à \c\ irrégulier (ou 6322)\\
2\d \> FM, 17H+ régulier sans tenue, ou unicolore \c \\
2\s \> Bicolore cher, 17H+\\
3\c \> 6 cartes à \c , 14-16H
\end{bidtable}

\subsubsection{Le répondant utilise le cue-bid à 2 \pdfd}

Cette enchère est FM. Elle dénie 4 cartes à \s\ 
S’utilise soit à la recherche de l’arrêt \d , soit avec 5 cartes à \h\ et 14H+ (voir avant).

\subsection{Développements après 1\pdfc-1\pdfd-1\pdfh-P-1\pdfs}

\subsubsection{Exprimer un soutien \pdfs}

Le répondant possède donc 5 cartes à \s 

\begin{bidtable}
1\c-1\d-1\h-P\\
1\s--\\
2\d \> 14H+\\
2\s \> 6-9H\\
3\s \> 9+-11H\\
4\s \> 11+-13H
\end{bidtable}

\subsubsection{Exprimer son soutien à \pdfc}

Il promet 5 cartes à \c\ et dénie 5 cartes à \s 

\begin{bidtable}
1\c-1\d-1\h-P\\
1\s--\\
2\c \> 6-10H\\
3\c \> 11H+ FM
\end{bidtable}

\subsubsection{Exprimer 4 cartes à \pdfh}

L’enchère de 2 \h\ est naturelle, toute zone et donc forcing pour un tour.
Naturel, forcing illimité 6H+

\begin{bidtable}
1\c-1\d-1\h-P\\
1\s-P-2\h-P--\\
2\s \> 12/14H irrégulier (ou 3226)\\
2NT \> 18/19H régulier sans tenue \d\ (on jouera à \s\ une manche ou plus)\\
3\c \> 15H+ irrégulier\\
3\d \> 4 cartes à \h , 17H+\\
3\h \> 4 cartes à \h , 12/13H\\
4\h \> 4 cartes à \h , 14/16H
\end{bidtable}

\subsubsection{Le relais à 1SA}

Relais forcing un tour sans enchère naturelle. Dénie 5 cartes à \s\ et dénie 4 cartes à \h . 
Si on est FM, on garantit une tenue à \d\ (faible, pas besoin de tenue).

\begin{bidtable}
1\c-1\d-1\h-P\\
1\s-P-1NT-P--\\
2\c \> irreg ,11/14H, 5 ou 6 cartes à \c\ (3226 possible)\\
2\d \> irreg ,17H+ sans 4 cartes à \h \\
2\h \> irreg ,17H+ naturel bicolore cher\\
2\s \> irreg ,15/16H, 5 cartes à \c\ . Non forcing (singleton \d\ ou \h\ mais on devrait pouvoir le deviner)\\
2NT \> reg, 18/19H sans tenue à \d \\
3\c \> irreg ,15/16H, 6 cartes à \c \+\\
3\d \> relai\+\\
3\h \> (résidu) avec singleton \d \\
3\s \> avec singleton \h \-\-
\end{bidtable}

\subsubsection{Le cue-bid à 2 \pdfd}

Soit un soutien \s\ 14H+
Soit une main régulière, FM sans tenue à \d 

\begin{bidtable}
1\c-1\d-1\h-P\\
1\s-P-2\d-P--\\
2\h \> résidu 3 ou 4 cartes à \h\ et donc cours à \d \\
2\s \> montre 6322 sans tenue à \d \\
2NT \> tenue à \d , 6322 ou singleton \h\ possible.\\
3\c \> courte à \h\ sans tenue \d \\
3\d \> 18-19H régulière sans tenue \d 
\end{bidtable}

\subsection{Développements après 1\pdfc-1\pdfh-X-P-1\pdfs}

Peu de modifications par rapport à la séquence
1C-1D-1H-P
1S

Les enchères de soutiens et le relais à 1SA restent inchangés.
Le cue-bid des deux joueurs s’exprime par l’enchère de 2\h\ 
L’enchère de 2\d\ est un bicolore cher chez l’ouvreur et non forcing, canapé avec 4\s\ 5+\d\ chez le 
répondant.

\subsection{Développements après une enchère au niveau de 2}

\begin{bidtable}
1\c-1\d/X-2\d-P--\\
2\h \> Zone 1, une ou deux cartes\\
2\s \> naturel (bicolore cher), singleton \h , F1\\
2NT \> Soutien FM\+\\
3\d \> Re-Texas\+\\
3\h\+\\
4\h \> Deux faible\\
3\s \> TDC\\
3NT \> Zone 2\-\-\-\\
3\c \> 6 cartes, singleton \h \\
3\d \> FM sans soutien \h\ ni enchère naturelle à \s\ (même sur le contre)\\
3\h \> Barrage. Le répondant nomme la manche en zone 2, 3 et +\\
4\h \> gambling en face d'un 2 faible.
\end{bidtable}

\begin{bidtable}
1\c-1\d/X-2\h-P--\\
2\s \> Zone 1, une ou deux cartes\\
2NT \> Soutien FM\+\\
3\h \> Re-Texas\+\\
3\s\+\\
4\s \> Deux faible\\
4\c \> TDC\\
3NT \> Zone 2\-\-\-\\
3\c \> 6 cartes, singleton \s \\
3\d \> FM sans soutien \s\ (même sur le contre)\\
3\h \> naturel (bicolore cher), singleton \s , F1\\
3\s \> Barrage. Le répondant nomme la manche en zone 2, 3 et +\\
4\s \> Gambling en face d'un 2 faible
\end{bidtable}

\begin{bidtable}
1\c\d-1\h-2\h-P--\\
2\s \> rectification pas nécessairement fittée (passe sur une enchère naturelle d'un 2\s )\+\\
3\s \> Et/ou toute autre enchère FM\-\\
2NT \> Soutien FM\+\\
3\h \> Re-Texas\+\\
3\s\+\\
4\s \> Deux faible\\
4\c \> TDC\\
3NT \> Zone 2\-\-\-\\
3\c\d \> NF\\
3\h \> Forcing dans la mineure d'ouverture\\
3\s \> Proposition en face d'un 2 faible\\
4\s \> Gambling en face d'un 2 faible
\end{bidtable}

\begin{bidtable}
1\c-X-2\s-P--\\
2NT \> Naturel, tenue dans les majeures\\
3\c \> Régulier, problème de tenue\\
3\d\h\s \> Singleton
\end{bidtable}

\begin{bidtable}
1\c-1\d-2\s-P--\\
2NT \> Tenue à \d , toutes distributions (singleton possible)\\
3\c \> Régulier, sans tenue \d \\
3\d\h\s \> Singleton sans tenue \d 
\end{bidtable}

\section{Ouverture de 1NT}

L’ouverture de 1SA se fait avec une main régulière de 15-17H. 
Les mains 5m-4\h 22 et 54m22 de 15-17H sont également recommandées pour cette 
ouverture.
Elle peut comporter une majeure cinquième de 15-16H dans une main régulière. Il est 
conseillé alors que le doubleton comporte un honneur. L’autre majeure troisième est un plus 
également

\begin{bidtable}
1NT\+\\
2\c \> Stayman\\
2\d \> Texas \h \\
2\h \> Texas \s \\
2\s \> ambigu : 8-9 H REG ou Texas \c \\
2NT \> Texas \d \\
3\c \> Puppet Stayman\\
3\d \> 6 belles \d , 6-7 H sans singleton, NF.\\
3\h \> 5/4m court \h , FM\\
3\s \> 5/4m court \s , FM\\
3NT \> Pour jouer\\
4\c \> Bic mineur\+\\
4\d \> fit \d \\
4\h\s \> fit \c \\
4NT \> Coup de frein\-\\
4\d \> Bic M : certitude de manche ou de chelem\\
4\h\s \> Pour jouer\\
4NT \> Quantitatif\\
5\c\d \> Pour jouer\-
\end{bidtable}

\subsection{Stayman}

Pas utilisé avec des main régulières, FM, 1 seule majeure 4e.
Utilisé avec une majeure 5e dans la zone 8-9H.

\begin{bidtable}
1NT-2\c\\
2\d \> Pas de majeure 4e\+\\
2\h \> Misère dorée : 8-9H et 5\h \\
2\s \> Misère dorée : 8-9H et 5\s \\
2NT \> 8-9HL, NF\\
3\c\d \> 5\c /\d , FM, problème pour 3SA ou TDC\\
3\h \> 5\s /4\h , FM\\
3\s \> 5\h /4\s , FM\\
4m \> BIC M TDC, court m\+\\
4M \> coup de frein\\
4NT \> BW \h \-\\
4NT \> Quantitatif\-
\end{bidtable}

\begin{bidtable}
2\h\+\\
2\s \> Misère dorée : 8-9H et 5\s \\
2NT \> 8-9HL, NF\\
3\c\d \> 5\c /\d , FM, problème pour 3SA ou TDC\\
3\h \> fit \h\ NF\\
3\s \> Fit \h\ TDC\\
4\c\d \> Splinter\\
4NT \> Quantitatif\-\\
2\s\+\\
2NT \> 8-9HL, NF. L'ouvreur dit 3\h\ max avec 3\h\ pour retrouver le fit en cas de misère dorée\\
3\c\d \> 5\c /\d , FM, problème pour 3SA ou TDC\\
3\h \> fit \s\ TDC\\
3\s \> Fit \s\ NF\\
4\c\d \> Splinter\\
4NT \> Quantitatif\-\\
2NT\+\\
3\c \> Texas \h \\
3\d \> Texas \s \\
3\h \> TDC\\
3\s \> TDC\\
4\c \> Texas \h \\
4\d \> Texas \s \\
4\h\s \> to play\-
\end{bidtable}

\subsection{1NT - 2\pdfd\ : Texas \pdfh}

Remarques :

\begin{itemize}
\item Avec 4 cartes à \h\ et minimum, 4333 on rectifie le plus souvent à 3\h\ 

\item avec 4 cartes à \h\ et maximum, on rectifie à 2NT

\end{itemize}

\begin{bidtable}
1NT-2\d\\
2\h\+\\
2\s \> Bicolore 5\h /5x dans la zone 6-7H pour trouver une manche miracle\+\\
2NT \> Relais\+\\
3\c \> \h /\c \\
3\d \> \h /\d \\
3\h \> \h /\s \-\-\\
2NT \> Relais FM. Soit BIC 5\h /4m, soit Unicolore \h\ TDC\+\\
3\c \> 2 cartes à \h \+\\
3\d \> Singleton \s \+\\
3\h\+\\
3\s \> les \c \\
3NT \> les \d\ NF\\
4\c \> les \d\ Forcing\-\-\\
3\h \> Singleton \c \\
3\s \> singleton \d \\
3NT \> 5\h 4m22 honneurs concentrés\\
4\c\d\h \> contrôle, unicolore \h\ TDC\-\\
3\d \> Fit \h , max\+\\
3NT \> enclenche les contrôles\+\\
3\s4\c\d \> court\-\-\\
3\h \> Fit \h , min\+\\
3NT \> enclenche les contrôles\+\\
3\s4\c\d \> court\-\-\\
3\s \> 5 cartes à \s\ sans 3 cartes à \h \-\\
3m \> BIC 5+\h /5+m, TDC (au moins 11H concentrés). Sans ambition de chelem, le 5/5 sera traité comme un 5/4\\
3\s/4\c/4\d \> Splinter\\
4NT \> Quantitatif 16-17 HL - 5332\-
\end{bidtable}

\subsection{1NT - 2\pdfh\ : Texas \pdfs}

Remarques :

\begin{itemize}
\item avec 4 cartes à \s\ et maximum, on rectifie à 2NT

\item Avec 4 cartes à \s\ et maximum 2SA, ensuite 4 \h\ est Texas.

\end{itemize}

\begin{bidtable}
1NT-2\h\\
2\s\+\\
2NT \> Relais FM. Soit BIC 5\s /4m, soit Unicolore \s\ TDC\+\\
3\c \> 2 cartes à \h \+\\
3\d \> Singleton \h \+\\
3\h\+\\
3\s \> les \c \\
3NT \> les \d\ NF\\
4\c \> les \d\ Forcing\-\-\\
3\h \> Singleton \c \\
3\s \> singleton \d \\
3NT \> 5\s 4m22 honneurs concentrés\\
4\c\d\h \> contrôle, unicolore \s\ TDC\-\\
3\s \> 3 cartes, pas intéressé par m donc sans 4m\-\\
3m \> BIC 5+\s /5+m, TDC (au moins 11H concentrés). Sans ambition de chelem, le 5/5 sera traité comme un 5/4\\
3\h \> BIC majeur fort\\
3\s/4\c/4\d \> Splinter\\
4NT \> Quantitatif 16-17 HL - 5332\-
\end{bidtable}

\subsection{1NT-2S : Texas \pdfc\ ou 8-9H régulier}

\begin{bidtable}
1NT-2\s\\
2NT \> mini\+\\
Pass \> 8-9H réguliers\\
3\c \> SO\\
3\d \> 5/5 mineur FM\\
3\h \> singleton \s \\
3\s \> singleton \h \\
3NT \> singleton \d \\
4\d \> singleton \d\ ambition de chelem\\
4\h \> 6\c /5\h\ NF\\
4\s \> 6\c /5\s\ NF\-\\
3\c \> maxi\+\\
3NT \> 8-9 réguliers\\
3\d \> singleton \d\ ou 5/5 mineur\+\\
3\h \> relais\+\\
3\s \> 5/5 mineur\\
3NT \> singleton \d \\
4\d \> singleton \d\ ambition de chelem\-\-\\
3\h \> singleton \s \\
3\s \> singleton \h \\
4\h \> 6\c /5\h\ NF\\
4\s \> 6\c /5\s\ NF\-
\end{bidtable}

\subsection{1NT-2NT : Texas \pdfd}

Remarque : L'ouvreur peut moduler son soutien, 3\c\ max fitté et 3\d\ RAS.
Il est préférable que l'ouvreur ait une tenue à \c\ lorsqu'il répond 3\c .

\begin{bidtable}
1NT-2NT\\
3\d\+\\
Pass \> 0-6 HL\\
3\h \> singleton \s \\
3\s \> singleton \h \\
3NT \> singleton \c \\
4\c \> TDC \d , singleton \c \\
4\d \> TDC \d\ sans singleton\\
4\h \> 6\c /5\h\ NF\\
4\s \> 6\c /5\s\ NF\-\\
3\c \> maxi\+\\
3\d \> 0-6HL\\
3\h \> singleton \s \\
3\s \> singleton \h \\
3NT \> singleton \c \\
4\c \> TDC \d , singleton \c \\
4\d \> TDC \d\ sans singleton\\
4\h \> 6\c /5\h\ NF\\
4\s \> 6\c /5\s\ NF\-
\end{bidtable}

\subsection{1NT-3\pdfc\ : Puppet Stayman}

Remarque importante

\begin{itemize}
\item Forcing manche

\item Ne s'emploie pas avec des mains comportant un singleton 

\item Ne s'emploie pas avec les 2 maheures 4e

\end{itemize}

\begin{bidtable}
1NT-3\c\\
3\d \> 0, 1 ou 2 majeures 4e\+\\
3\h \> 4\s . L'ouvreur avec 4\s\ fit en disant 3\s\ (3SA = camouflage)\\
3\s \> 4\h \\
3NT \> pas de majeures 4e, pour jouer\\
4\c\d \> 5 cartes 15+, 5332\-\\
3\h \> 5 cartes à \h \+\\
3\s \> ambition de chelem à \h \\
3NT \> pour jouer\\
4\c\d \> 15+, 5332\-\\
3\s \> 5 cartes à \s \+\\
4\h \> ambition de chelem à \s \\
3NT \> pour jouer\\
4\c\d \> 15+, 5332\-
\end{bidtable}

\subsection{1NT-3\pdfh/\pdfs\ : 5/4 mineurs court \pdfh/\pdfs}

Bicolore mineur 5/4 avec une courte dans la couleur annoncée, forcing manche avec au 
moins 9H. L’ouvreur donne 3SA avec au moins 1 arrêt et demi dans la couleur de la courte.
Sinon, il se décrit de manière naturelle. Il peut même fitter le résidu s’il possède cette majeure 5ème.

\section{Défense sur 1SA}

\subsection{Sur le SA fort}

\begin{bidtable}
1NT \> adverse\+\\
X \> Texas \c\ ou Landy (BIC M 4+/4+) si suivi de 2\d \\
\>2C Avec une préférence M (ou sans préférence avec un unicolore \c )\+\\
2\d \> Landy\+\\
2\h/2\s \> Nat NF\\
2NT \> Relais\+\\
3\c \> mini\+\\
3\d \> Relais demandant la M cinquième\\
3M \> Invit\-\\
3\d \> Max 5+\h 4\s \\
3\h \> Max 5+\s 4\h \\
4\c \> 5/5 court \c \\
4\d \> 5/5 court \d \-\\
3\c \> NF\\
3\d \> F1\\
3M \> Invit\-\\
2\d \> Sans préférence M (au mieux 2/2)\+\\
Le \> répondant dit 2\d\ avec un unicolore \d\ ou un semi-bicolore m. L'intervenant passe, nomme une majeure cinquième ou 3\c\ naturel NF. Les autres enchères sont naturelles avec du \c .\-\-\-
\end{bidtable}

\subsection{Sur le SA faible}

Mohan : Le contre montre une bonne main, meilleure que l'ouverture d'1NT adverse (15+ HCP, à adapter aux vulnérabilités et aux atouts de la main). En réveil, la force du contre doit être équivalente.
Ensuite, le partenaire enchérit comme si le partenaire avait ouvert d'1NT (en incluant le Stayman faible ? à discuter).
Après le contre, le camp adverse ne peut jouer un contrat non contré en-dessous de 2\s . De plus, le premier contre par chaque joueur en situation forcing est d'appel.

Sur 1SA - X - 2x :

\begin{itemize}
\item Contre : des points et court dans la couleur adverse

\item Passe : des points, mais 3+ cartes dans la couleur adverse

\item 2y : faible, pas une main pour défendre

\end{itemize}

Autres réponses :

\begin{bidtable}
2\c \> Landy\+\\
2\d \> Choisis ta majeure\+\\
2M\+\\
3\d \> fit la majeure nommée, FM\-\-\\
2\h/2\s \> Forcing passe\\
2NT \> Nat NF\\
3\c/3\d \> Nat NF\-\\
2\d \> Texas \h \+\\
2\s/3\c \> Nat NF\\
2NT \> Invit\-\\
2\h \> Texas \s \\
2\s \> \s\ et une Mineure\+\\
2NT \> Demande la mineure\-\\
2NT \> Les deux mineures
\end{bidtable}

\section{Ouverture de 2\pdfc}

\subsection{Définition}

\begin{itemize}
\item FAIBLE en \d\ (constructif VUL et en 1ère et 2ème NV) et 11-13 en 4ème

\item 22-24 BAL

\item GAMBLING solide en \c\ 

\item FM sauf Bicolores Majeurs

\end{itemize}

\begin{bidtable}
2\c--\+\\
2\d \> relais\+\\
Pass \> FAIBLE \d \\
2\h \> 24+ balancé\\
\>Bicolore avec au moins 4\h \\
\>Unicolore \h\ balancé 22-25\+\\
2\s \> relais\+\\
2NT \> 25H et + (développements comme sur 2N)\\
3\c \> 5+\c , =4\h \\
3\d \> 5+\d , =4\h \\
3\h \> 5+\h , 4+\c \\
3\s \> 5+\h , 4+\d \\
3NT \> Unicolore \h\ Gambling (style régulier avec 6/7\h\ plein et 2 As)\\
4\c\d \> 6\h /5m moins fort que 3\h\ suivi de 4\c\ (5+/5)\\
\>Ex: \s\ 2 \h\ ADV987 \d\ 8 \c\ ARDT2\-\-\\
2\s \> Bicolore avec 4+\s \\
\>Unicolore \s\ balancé 22-25\+\\
2NT \> relais\+\\
3\c \> 5+\c , =4\s \\
3\d \> 5+\d , =4\s \\
3\h \> 5+\s , 4+\c \\
3\s \> 5+\s , 4+\d \\
3NT \> Unicolore \s\ Gambling\\
4\c\d \> 6\s /5m moins fort que 3\s\ suivi de 4\c\ (5+/5)\-\-\\
2NT \> 22+ - 24H\\
3\c\d \> Unicolores ou bicolore mineur (la collante est ambigue)\\
3\h\s \> Unicolores irréguliers\\
3NT \> Gambling \c \\
4\c\d \> Bicolores 6/5m \h\ (6\h -5\c /\d\ + faible que 2\c -2\d -2\h -2\s -4\c /\d )\\
4\h\s \> Bicolores 6/5m \s\ (pareil)\-\\
2\h\s3\c \> F1T (pas forcing en paires !)\\
\>Nouvelle couleur GF\+\\
2NT \> Max misfit\\
3\d \> Min\\
3x/4x \> Fit, Splinters mains faibles\-\\
2NT \> Relais F1 avec espoir de manche en face d’un faible à \d \+\\
3\c \> faible \d\ et une courte\+\\
3\d \> NF courte s’annonce par paliers si max\\
\>(=> Passe si min du faible, annonce la courte par palier si max du faible)\-\\
3\d \> faible \d\ sans courte\\
3\h \> Max pièce \c /\h , pas de courte\+\\
3P \> relais\+\\
3NT \> court \c \\
4\c \> court \h \-\-\\
3\s \> Max force sans courte\\
3NT \> beaux \d\ ARD ou ARVT sans courte\\
4X \> GF\-\\
3\d \> 7-11H (Forcing > 4NT si main forte, mêmes dvpts)\\
3\h\s4\c \> NAT + Fit\-
\end{bidtable}

Avec les mains TRICOLORES, on reparle toujours sur 3NT.

Ex : \s\ - \h\ ADV8 \d\ ARD98 \c\ ARV10
(2\c -2\d -2\h -2\s -3\d -3NT-4\c -...)

\subsection{Après intervention}

\begin{itemize}
\item Contre punitif

\item Toute enchère F1T

\item 2NT R fort

\item Cue demande d'arret

\end{itemize}

\subsection{Après contre}

\begin{itemize}
\item Passe des \c !

\item Surcontre FORT ou UNICOLORE

\item 2\d\ ambigu

\item Nouvelle couleur F1T

\end{itemize}

\section{Ouverture de 2\pdfd}

\subsection{Description}

1/2/3e main

\begin{itemize}
\item Acol fort indeterminé à \c\ ou \d\ 

\item Unicolore 6e à \h\ 5-10 HCP 

\item Bicolore 5/5 \h\ + mineure, 5-10 HCP

\item Bicolore majeur FI ou FM

\end{itemize}

\begin{bidtable}
4e \> main : Acol FI
\end{bidtable}

\subsection{Développements}

1/2/3e main

\begin{bidtable}
2\d\+\\
2\h \> Pour jouer si faible à \h\ (< 15 HCP)\+\\
Pass \> Unicolore \h\ faible ou 5/5 faible avec de beaux \h\ (rare)\\
2\s \> 5+\s /5+\h \\
2NT \> Bicolore 5+/5+ \h /\c\ ou \h /\d \+\\
3\c \> Pour jouer avec des \c\ ou corriger avec des \d \\
3\d \> Pour jouer avec des \d\ ou proposition de manche avec des \c \\
3NT \> Proposition naturelle\\
4/5\c \> Double fit \h /\c\ (4\c\ = NF)\\
4\d/4\h \> Chicane ! - TdC dans la mineure de l'ouvreur\-\\
3\c/3\d \> Acol naturel NF\\
3\h \> 5/5 Majeur FI\\
3\s \> 6\s /4\h\ FI\-\\
2\s \> Relais + misfit \h\ (max Hx), n'envisage normalement pas 4\h\ en face d'un 5/5\+\\
2NT \> Acol Mineur\+\\
3\c \> Relais\+\\
3\d \> Acol \c \\
3\h \> Acol \d \-\\
3x \> Naturel Forcing\-\\
3\c \> 5+ \c \+\\
3\d \> Fitté \c\ encourageant (pas FM)\\
3\h \> Hx\\
3\s \> Naturel, 6\s , forcing\\
3NT \> to play\\
4\c \> Fitté \c\ Fort (forcing)\-\\
3\d \> 5+ \d \+\\
3\h \> Pour jouer\\
3\s \> Nat forcing\\
3NT \> To play\\
4\c \> Limite à \d \\
4\d \> forcing à \d \\
4M \> To play\-\\
3\h \> 6\h\ min\+\\
3\s \> Nat forcing\\
3NT \> To play\\
4\c/4\d \> Cue \h \\
4M \> To play\-\\
3\s \> 6\h\ max\+\\
3NT \> To play\\
4\c/4\d \> Cue \h \\
4M \> To play\-\\
3NT \> ARDxxx \h \\
4\c/4\d \> 1er cue-bid dans un Bic M 5/5 FI ou FM\\
4\h \> 6\s /4\h\ FI\-\\
2NT \> Relais fitté \h\ au moins 3 cartes, pas FM\+\\
3\c \> 5\c\ mini\+\\
3\h \> encourageant NF\-\\
3\d \> 5\d\ mini\+\\
3\h \> encourageant NF\-\\
3\h \> 6\h\ mini\\
3\s \> 6\h\ maxi\\
3NT \> ARDxxx \h \\
4\c/4\d \> Acol \c /\d \\
4\h \> Bic M 5/5 Fi ou FM\\
4\s \> 6\s /4\h\ FI\-\\
3\c/3\d/3\s \> Texas \d , \s\ et \c\ - Ce sont des enchères positives en Texas mais pas nécessairent FM. L'ouvreur se décrit naturellement en privilégiant le fit. La rectification est négative.\+\\
3K \> (sur 3\c ) négatif\\
3\h \> Naturel + fit \d\ (positif)\\
3\s \> Bic M\\
3NT \> Acol \c\ (sans fit \d\ ou sans ambition)\\
4\d \> Fit setting fort (Acol \c\ ou Bic M avec 3\d\ ou Hx)\-\\
3\h \> 3/4 \h\ et 7-13 HCP\+\\
Pass \> L'enchère la plus fréquente de l'ouvreur faible\\
3\s \> Bic M fort\+\\
3NT \> Négatif à \s\ et à \h , minimum\\
4\c/4\d \> Naturel F1\-\\
3NT \> Montre un Acol mineur - NF\\
4\h \> Montre une main faible avec un supplément de distribution (6/5, 6/4, ...)\\
4\c/4\d \> Acol mineur très excentré\-\\
3NT \> Artificiel, propose de jouer 3NT en face d'un Acol mineur ou 4\h\ en face d'une version faible.\\
4\c/4\d \> Naturel, bon fit \h , 10+ HCP, l'ouvreur peut décider d'être compétitif au palier supérieur\\
4\h \> Pour jouer 4\h\ si ouvreur faible mais du jeu pour supporter le chelem si ouvreur fort\\
4\s \> Pour jouer en face d'un 2 faible ou au moins 5\h\ en face d'un jeu fort\-
\end{bidtable}

\begin{bidtable}
4e \> main :
\end{bidtable}

\begin{bidtable}
2\d\+\\
2\h \> Relais\\
2\s/3\c/3\d \> Naturel, 2 gros H\-
\end{bidtable}

\subsection{Après intervention}

\begin{bidtable}
2\d\+\\
Dbl\+\\
Pass \> pas 3 cartes à \h \+\\
Pass\+\\
ReDbl \> 6\h\ (Principe : faire jouer le répondant le + souvent possible !)\\
2\h \> 5\h\ + 5m\\
2\s \> Bic M fort\\
3\c \> Acol \c \-\-\\
ReDbl \> Unicolore autonome soit misfit \h\ soit semi-fit \h\ avec tolérance pour les mineures\+\\
Pass\+\\
2\h \> Version faible\\
2\s \> Bic M Fort\-\-\\
2h \> 2+ \h \\
2\s/3\c/3\d \> Naturel F1 mais pas auto-forcing\\
3\h \> permet de dire 4\h \+\\
Dbl\+\\
Pass \> 6\s\ Toujours selon le même principe\\
ReDbl \> 5\s\ + 5m\\
2NT \> Acol \h\ + arrêt \s \\
3\c \> 5\s\ + 6\c\ faible\\
3\d \> 5\s\ + 6\d\ faible\\
3\h \> Acol \h\ pour jouer à la couleur (peu/pas d'arrêt \s )\-\-\\
2NT \> fitté\\
3X/4\h \> System on\-\\
Pass\+\\
2\h\+\\
Dbl\+\\
Pass \> 6\h\ Principe : Faire jouer le répondant le plus souvent possible\\
ReDbl \> 5\h /5m\\
2\s \> Bic M ou 6\s /4\h\ FI (NF)\\
2NT \> 5\h\ + 6\c /6\d \\
3\c/3\d \> Acol\\
3M \> Bic M FM (et ne veut pas passer sur 2Cx)\-\-\\
2\s\+\\
Dbl\+\\
System \> on\-\-\-\-
\end{bidtable}

\section{Ouverture de 2\pdfh}

\subsection{Description}

1/2/3e main

\begin{itemize}
\item Acol fort indeterminé à \h\ 

\item Unicolore 6e à \s\ 5-10 HCP 

\item Bicolore 5/5 \s\ + mineure, 5-10 HCP

\end{itemize}

\begin{bidtable}
4e \> main : Unicolore 6e \h\ 10-13 HCP
\end{bidtable}

\subsection{Développements}

\begin{bidtable}
2\h\+\\
Pass\\
2\s \> Pour jouer si faible à \s\ (< 15 HCP)\+\\
Pass \> Unicolore \s\ faible ou 5/5 faible avec de beaux \s\ (rare)\\
2NT \> Bicolore 5+/5+ \s /\c\ ou \s /\d \+\\
3\c \> Pour jouer avec des \c\ ou corriger avec des \d \\
3\d \> Pour jouer avec des \d\ ou proposition de manche avec des \c \\
3NT \> Proposition naturelle\\
4/5\c \> Double fit \s /\c \\
4\d/4\h/4\s \> Chicane - TdC\-\\
3\c/3\d \> 2e couleur ou résidu 3e si la main possède une courte\\
3\h \> 6\s\ tendance 6322\-\\
2NT \> Relais fort\+\\
3\c \> 5\c\ mini\+\\
3\d \> Tendance naturel FM\\
3\h \> Fit \s\ Forcing\\
3\s \> Fit \s\ Invit\\
4\c \> Fit \c\ Forcing\-\\
3\d \> 5\d\ mini\+\\
3\h \> Fit \s\ Forcing\\
3\s \> Fit \s\ Invit\\
4\c \> Fit \d\ + contrôle \c \\
4\d \> Fit \d\ Forcing, pas de ctrl \c \-\\
3\h \> 6\s \+\\
3\s \> NF (oriente la main)\-\\
3\s \> 5\c\ maxi\\
3NT \> 5\d\ maxi\\
4\c/4\d \> Acol \h\ - Premier contrôle disponible\\
4\h \> Acol \h \\
4\s \> Acol \h\ (cue-bid)\-\\
3\c/3\d/3\h \> Texas \d , \h\ et \c\ - Ce sont des enchères positives en Texas mais pas nécessairent FM. L'ouvreur se décrit naturellement en privilégiant le fit. La rectification est négative.\\
3\s \> 3/4 \s\ et 7-13 HCP\+\\
Pass \> L'enchère la plus fréquente de l'ouvreur faible\\
3NT \> Proposition de jouer 3NT sur base d'un Acol \h \\
4\s \> Montre une main faible avec un supplément de distribution (6/5, 6/4, ...)\\
4\c/4\d \> Tentative à base d'un Acol \h\ (généralement avec une courte)\-\\
3NT \> Pour jouer\\
4\c/4\d \> Naturel, bon fit \s , 10+ HCP\\
4\h \> !! Pour jouer en face d'un Acol Fort ou 4P en face du 2 faible.\\
4\s \> Pour jouer en face d'un 2 faible ou au moins 5\h\ en face d'un jeu fort\-
\end{bidtable}

\subsection{Après intervention}

\begin{bidtable}
2\h\+\\
Dbl\+\\
Pass \> pas 3 cartes à \s \+\\
Pass\+\\
ReDbl \> 6\s\ (Principe : faire jouer le répondant le + souvent possible !)\\
2\s \> 5\s\ + 5m\-\-\\
ReDbl \> Unicolore autonome soit misfit \s\ soit semi-fit \s\ avec tolérance pour les mineures\+\\
Pass\+\\
Pass \> Acol \h \\
2\s \> 6\s \\
2NT/3\c \> 5\c /5\d\ pour faire jouer le partenaire dans son unicolore.\-\-\\
2\s \> au moins 2\s \+\\
Dbl\+\\
Pass \> 6\s\ Toujours selon le même principe\\
ReDbl \> 5\s\ + 5m\\
2NT \> Acol \h\ + arrêt \s \\
3\c \> 5\s\ + 6\c\ faible\\
3\d \> 5\s\ + 6\d\ faible\\
3\h \> Acol \h\ pour jouer à la couleur (peu/pas d'arrêt \s )\-\-\\
2NT \> fitté\\
3X/4\h \> System on\-\-
\end{bidtable}

\section{Ouverture de 2\pdfs}

\subsection{Description}

1/2/3e main

\begin{itemize}
\item Acol fort indeterminé à \s\ 

\item Bicolore mineur 5+/5+, 5-10 HCP

\end{itemize}

\begin{bidtable}
4e \> main : Unicolore 6e \s\ 10-13 HCP
\end{bidtable}

\subsection{Développements}

\begin{bidtable}
2\s\+\\
Pass\\
2NT \> Relais\+\\
3\c \> Bic mineur minimum\+\\
Pass\\
3\d \> Pour jouer\\
3\h/3\s \> Naturel F1\\
3NT \> Pour jouer\-\\
3\d \> Bic mineur maximum\+\\
3\h/3\s \> Naturel F1\\
3NT \> Pour jouer\-\\
3\h \> Acol \s\ avec une courte indéterminée\+\\
3\s \> Fit \s \\
3NT \> Forcing\\
4\c/4\d/4\h \> Naturel (a priori misfit \s )\+\\
4\s \> Pas fitté\\
4NT \> KBB\-\\
4\s \> Pour jouer\\
4NT \> KBB\-\\
3\s \> Acol \s\ 6322 min (18-19)\+\\
3NT \> Forcing (a priori misfit \s )\\
4\c/4\d/4\h \> Naturel (a priori misfit \s , recherche de fit)\+\\
4\s \> Pas fitté\\
4NT \> KBB\-\\
4\s \> Forcing\\
4NT \> KBB à \s \-\\
3NT \> Acol \s\ 6322 max (20-21)\+\\
4\c/4\d/4\h \> Naturel (a priori misfit \s , recherche de fit)\+\\
4\s \> Pas fitté\\
4NT \> KBB\-\\
4\s \> Forcing\\
4NT \> KBB à \s \-\\
4\c \> 1156\\
4\d \> 1165\\
4\h/4\s \> 6/5 + chicane\-\\
3\c \> Pour jouer\+\\
3\d \> Acol \s\ + couleur \d\ 3+ cartes\\
3\h \> Acol \s\ + couleur \d\ 3+ cartes (forte probabilité de courte \s )\\
3\s \> Acol \s\ NF\\
3NT \> 6\s\ (3-2-2) 20-21\-\\
3\d \> Pour jouer\+\\
Cfr \> 3\c \-\\
3\h \> Naturel NF\\
3\s \> Naturel NF\\
3NT \> Pour jouer\\
4\c \> "Barrage" 4/5 cartes 7-12 HCP\\
4\d \> "Barrage" 4/5 cartes 7-12 HCP\\
4\h \> Pour jouer (aussi en face de la main faible)\\
4\s \> Pour jouer en face de la main forte\-
\end{bidtable}

\subsection{Après intervention}

\begin{bidtable}
2\s\+\\
Dbl\+\\
Pass \> Propose d'en rester là sur base de sa propre couleur (faible + \s )\\
ReDbl \> Propose de jouer 2\s\ XX (plus punitif éventuel ultérieur) sauf si ouvreur a 1- \s \\
2NT \> Game try en mineure ou SA, l'ouvreur développe comme sur séquence de base\\
3\c/3\d \> préférence\\
3\h \> Naturel NF\\
3\s \> N'existe pas\\
3NT \> Naturel\\
4\c/4\d \> Barrage\\
4\h \> Pour jouer\-\\
Pass\+\\
3\c/3\d\+\\
Dbl\+\\
Pass \> Main faible, le répondant peut annoncer une longue Maj ou XX = Bic M\\
ReDbl \> Acol \s\ mini\\
3\h \> 6\s\ + 3\h \\
3\s \> Acol \s\ max\-\\
Pass-Pass\+\\
Dbl\+\\
ReDbl \> Misfit total, Bic M\\
3\h/3\s \> Longue personnelle\-\-\-\-\\
3X/4X\+\\
Pass \> Forcing pour la main forte\\
Dbl \> Punitif face à la main faible\\
4\c/4\d \> Compétitif, fit face à la main faible\\
Couleur \> Naturel F1\\
3NT/4NT \> Naturel face à la main faible\\
5\c/5K \> Pour jouer\-\-
\end{bidtable}

\section{Ouverture de 2NT}

Cette ouverture peut contenir une majeure 5e.

\begin{bidtable}
2NT---\\
3\c \> Puppet Stayman\\
3\d \> Texas \h \\
3\h \> Texas \s \\
3\s \> Texas pour 3N\\
3NT \> 5\s\ et 4\h \\
4\c \> Bicolore mineur TDC\\
4\d \> Bicolore majeur, limité à la manche ou certitude de chelem
\end{bidtable}

\subsection{Le Puppet Stayman}

\begin{bidtable}
2NT-3\c \> ---\\
3\d \> 4\h\ et/ou 4\s \+\\
3\h \> 4 cartes à \s \+\\
3\s \> fit \s \+\\
4X \> envie de chelem\-\\
3NT \> 4 cartes à \h\ (il est donc nécessaire de passer par 3\h\ avec\\
\>les deux majeures 4e et des envies de chelem)\\
\>4X fit avec envie de chelem\-\\
3\s \> 4 cartes à \h\ et pas 4 cartes à \s \\
3NT \> 3-\h\ et 3-\s \\
4\d \> 4\h\ et 4\s\ sans envie de chelem\-\\
3\h \> 5\h \\
3\s \> 5\s \\
3NT \> 3-\h\ et 3-\s 
\end{bidtable}

\subsection{Les Texas}

\begin{bidtable}
2NT---\\
3\d\+\\
3\h \> 2\h \\
3\s \> 5\s\ et 2\h \+\\
4\c/4\d \> ctrl \c /\d\ et fit \s \\
4\h/4\s \> Arrêt\-\\
3NT \> 3+\h\ et ctrl \s \\
4\c \> 3+\h\ et ctrl \c\ (sans ctrl \s )\\
4\d \> 3+\h\ et ctrl \d \\
4\h \> 3+\h\ et tous les ctrl\-
\end{bidtable}

\begin{bidtable}
3\h\+\\
3\s \> 2\s \\
3NT \> 5\h\ et 2\s \+\\
4\c/4\d \> ctrl \c /\d\ et fit \h \\
4\h/4\s \> Arrêt\-\\
4\c \> 3+\s\ et ctrl \c \\
4\d \> 3+\s\ et ctrl \d\ (sans ctrl \c )\\
4\h \> 3+\s\ et ctrl \h \\
4\s \> 3+\s\ et tous les ctrl\-
\end{bidtable}

\begin{bidtable}
3\s\+\\
3NT\+\\
4\c/4\d \> Envie ou certitude de chelem à \c /\d . L'ouvreur décourage par 4SA.\\
4\h/S \> Envie de chelem à \h /\s . L'ouvreur passe ou demande les clés.\\
\>Avec certitude de chelem, il faut passer par un texas.\-\-
\end{bidtable}

\subsection{Autres réponses}

\begin{bidtable}
4\c\+\\
4\d \> fit \d \\
4\h \> fit \c\ et ctrl \h \\
4\s \> fit \c , ctrl \s , pas de ctrl \h \-
\end{bidtable}

\begin{bidtable}
4\d \> L'ouvreur choisit sa majeure
\end{bidtable}

\begin{bidtable}
4NT \> Minimum, l'ouvreur passe.\\
\>Maximum, les réponses sont les suivantes :\+\\
5\c \> 1 ou 4 As\\
5\d \> 0 ou 3 As\\
5\h \> 2 As et 5C\\
5\s \> 2 As et 5D\\
5NT \> 2 As et pas de min 5e\-
\end{bidtable}

\section{Défense sur les ouvertures Multi}

Principe général sur les ouvertures de type Multifonction :

\begin{itemize}
\item X montre soit 12-15 semi-balancé, soit 16+ si irrégulier ou 20+ si balancé.

\item Passe puis X est d'appel.

\item L'annonce de la couleur adverse (si une seule couleur connue) est d'appel (style tricolore, 4441/5431) ou bicolore cher).

\item 2X est naturel avec l'ouverture.

\item 3X est naturel avec 7-8 levées.

\item 2SA est balancé 16-19 avec arrêt(s).

\item Ensuite, sur un X d'appel et une enchère adverse en 3è, X est d'appel (du saut et transformable).

\end{itemize}

En pratique :

\subsection{2\pdfc\ multi (les deux M faibles ou fort)}

\begin{itemize}
\item X 12-15 semi-balancé ou fort irrégulier ou 20-22 balancé

\item 2\d\ naturel

\item 2M semi-saturel (4+ cartes, voire 3 belles) 13-17

\item 2NT semi-balancé 16-19 avec arrêts majeurs

\item 3X NAT 7/8 levées

\item 3SA pour jouer

\item Passe puis 2SA avec les mineures (5+ 4+)

\item Passe puis 3SA bicolore m très distribué

\end{itemize}

\subsection{2\pdfd\ Multi (unicolore M faible ou fort)}

\begin{itemize}
\item X 12-15 semi balancé, en principe 3+ 3+ dans les M, ou fort

\item 2M naturel

\item 2NT semi-BAL 16-19 avec les arrêts majeurs

\item 3X naturel, 7-8 levées

\item Passe puis X d'appel

\item Passe puis 2NT avec les mineures (5+ 4+)

\end{itemize}

Après 2\d\ - 2M, en 4è :

\begin{itemize}
\item X est d'appel sur la M et 3\c\ d'appel sur la M'

\item 2NT ou 3SA naturel

\item 3M bicolore m 

\item 4m bicolore mM'

\end{itemize}

\section{Les barrages et ouvertures à haut palier}

\subsection{Namyats 4\pdfc/4\pdfd}

Cette ouverture montre une couleur \h /\s\ 8e fermée ou 7e fermée avec un as.
La rectification à la manche est un arrêt.

L'enchère juste au-dessus montre un intérêt pour le chelem et demande à l'ouvreur de:

\begin{itemize}
\item Nommer la manche sans as annexe

\item Nommer son as annexe

\end{itemize}

\subsection{3SA}

Mineure 8e fermée sans rien à côté. 
4/5\c \d\ PoC

\section{Enchères après passe}

\begin{bidtable}
1\s\+\\
1NT \> 6-11H, misfit\\
2\c \> Drury, 3\s\ 10-12 DH/ 4\s\ jeu plat, (2\s\ avec GH et 11 points H)\+\\
2\d/2\h \> naturel avec une ouverture correcte\\
2\s \> Main n'ayant pas d'espoir de manche en face d'un passe initial\\
2NT \> 14+ 17-, jeu régulier avec des arrêts un peu partout\\
3\c/3\d/3\h \> Beau 5-5, 16+\\
3\s \> 6\s , 16/18\\
3NT \> 17+ 19, jeu régulier\-\\
2\d/2\h \> Naturel. Belle couleur souvent 6ème, 8-11 H.\\
2\s \> 6-9HL\\
2NT \> Fit 4e et un singleton\+\\
3\c \> Relais\+\\
3\d \> Singleton \d \\
3\h \> Singleton \h\ (ou \c\ si fit \h )\\
3\s \> Singleton \c\ (ou \s\ si fit \h )\-\-\\
3\c/3\d/3\h \> Rencontre : 5 belles cartes, 4+ \s , 9/11H : L'ouvreur revient à la majeur au palier qu'il souhaite ou nomme un contrôle si TDC. Si double fit, BW 6 clés ?\\
3\s \> Barrage\\
4\c/4\d/4\h \> Splinter avec 5 atouts\\
4\s \> Barrage\-
\end{bidtable}

\subsection{Après intervention}

Le contre remplace le Drury (en plus de garder sa signification habituelle). Le cue-bid montre 4 cartes.

Les enchères de rencontre restent d'application, le Splinter s'effectue uniquement dans la couleur d'intervention.

\section{Enchères de chelem}

L'enchère de 4NT est un Blackwood à 5 clés incluant le roi d'atout. L'atout est la couleur explicitement ou implicitement (via cue-bid) fittée ou à défaut la dernière couleur nommée naturellement

\begin{bidtable}
4NT\+\\
5\c \> 4 ou 1 clé(s)\\
5\d \> 3 ou 0 clé(s)\\
5\h \> 2 clés sans dame d'atout\\
5\s \> 2 clés avec dame d'atout\\
5NT \> nombre impair de clé et une chicane\\
6x \> nombre pair de clés et chicane x utile (ou chicane au-dessus si x = atout)\-
\end{bidtable}

Sur une réponse 5m, le palier immédiatement supérieur en dehors de l'atout est un relais pour la demande de la dame d'atout et des rois spécifiques.
Avec la dame d'atout, on nomme économiquement la couleur du premier roi ou 5SA déniant un roi.

Le retour à l'atout au palier le plus économique dénie la dame d'atout. Ensuite, le palier au-dessus demande de nommer les rois dans un ordre économique.

Une autre enchère non-ambigüe explore pour le grand chelem, demandant un complément.

Blackwood d'exclusion : Double saut anormal avec fit précisé ! 
Réponses : 1, 3/0, 2, 2+Q.

Une réponse de 5NT avec une zone initiale de 2 pts (par exemple l'ouverture de 2NT) propose de découvrir un fit 4-4 au palier de 6.

En cas de bicolore et sans fit précisé par manque de place, le Blackwood est à 6 clés :

\begin{itemize}
\item 5H : pas de dame

\item 5S : la dame la moins chère

\item 5NT : la plus chère

\item 6C : les deux

\end{itemize}

Après le BW, 5NT est une demande de roi :

\begin{bidtable}
5NT\+\\
Retour \> à l'atout : pas de roi.\\
6X \> : Roi X ou les deux autres.\\
6NT \> : Tous les rois.\-
\end{bidtable}

Sur un 4NT quantitatif :

\begin{bidtable}
4NT\+\\
Passe \> mini\\
5\c \> 4 ou 1 As\\
5\d \> 3 ou 0 As\\
5\h \> 2 As et 5\c \\
5\s \> 2 As et 5\d \-
\end{bidtable}

D0PI/R0PI

Après intervention adverse sur le Blackwood :
Double 30 Passe 41n ensuite 2 et 2+Q

Après le contre d'une couleur du Blackwood :
Redouble 30 Passe 41, ensuite 2 et 2+Q

SI intervention adverse au-dessus de 5 de notre couleur :
DEP0/REP0 - Double Even, Pass Odd.

Intervention adverse par contre sur les cue-bids :

\begin{itemize}
\item Passe : sans contrôle

\item XX : contrôle du premier tour

\item Autre : contrôle du second tour (retour à l'atout = second tour sans autre cue).

\end{itemize}

Intervention adverse en couleur sur les cue-bids :

\begin{itemize}
\item X : punitif

\item Passe : sans contrôle

\item Autre : souvent court dans la couleur

\end{itemize}

Intervention adverse par contre sur un cue-bid dans une courte connue (Splinter par exemple) :

\begin{itemize}
\item Passe : pas de contrôle

\item XX : contrôle du premier tour

\item Autre : cue-bid + contrôle du second tour

\end{itemize}

\section{Enchères compétitives}

\section{Jeu de la carte}

\subsection{L'entame}

\subsubsection{Entame à SA}

Principe de la quatrième meilleure :

\begin{itemize}
\item Quatrième carte d'une couleur contenant un honneur (le 10 n'est pas considéré comme un honneur, sauf 109xx ou 10 cinquième. )

\item Tête de séquence d'une séquence d'au moins trois cartes dont un honneur

\item Séquence brisée, le plus gros honneur du début de séquence (A/R/D 109xx - le 10, jusqu'au 9 - D98x le 9.)

\item Sur entame du Roi, si le mort est court - appel par les petites.

\item Top of nothing : \textbf{x}xx - la plus grosse; x \texttt{x} x \emph{x} 

\item Séquence = 3 cartes avec au moins 1 honneur (àpd T).

\item Si le partenaire ouvre d'1\c\ et que les adversaires jouent à NT, 4e meilleure même à trèfle.

\end{itemize}

Sur l'entame : 
-	Roi : Déblocage, Parité. Si le mort est court - appel par les petites
-	As : on appelle avec des couleurs longues (le partenaire est court) et pas des honneurs
-	Dame : Attitude (on dit qu’on aime quand on a des honneurs car son entame provient souvent d’une longue)
-   Valet : Parité (on confirmera par la suite siu intérêt)

\subsubsection{Entame à la couleur}

\begin{itemize}
\item Tête de séquence de 2 cartes ou séquence brisée.

\item Parité : 
    Dans 4 cartes, la 2e si pas d'honneur, la 3e si un honneur.
    Dans 6 cartes, le 3e si pas d'honneur, la 5e si un honneur.
Jeu des honneurs : 
    En 3e position dans la levée, on met l'honneur le moins élevé.
    Si le partenaire entame le Roi - la Dame promet le Valet.

\end{itemize}

\subsubsection{Dans le cours du jeu}

Switch petit prometteur - en fonction du mort.
QUand le partenaire entame d'un honneur et que le mort est court - italien.
Confirmation d'entame par les petites - si nécessaire.
Première défausse Lavinthal.

\end{document}
