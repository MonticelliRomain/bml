\documentclass[a4paper]{article}
\usepackage[T1]{fontenc}
\usepackage[utf8]{inputenc}
\usepackage{newcent}
\usepackage{helvet}
\usepackage{graphicx}
\usepackage[pdftex, pdfborder={0 0 0}]{hyperref}
\frenchspacing

\include{bml}
\title{Systeme Competition B et C}
\author{Clovis}
\begin{document}
\maketitle
\tableofcontents

\section{Ouverture de 2\pdfc}

\subsection{Définition}

\begin{itemize}
\item FAIBLE en \d\ (constructif VUL et en 1ère et 2ème NV) et 11-13 en 4ème

\item 22-24 BAL

\item GAMBLING solide en \c\ 

\item FM sauf Bicolores Majeurs

\end{itemize}

\begin{bidtable}
2\c--\+\\
2\d \> relais\+\\
Pass \> FAIBLE \d \\
2\h \> 24+ balancé\\
\>Bicolore avec au moins 4\h \\
\>Unicolore \h\ balancé 22-25\+\\
2\s \> relais\+\\
2NT \> 25H et + (développements comme sur 2N)\\
3\c \> 5+\c , =4\h \\
3\d \> 5+\d , =4\h \\
3\h \> 5+\h , 4+\c \\
3\s \> 5+\h , 4+\d \\
3NT \> Unicolore \h\ Gambling (style régulier avec 6/7\h\ plein et 2 As)\\
4\c\d \> 6\h /5m moins fort que 3\h\ suivi de 4\c\ (5+/5)\\
\>Ex: \s\ 2 \h\ ADV987 \d\ 8 \c\ ARDT2\-\-\\
2\s \> Bicolore avec 4+\s \\
\>Unicolore \s\ balancé 22-25\+\\
2NT \> relais\+\\
3\c \> 5+\c , =4\s \\
3\d \> 5+\d , =4\s \\
3\h \> 5+\s , 4+\c \\
3\s \> 5+\s , 4+\d \\
3NT \> Unicolore \s\ Gambling\\
4\c\d \> 6\s /5m moins fort que 3\s\ suivi de 4\c\ (5+/5)\-\-\\
2NT \> 22+ - 24H\\
3\c\d \> Unicolores ou bicolore mineur (la collante est ambigue)\\
3\h\s \> Unicolores irréguliers\\
3NT \> Gambling \c \\
4\c\d \> Bicolores 6/5m \h\ (6\h -5\c /\d\ + faible que 2\c -2\d -2\h -2\s -4\c /\d )\\
4\h\s \> Bicolores 6/5m \s\ (pareil)\-\\
2\h\s3\c \> F1T (pas forcing en paires !)\\
\>Nouvelle couleur GF\+\\
2NT \> Max misfit\\
3\d \> Min\\
3x/4x \> Fit, Splinters mains faibles\-\\
2NT \> Relais F1 avec espoir de manche en face d’un faible à \d \+\\
3\c \> faible \d\ et une courte\+\\
3\d \> NF courte s’annonce par paliers si max\\
\>(=> Passe si min du faible, annonce la courte par palier si max du faible)\-\\
3\d \> faible \d\ sans courte\\
3\h \> Max pièce \c /\h , pas de courte\+\\
3P \> relais\+\\
3NT \> court \c \\
4\c \> court \h \-\-\\
3\s \> Max force sans courte\\
3NT \> beaux \d\ ARD ou ARVT sans courte\\
4X \> GF\-\\
3\d \> 7-11H (Forcing > 4NT si main forte, mêmes dvpts)\\
3\h\s4\c \> NAT + Fit\-
\end{bidtable}

Avec les mains TRICOLORES, on reparle toujours sur 3NT.

Ex : \s\ - \h\ ADV8 \d\ ARD98 \c\ ARV10
(2\c -2\d -2\h -2\s -3\d -3NT-4\c -...)

\subsection{Après intervention}

\begin{itemize}
\item Contre punitif

\item Toute enchère F1T

\item 2NT R fort

\item Cue demande d'arret

\end{itemize}

\subsection{Après contre}

\begin{itemize}
\item Passe des \c !

\item Surcontre FORT ou UNICOLORE

\item 2\d\ ambigu

\item Nouvelle couleur F1T

\end{itemize}

\section{Ouverture de 2\pdfd}

\subsection{Description}

Quatrième main : Idem mais zone faible 9-13.
1/2/3e main :

\begin{itemize}
\item Acol fort indeterminé à \c\ ou \d\ 

\item Unicolore 6e à \h\ 5-10 HCP 

\item Bicolore 5/5 \h\ + mineure, 5-10 HCP

\item Bicolore majeur FI ou FM

\end{itemize}

\subsection{Développements}

1/2/3e main

\begin{bidtable}
2\d-2\h \> Pour jouer si faible à \h\ (< 15 HCP)\+\\
Pass \> Unicolore \h\ faible ou 5/5 faible avec de beaux \h\ (rare)\\
2\s \> 4+\s /4+\h\ GF\+\\
2NT \> Relais\+\\
3\c \> Montre des \h\ + longs (5+)\+\\
3\d \> relais\+\\
3\h \> Résidu \c\ ou solo chelem 6+\\
3\s \> Résidu \d \\
3NT \> 4522\\
4\c\d \> 4504/4540\\
4\h \> 6/5 mini\-\-\\
3\d \> 5+ \s \+\\
3\h \> relais\+\\
3\s \> Résidu \c \\
3NT \> résidu \d \-\-\\
3\h \> 5+/5\\
3\s \> 6/4\\
3NT \> 5422/5431 22H+\\
4\c\d \> 5404/5440\\
4\h \> 6/5 Mini\-\\
3\c\d \> 6 très belles cartes\\
3\h\s \> 4 cartes 5DH+\\
3NT \> 5/5 ou 6/5 m faible\\
4\c\d \> Chicane\\
4\h\s \> Naturel < 5DH\-\\
2NT \> Bicolore 5+/5+ \h /\c\ ou \h /\d \+\\
3\c \> PoC\\
3\d \> PoC\\
3\s \> Relais le + fort\\
3NT \> Proposition naturelle\\
4/5\c \> PoC\\
4\h \> Chicane \h\ - TdC dans la mineure de l'ouvreur\-\\
3\c/3\d \> Acol naturel NF\\
3\h \> 5/5 Majeur FI\\
3\s \> 6\s /4\h\ FI\\
3NT \> Gambling à \d \-
\end{bidtable}

\begin{bidtable}
2\d\+\\
2\s \> Relais + misfit \h\ (max Hx), n'envisage normalement pas 4\h\ en face d'un 5/5\+\\
2NT \> 5+\c\ mini\+\\
3\c \> NF encourageant\-\\
3\c \> 5+\d\ mini\+\\
3\d \> idem\-\\
3\d \> 6\h \+\\
3\h \> NF encourageant\\
3\s \> 6 cartes à \s\ fort\-\\
3\h \> 5+\c\ max\\
3\s \> 5+/4+ M FM\\
3NT \> 5+\d\ max\\
4\c/4\d \> Acol \c /\d \\
4\h \> 5/5 M FI\\
4\s \> 6\s /4\h\ FI\-\\
2NT \> Relais fitté \h\ au moins 3 cartes, pas FM\+\\
3\c \> 5\h /5\d \+\\
3\h \> encourageant NF\-\\
3\d \> 6\h \+\\
3\h \> encourageant NF\-\\
3\h \> 5\h /5\c\ mini NF\\
3\s \> 4+/4+ M FI/FM\\
3NT \> 5\h /5\c\ max\\
4\c/4\d \> Acol \c /\d \\
4\h \> Bic M 5/5 FI\\
4\s \> 6\s /4\h\ FI\-\\
3\c/3\d/3\s \> Texas \d , \s\ et \c\ - Ce sont des enchères positives en Texas mais pas nécessairement FM.\\
\>L'ouvreur se décrit naturellement en privilégiant le fit. La rectification est négative.\+\\
3\d \> (sur 3\c ) négatif\\
3\h \> Naturel + fit \d\ (positif)\\
3\s \> Bic M\\
3NT \> Acol \c\ (sans fit \d\ ou sans ambition)\\
4\d \> Fit setting fort (Acol \c\ ou Bic M avec 3\d\ ou Hx)\-\\
3\h \> 3/4 \h\ et 7-13 HCP\+\\
Pass \> L'enchère la plus fréquente de l'ouvreur faible\\
3\s \> Bic M fort\+\\
3NT \> Négatif à \s\ et à \h , minimum\\
4\c/4\d \> Naturel F1\-\\
3NT \> Montre un Acol mineur - NF\\
4\h \> Montre une main faible avec un supplément de distribution (6/5, 6/4, ...)\\
4\c/4\d \> Acol mineur très excentré\-\\
3\s \> Pour jouer 3NT en face d'un Acol à \c\ (! double sens)\\
3NT \> Pour jouer en face d'un FI à \d \\
4\c/4\d \> Naturel, bon fit \h , 10+ HCP, l'ouvreur peut décider d'être compétitif au palier supérieur\\
4\h \> Pour jouer 4\h\ si ouvreur faible mais du jeu pour supporter le chelem si ouvreur fort\\
4\s \> Pour jouer en face d'un 2 faible ou au moins 5\h\ en face d'un jeu fort\-
\end{bidtable}

\subsection{Après intervention}

\begin{bidtable}
2\d\+\\
(Dbl)\+\\
Pass \> si tu es 5\h /5\d , je veux jouer 2\d \+\\
(Pass)\+\\
Rdbl \> 6\h \\
2\h \> 5\h /5\c \\
2\s \> Bic M fort\\
2NT \> 5\h /6\c \\
3\c \> Acol \c \-\-\\
2\h \> si tu es 5\h /5\d , je veux jouer 2\h \+\\
(Dbl)\+\\
Pass \> 6\h\ (Principe : faire jouer le répondant le + souvent possible !)\\
2\h \> 5\h /5\c \\
2\s \> Bic M fort\\
2NT \> 5\h /6\c \\
3\c \> Acol \c \-\-\\
Rdbl \> Unicolore autonome soit misfit \h\ soit semi-fit \h\ avec tolérance pour les mineures\+\\
(Pass)\+\\
2\h \> Version faible\\
2\s \> Bic M Fort\-\-\\
2\s/3\c/3\d \> Naturel F1 mais pas auto-forcing\\
3\h \> permet de dire 4\h\ (suivre le bon sens)\\
2NT \> fitté\\
3X/4\h \> System on\-\\
(Pass)\+\\
2\h\+\\
(Dbl)\+\\
Pass \> 6\h\ Principe : Faire jouer le répondant le plus souvent possible\\
Rdbl \> 5\h /5m\\
2\s \> Bic M ou 6\s /4\h\ FI (NF)\\
2NT \> 5\h\ + 6\c /6\d \\
3\c/3\d \> Acol\\
3M \> Bic M FM (et ne veut pas passer sur 2Cx)\-\-\\
2\s\+\\
(Dbl)\+\\
System \> on\-\-\-\-
\end{bidtable}

\section{Ouverture de 2\pdfh}

\subsection{Description}

1/2/3e main

\begin{itemize}
\item Acol fort indeterminé à \h\ 

\item Unicolore 6e à \s\ 5-10 HCP 

\item Bicolore 5/5 \s\ + mineure, 5-10 HCP

\end{itemize}

\begin{bidtable}
4e \> main : Ideam mais zone faible 9-13 HCP
\end{bidtable}

\subsection{Développements}

\begin{bidtable}
2\h\+\\
Pass\\
2\s \> Pour jouer si faible à \s\ (< 15 HCP)\+\\
Pass \> Unicolore \s\ faible ou 5/5 faible avec de beaux \s\ (rare)\\
2NT \> Bicolore 5+/5+ \s /\c\ ou \s /\d \+\\
3\c \> Pour jouer avec des \c\ ou corriger avec des \d \\
3\d \> Pour jouer avec des \d\ ou proposition de manche avec des \c \\
3NT \> Proposition naturelle\\
4/5\c \> PoC\\
4\h/4\s \> Chicane - TdC\-\\
3\c\d\s \> 2e couleur ou résidu 3e si la main possède une courte\\
3\h \> 6\h\ tendance 6322 18-20\\
3NT \> 6\h\ tendance régulier 21-22\-\\
2NT \> Relais fort\+\\
3\c \> 5\c\ mini\+\\
3\d \> Tendance naturel FM\\
3\h \> Fit \s\ Forcing\\
3\s \> Fit \s\ Invit\\
4\c \> Fit \c\ Forcing\-\\
3\d \> 5\d\ mini\+\\
3\h \> Fit \s\ Forcing\\
3\s \> Fit \s\ Invit\\
4\c \> Fit \d\ + contrôle \c \\
4\d \> Fit \d\ Forcing, pas de ctrl \c \-\\
3\h \> 6\s \+\\
3\s \> NF (oriente la main)\-\\
3\s \> 5\c\ maxi\\
3NT \> 5\d\ maxi\\
4\c/4\d \> Acol \h\ - Premier contrôle disponible\\
4\h \> Acol \h \-\\
3\c/3\d/3\h \> Texas \d , \h\ et \c\ - Ce sont des enchères positives en Texas mais pas nécessairent FM.\\
\>L'ouvreur se décrit naturellement en privilégiant le fit. La rectification est négative.\\
3\s \> 3/4 \s\ et 7-13 HCP\+\\
Pass \> L'enchère la plus fréquente de l'ouvreur faible\\
3NT \> Proposition de jouer 3NT sur base d'un Acol \h \\
4\s \> Montre une main faible avec un supplément de distribution (6/5, 6/4, ...)\\
4\c/4\d \> Tentative à base d'un Acol \h\ (généralement avec une courte)\-\\
3NT \> Pour jouer\\
4\c/4\d \> Naturel, bon fit \s , 10+ HCP\\
4\h \> !! Pour jouer en face d'un Acol Fort ou 4\s\ en face du 2 faible.\\
4\s \> Pour jouer en face d'un 2 faible ou au moins 5\h\ en face d'un jeu fort\-
\end{bidtable}

\subsection{Après intervention}

\begin{bidtable}
2\h\+\\
(Dbl)\+\\
Pass \> 2- \s \+\\
(Pass)\+\\
Rdbl \> 6\s\ (Principe : faire jouer le répondant le + souvent possible !)\\
2\s \> 5\s\ + 5m\-\-\\
Rdbl \> Unicolore ou fort\+\\
(Pass)\+\\
Pass \> Acol \h \\
2\s \> 6\s \\
2NT/3\c \> 5\c /5\d\ pour faire jouer le partenaire dans son unicolore.\-\-\\
2\s \> au moins 2\s \+\\
(Dbl)\+\\
Pass \> 6\s\ Toujours selon le même principe\\
Rdbl \> 5\s\ + 5m  (pas obligé sur Dbl mais bien 2\h -P-2\s )\\
2NT \> Acol \h\ + arrêt \s \\
3\c \> 5\s\ + 6\c\ faible\\
3\d \> 5\s\ + 6\d\ faible\\
3\h \> Acol \h\ pour jouer à la couleur (peu/pas d'arrêt \s )\-\-\\
2NT \> fitté\\
3X/4\h \> System on\-\-
\end{bidtable}

\section{Ouverture de 2\pdfs}

\subsection{Description}

Quatrième main : Bicolore mineur 5+/5+ 9-13 HCP ou Acol fort indeterminé à \s . 1/2/3e main

\begin{itemize}
\item Acol fort indeterminé à \s\ 

\item Bicolore mineur 5+/5+, 5-10 HCP

\end{itemize}

\subsection{Développements}

\begin{bidtable}
2\s\+\\
Pass\\
2NT \> Relais\+\\
3\c \> Bic mineur minimum\+\\
Pass\\
3\d \> Pour jouer\\
3\h/3\s \> Naturel F1\\
3NT \> Pour jouer\-\\
3\d \> Bic mineur maximum\+\\
3\h/3\s \> Naturel F1\\
3NT \> Pour jouer\-\\
3\h \> 3 cartes d'un honneur (dans la version forte)\+\\
3\s \> Fit \s \\
3NT \> Forcing\\
4\c/4\d/4\h \> Naturel (a priori misfit \s )\+\\
4\s \> Pas fitté\\
4NT \> KBB\-\\
4\s \> Pour jouer\\
4NT \> KBB\-\\
3\s \> Acol \s\ 6322 min (18-19)\+\\
3NT \> Forcing (a priori misfit \s )\\
4\c/4\d/4\h \> Naturel (a priori misfit \s , recherche de fit)\+\\
4\s \> Pas fitté\\
4NT \> KBB\-\\
4\s \> Forcing\\
4NT \> KBB à \s \-\\
3NT \> Acol \s\ 6322 max (20-21)\+\\
4\c/4\d/4\h \> Naturel (a priori misfit \s , recherche de fit)\+\\
4\s \> Pas fitté\\
4NT \> KBB\-\\
4\s \> Forcing\\
4NT \> KBB à \s \-\\
4\c \> 1156\\
4\d \> 1165\\
4\h/4\s \> 6/5 + chicane\-\\
3\c \> Pour jouer\+\\
3\d \> Acol \s\ + couleur \d\ 3+ cartes\\
3\h \> Acol \s\ + couleur \h\ 3+ cartes (forte probabilité de courte \s )\\
3\s \> Acol \s\ NF\\
3NT \> 6\s\ (3-2-2) 20-21\-\\
3\d \> Pour jouer\+\\
Cfr \> 3\c \-\\
3\h \> Naturel NF\\
3\s \> Naturel NF\\
3NT \> Pour jouer\\
4\c \> "Barrage" 4/5 cartes 7-12 HCP\\
4\d \> "Barrage" 4/5 cartes 7-12 HCP\\
4\h \> Pour jouer (aussi en face de la main faible)\\
4\s \> Pour jouer en face de la main forte\-
\end{bidtable}

\subsection{Après intervention}

\begin{bidtable}
2\s\+\\
(Dbl)\+\\
Pass \> Propose d'en rester là sur base de sa propre couleur (faible + \s )\\
Rdbl \> Propose de jouer 2\s\ XX (plus punitif éventuel ultérieur) sauf si ouvreur a 1- \s \\
2NT \> Game try en mineure ou SA, l'ouvreur développe comme sur séquence de base\\
3\c/3\d \> préférence\\
3\h \> Naturel NF\\
3\s \> N'existe pas\\
3NT \> Naturel\\
4\c/4\d \> Barrage\\
4\h \> Pour jouer\-\\
(Pass)\+\\
3\c/3\d\+\\
(Dbl)\+\\
Pass \> Main faible, le répondant peut annoncer une longue Maj\\
\>ou XX = Bic M \\
Rdbl \> Acol \s\ mini\\
3\h \> 6\s\ + 3\h \\
3\s \> Acol \s\ max\-\\
(Pass)-Pass\+\\
(Dbl)\+\\
Rdbl \> Misfit total, Bic M\\
3\h/3\s \> Longue personnelle\-\-\-\-\\
3X/4X\+\\
Pass \> Forcing pour la main forte\\
Dbl \> Punitif face à la main faible\\
4\c/4\d \> Compétitif, fit face à la main faible\\
Couleur \> Naturel F1\\
3NT/4NT \> Naturel face à la main faible\\
5\c/5\d \> Pour jouer\-\-
\end{bidtable}

\end{document}
